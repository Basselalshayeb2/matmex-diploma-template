% !TEX TS-program = xelatex
% !BIB program = bibtex
% !TeX spellcheck = ru_RU

% About magic macros see also
% https://tex.stackexchange.com/questions/78101/

% По умолчанию используется шрифт 14 размера.
% Если Вы не влезаете в лимит страниц и нужен 12-й шрифт,
% то уберите опцию [14pt]

\documentclass[14pt, russian]{matmex-diploma-custom}


% !TeX spellcheck = ru_RU
% !TEX root = vkr.tex
% Опциональные добавления используемых пакетов. Вполне может быть, что они вам не понадобятся, но в шаблоне приведены примеры их использования.
\usepackage{tikz} % Мощный пакет для создание рисунков, однако может очень сильно замедлять компиляцию
\usetikzlibrary{decorations.pathreplacing,calc,shapes,positioning,tikzmark}

% Библиотека для TikZ, которая генерирует отдельные файлы для каждого рисунка
% Позволяет ускорить компиляцию, однако имеет свои ограничения
% Например, ломает пример выделения кода в листинге из шаблона
% \usetikzlibrary{external}
% \tikzexternalize[prefix=figures/]

\newcounter{tmkcount}

\tikzset{
    use tikzmark/.style={
            remember picture,
            overlay,
            execute at end picture={
                    \stepcounter{tmkcount}
                },
        },
    tikzmark suffix={-\thetmkcount}
}

\usepackage{booktabs} % Пакет для верстки "более книжных" таблиц, вполне годится для оформления результатов
% В шаблоне есть команда \multirowcell, которой нужен этот пакет.
\usepackage{multirow}
\usepackage{siunitx} % для таблиц с единицами измерений

% Для названий стоит использовать \textsc{}
\newcommand{\OCaml}{\textsc{OCaml}}
\newcommand{\miniKanren}{\textsc{miniKanren}}
\newcommand{\BibTeX}{\textsc{BibTeX}}
\newcommand{\vsharp}{\textsc{V$\sharp$}}
\newcommand{\fsharp}{\textsc{F$\sharp$}}
\newcommand{\csharp}{\textsc{C\#}}
\newcommand{\GitHub}{\textsc{GitHub}}
\newcommand{\SMT}{\textsc{SMT}}

\definecolor{eclipseGreen}{RGB}{63,127,95}
% \lstdefinelanguage{ocaml}{
% keywords={@type, function, fun, let, in, match, with, when, class, type,
% nonrec, object, method, of, rec, repeat, until, while, not, do, done, as, val, inherit, and,
% new, module, sig, deriving, datatype, struct, if, then, else, open, private, virtual, include, success, failure,
% lazy, assert, true, false, end},
% sensitive=true,
% commentstyle=\small\itshape\ttfamily,
% keywordstyle=\ttfamily\bfseries, %\underbar,
% identifierstyle=\ttfamily,
% basewidth={0.5em,0.5em},
% columns=fixed,
% fontadjust=true,
% literate={->}{{$\to$}}3 {===}{{$\equiv$}}1 {=/=}{{$\not\equiv$}}1 {|>}{{$\triangleright$}}3 {\\/}{{$\vee$}}2 {/\\}{{$\wedge$}}2 {>=}{{$\ge$}}1 {<=}{{$\le$}} 1,
% morecomment=[s]{(*}{*)}
% }

\makeatletter
\@ifclassloaded{beamer}{
    %%% Обязательные пакеты
    %% Beamer
    \usepackage{beamerthemesplit}
    \usetheme{SPbGU}
    \beamertemplatenavigationsymbolsempty
    \usepackage{appendixnumberbeamer}

    %% Локализация
    \usepackage{fontspec}
    \setmainfont{CMU Serif}
    \setsansfont{CMU Sans Serif}
    \setmonofont{CMU Typewriter Text}
    %\setmonofont{Fira Code}[Contextuals=Alternate,Scale=0.9]
    %\setmonofont{Inconsolata}
    \usepackage{polyglossia}
    \setmainlanguage{russian}
    \setotherlanguage{english}

    %% Графика
    \usepackage{pdfpages} % Позволяет вставлять многостраничные pdf документы в текст

    % Математические окружения с русским названием
    \newtheorem{rutheorem}{Теорема}
    \newtheorem{ruproof}{Доказательство}
    \newtheorem{rudefinition}{Определение}
    \newtheorem{rulemma}{Лемма}
    \usepackage{fancyvrb}
}
{}
\makeatother

\usepackage[autostyle]{csquotes} % Правильные кавычки в зависимости от языка
\usepackage{totcount}
\usepackage{setspace}
\usepackage{amsmath, amsfonts, amssymb, amsthm, mathtools} % "Адекватная" работа с математикой в LaTeX



\begin{document}

% !TeX spellcheck = ru_RU
% !TEX root = vkr.tex


%% Если что-то забыли, при компиляции будут ошибки Undefined control sequence \my@title@<что забыли>@ru
%% Если англоязычная титульная страница не нужна, то ее можно просто удалить.
\filltitle{ru}{
    date = {27 марта 2025},
    %% Актуально только для курсовых/практик. ВКР защищаются не на кафедре а в ГЭК по направлению,
    %%   и к моменту защиты вы будете уже не в группе.
    chair              = {Программная инженерия},
    group              = {24.М71-мм},
    %
    %% Макрос filltitle ненавидит пустые строки, поэтому обязателен хотя бы символ комментария на строке
    %% Актуально всем.
    title              = {Структурная декомпозиция (WBS) и план управления проектом},
    %
    %% Здесь указывается тип работы. Возможные значения:
    %%   production - производственная практика;
    %%   coursework - отчёт по курсовой работе (ОБРАТИТЕ ВНИМАНИЕ, у техпрога и ПИ нет курсовых, только практики);
    %%   practice - отчёт по учебной практике;
    %%   prediploma - отчёт по преддипломной практике;
    %%   master - ВКР магистра;
    %%   bachelor - ВКР бакалавра.
    type               = {groupwork},
    %
    %% Здесь указывается вид работы. От вида работы зависят критерии оценивания.
    %%   solution - «Решение». Обучающемуся поручили найти способ решения проблемы в области разработки программного обеспечения или теоретической информатики с учётом набора ограничений.
    %%   experiment - «Эксперимент». Обучающемуся поручили изучить возможности, достоинства и недостатки новой технологии, платформы, языка и т. д. на примере какой-то задачи.
    %%   production - «Производственное задание». Автору поручили реализовать потенциально полезное программное обеспечение.
    %%   comparison - «Сравнение». Обучающемуся поручили сравнить несколько существующих продуктов и/или подходов.
    %%   theoretical - «Теоретическое исследование». Автору поручили доказать какое-то утверждение, исследовать свойства алгоритма и т.п., при этом не требуя написания кода.
    kind               = {none},
    %
    author             = {Juniors},
    %
    %% Актуально только для ВКР. Указывается код и название направления подготовки. Типичные примеры:
    %%   02.03.03 \enquote{Математическое обеспечение и администрирование информационных систем}
    %%   02.04.03 \enquote{Математическое обеспечение и администрирование информационных систем}
    %%   09.03.04 \enquote{Программная инженерия}
    %%   09.04.04 \enquote{Программная инженерия}
    %% Те, что с 03 в середине --- бакалавриат, с 04 --- магистратура.
    specialty          = {09.04.04 \enquote{Математическое обеспечение и администрирование информационных систем}},
    %
    %% Актуально только для ВКР. Указывается шифр и название образовательной программы. Типичные примеры:
    %%   СВ.5162.2020 \enquote{Технологии программирования}
    %%   СВ.5080.2020 \enquote{Программная инженерия}
    %%   ВМ.5665.2022 \enquote{Математическое обеспечение и администрирование информационных систем}
    %%   ВМ.5666.2022 \enquote{Программная инженерия}
    %% Шифр и название программы можно посмотреть в учебном плане, по которому вы учитесь.
    %% СВ.* --- бакалавриат, ВМ.* --- магистратура. В конце --- год поступления (не обязательно ваш, если вы были в академе/вылетали).
    programme          = {ВМ.5666.2024 \enquote{Технологии программирования}},
    %
    %% Актуально всем.
    %% Должно умещаться в одну строчку, допускается использование сокращений, но без переусердствования,
    %% короткая строка с большим количеством сокращений выглядит странно
    %supervisorPosition = {проф. кафeдры системного программирования, д.ф.-м.н.,}, % Терехов А. Н.
    %supervisorPosition = {ст. преподаватель кафедры ИАС, к.~ф.-м.~н. (если есть),}, % Смирнов К. К.
    supervisorPosition = {},
    supervisor         = {},
    teacher            = {Тимохин Д. В.}
    %
    %% Актуально только для практик и курсовых. Если консультанта нет или он совпадает с научником, закомментировать или удалить вовсе.
    % consultantPosition = {должность, ООО \enquote{Место работы}, степень  (если есть),},
    % consultant         = {Консультант~К.~К.},
    %
    %% Актуально только для ВКР.
    % reviewerPosition   = {должность, ООО \enquote{Место работы}, степень (если есть),},
    % reviewer           = {Рецензент~Р.~Р.},
}

% Английский титульник нужен только для ВКР, остальные виды работ могут его смело игнорировать.
\filltitle{en}{
    chair              = {Advisor's chair},
    group              = {ХХ.BХХ-mm},
    title              = {Template for SPbU qualification works},
    type               = {bachelor},
    author             = {FirstName Surname},
    %
    %% Possible choices:
    %%   02.03.03 \foreignquote{english}{Software and Administration of Information Systems}
    %%   02.04.03 \foreignquote{english}{Software and Administration of Information Systems}
    %%   09.03.04 \foreignquote{english}{Software Engineering}
    %%   09.04.04 \foreignquote{english}{Software Engineering}
    %% Те, что с 03 в середине --- бакалавриат, с 04 --- магистратура.
    specialty          = {02.03.03 \foreignquote{english}{Software and Administration of Information Systems}},
    %
    %% Possible choices:
    %%   СВ.5162.2020 \foreignquote{english}{Programming Technologies}
    %%   СВ.5080.2020 \foreignquote{english}{Software Engineering}
    %%   ВМ.5665.2022 \foreignquote{english}{Software and Administration of Information Systems}
    %%   ВМ.5666.2022 \foreignquote{english}{Software Engineering}
    programme          = {СВ.5162.2020 \foreignquote{english}{Programming Technologies}},
    %
    %% Note that common title translations are:
    %%   кандидат наук --- C.Sc. (NOT Ph.D.)
    %%   доктор ... наук --- Sc.D.
    %%   доцент --- docent (NOT assistant/associate prof.)
    %%   профессор --- prof.
    supervisorPosition = {Sc.D, prof.},
    supervisor         = {S.S. Supervisor},
    %
    consultantPosition = {position at \foreignquote{english}{Company}, degree if present},
    consultant         = {C.C. Consultant},
    %
    reviewerPosition   = {position at \foreignquote{english}{Company}, degree if present},
    reviewer           = {R.R. Reviewer},
}

\maketitle
\section{О документе}
Этот документ представляет собой комплексный план управления проектом MyGym, направленный на создание инновационной платформы для автоматизации управления фитнес-клубами. Он
включает основные этапы, ресурсы, риски, критерии приемки и структуру команды.

\section{Определение проекта}
\subsection{Идентификация проекта}
\begin{itemize}
\item \textbf{Название:} MyGym
\item \textbf{Заказчик:} Владелец сети фитнес-клубов "FitRU"
\item \textbf{Исполнитель:} Команда MyGym
\item \textbf{Дата старта:} 1 апреля 2025 года
\item \textbf{Дата завершения MVP:} 30 июня 2025 года
\end{itemize}

\subsection{Постановка задачи}
\subsubsection{Цели проекта}
\begin{itemize}
\item Разработка платформы для управления фитнес-клубами
\item Автоматизация процессов записи клиентов, платежей, аналитики
\item Упрощение взаимодействия с клиентами через мобильное приложение
\end{itemize}

\subsubsection{Рамки проекта}
\begin{itemize}
\item MVP будет включать управление членством, бронирование, платежи
\item В будущих версиях могут появиться рекомендации по тренировкам на основе искусственного интеллекта
\end{itemize}

\subsubsection{Структурная декомпозиция работ (WBS)}
\begin{enumerate}
    \item Планирование
    \begin{enumerate}
        \item Анализ требований заказчика;
        \item Исследование рынка и анализ конкурентов;
        \item Выбор технических ресурсов, инструментов реализации и распределение персонала;
        \item Создание документации проекта.
    \end{enumerate}

    \item Размещение:
    \begin{enumerate}
        \item Выбор необходимого оборудования;
        \item Закупка оборудования;
        \item Закрепление этапа в документации.
    \end{enumerate}

    \item Разработка:
    \begin{enumerate}
        \item Создание базы данных проекта:
        \begin{enumerate}
            \item Проектирование концептуальной модели;
            \item Проектирование логической модели;
            \item Проектирование физической модели;
            \item Закрепление этапа разработки документацией.
        \end{enumerate}
        \item Разработка бэкенда:
        \begin{enumerate}
            \item Проработка логики работы бэкенда;
            \item Создание MVP;
            \item Закрепление этапа разработки документацией.
        \end{enumerate}
        \item Разработка фронтэнд-части:
        \begin{enumerate}
            \item Создание плана внешнего вида с техническими элементами;
            \item Реализация плана;
            \item Закрепление этапа разработки документацией.
        \end{enumerate}
        \item Внедрение системы оплаты:
        \begin{enumerate}
            \item Выбор платёжной системы;
            \item Настройка API;
            \item Закрепление этапа разработки документацией.
        \end{enumerate}
        \item Внедрение систем регистрации, аутентификации и безопасности:
        \begin{enumerate}
            \item Выбор метода аутентификации;
            \item Реализация выбранного метода аутентификации;
            \item Закрепление этапа разработки документацией.
        \end{enumerate}
    \end{enumerate}

    \item Проведение тестирования:
    \begin{enumerate}
        \item Выбор методики тестирования;
        \item Проведение тестирования;
        \item Обновление и дополнение документации проекта.
    \end{enumerate}

    \item Поддержка:
    \begin{enumerate}
        \item Мониторинг и анализ;
        \item Реакция на отклик пользователей;
        \item Внедрение изменений;
        \item Внедрение изменений в документацию.
    \end{enumerate}

    \item Маркетинг собственной фирмы:
    \begin{enumerate}
        \item Выбор аудитории;
        \item Проектирование стратегии;
        \item Закупка рекламы.
    \end{enumerate}

\end{enumerate}

\subsubsection{Сроки проекта}
\begin{enumerate}
    \item \textbf{MVP:} 2 месяца
    \item \textbf{Полнофункциональная платформа:} 1 год
\end{enumerate}

\subsubsection{Основные фазы}
\begin{enumerate}
    \item \textbf{Планирование и исследование:} 15 дней
    \item \textbf{Разработка:} 1.5 месяц
    \item \textbf{Тестирование:} 1 месяц
    \item \textbf{Развертывание и маркетинг:} текущее
\end{enumerate}

\subsubsection{Основные ограничения}
\begin{enumerate}
    \item \textbf{Бюджет:} Ограниченный
    \item \textbf{Ресурсы:} Небольшая команда разработчиков
    \item \textbf{Технические ограничения:} Должно работать на нескольких устройствах
\end{enumerate}

\subsection{Критерии и процедуры приемки}
\begin{enumerate}
    \item Система должна пройти функциональные тесты и приемочные испытания для пользователей.
\end{enumerate}


\section{Организационная структура проекта}
\subsection{Заинтересованные лица}
\begin{itemize}
\item \textbf{Заказчик:} Владелец сети "FitRU"
\item \textbf{Руководитель проекта:} Кисельков Денис Андреевич
\item \textbf{Пользователи:} Тренажерные залы, тренеры и спортсмены
\end{itemize}

\subsection{Роли участников}
\begin{itemize}
\item \textbf{Технический руководитель:} Малыгин Даниил Александрович
\item \textbf{Маркетолог:} Макаров Павел Михайлович
\item \textbf{Разработчики:} Альшаеб Басель, Малыгин Даниил Александрович
\end{itemize}

\subsection{Ресурсы проекта}
\subsubsection{Финансовые ресурсы}
\begin{itemize}
\item Бюджет: 5 000 000 рублей
\item Основные статьи расходов:
\begin{itemize}
\item Разработка: 2 500 000 рублей
\item Инфраструктура: 1 000 000 рублей
\item Маркетинг: 1 000 000 рублей
\item Непредвиденные расходы: 500 000 рублей
\end{itemize}
\end{itemize}

\subsubsection{Технические ресурсы}
\begin{itemize}
\item Серверное оборудование
\item Облачные сервисы
\item Инструменты аналитики
\end{itemize}

\section{Управление конфигурацией}
% Todo: fill later

\section{Основные процедуры управления проектом}
\begin{itemize}
    \item \textbf{Методы планирования:} Методология Agile с еженедельными спринтами
    \item \textbf{Процедура управления изменениями:} Любое изменение функций должно быть одобрено руководителем проекта и заинтересованными сторонами
    \item \textbf{Процедуры анализа хода проекта:} Еженедельные отчеты о ходе работ
\end{itemize}

\section{Управление коммуникациями}
\begin{itemize}
\item \textbf{Участники коммуникации:} Вся команда со всеми участниками проекта
\item \textbf{Каналы связи:} Slack, Zoom, Email
\item \textbf{Регулярность встреч:} раз в неделю
\item \textbf{Отчеты:} раз в две недели
\end{itemize}

\section{План разработки}
% Gantt Chart to be added

\section{Тестирование в проекте}
% Gantt Chart to be added

\section{Основные риски и управление ими}
\begin{itemize}
\item Задержка сроков --- \textbf{Митигирование:} внедрение Agile-методов
\item Проблемы с финансированием --- \textbf{Митигирование:} привлечение инвесторов
\item Технические ошибки --- \textbf{Митигирование:} тестирование на ранних этапах
\end{itemize}

\section{График передачи результатов}
\begin{itemize}
\item 1 апреля 2025 --- Начало проекта
\item 1 июня 2025 --- Первая версия
\item 30 июня 2025 --- MVP
\end{itemize}


\section{Заключение}
Проект MyGym направлен на создание удобной платформы для автоматизации управления фитнес-клубами. В документе изложены основные аспекты проекта, структура команды, сроки и критерии приемки. План будет обновляться по мере выполнения этапов.

\end{document}
