% !TEX TS-program = xelatex
% !BIB program = bibtex
% !TeX spellcheck = ru_RU

% About magic macros see also
% https://tex.stackexchange.com/questions/78101/

% По умолчанию используется шрифт 14 размера.
% Если Вы не влезаете в лимит страниц и нужен 12-й шрифт,
% то уберите опцию [14pt]

\documentclass[14pt, russian]{matmex-diploma-custom}


\input{preamble.tex}

\begin{document}

\input{junior_02_title.tex}
\maketitle
\section{О документе}
Этот документ представляет собой комплексный план управления проектом MyGym, направленный на создание инновационной платформы для автоматизации управления фитнес-клубами. Он
включает основные этапы, ресурсы, риски, критерии приемки и структуру команды.

\section{Определение проекта}
\subsection{Идентификация проекта}
\begin{itemize}
\item \textbf{Название:} MyGym
\item \textbf{Заказчик:} Владелец сети фитнес-клубов "FitRU"
\item \textbf{Исполнитель:} Команда MyGym
\item \textbf{Дата старта:} 1 апреля 2025 года
\item \textbf{Дата завершения MVP:} 30 июня 2025 года
\end{itemize}

\subsection{Постановка задачи}
\subsubsection{Цели проекта}
\begin{itemize}
\item Разработка платформы для управления фитнес-клубами
\item Автоматизация процессов записи клиентов, платежей, аналитики
\item Упрощение взаимодействия с клиентами через мобильное приложение
\end{itemize}

\subsubsection{Рамки проекта}
\begin{itemize}
\item MVP будет включать управление членством, бронирование, платежи
\item В будущих версиях могут появиться рекомендации по тренировкам на основе искусственного интеллекта
\end{itemize}

\subsubsection{Структурная декомпозиция работ (WBS)}
\begin{enumerate}
    \item Планирование
    \begin{enumerate}
        \item Анализ требований заказчика;
        \item Исследование рынка и анализ конкурентов;
        \item Выбор технических ресурсов, инструментов реализации и распределение персонала;
        \item Создание документации проекта.
    \end{enumerate}

    \item Размещение:
    \begin{enumerate}
        \item Выбор необходимого оборудования;
        \item Закупка оборудования;
        \item Закрепление этапа в документации.
    \end{enumerate}

    \item Разработка:
    \begin{enumerate}
        \item Создание базы данных проекта:
        \begin{enumerate}
            \item Проектирование концептуальной модели;
            \item Проектирование логической модели;
            \item Проектирование физической модели;
            \item Закрепление этапа разработки документацией.
        \end{enumerate}
        \item Разработка бэкенда:
        \begin{enumerate}
            \item Проработка логики работы бэкенда;
            \item Создание MVP;
            \item Закрепление этапа разработки документацией.
        \end{enumerate}
        \item Разработка фронтэнд-части:
        \begin{enumerate}
            \item Создание плана внешнего вида с техническими элементами;
            \item Реализация плана;
            \item Закрепление этапа разработки документацией.
        \end{enumerate}
        \item Внедрение системы оплаты:
        \begin{enumerate}
            \item Выбор платёжной системы;
            \item Настройка API;
            \item Закрепление этапа разработки документацией.
        \end{enumerate}
        \item Внедрение систем регистрации, аутентификации и безопасности:
        \begin{enumerate}
            \item Выбор метода аутентификации;
            \item Реализация выбранного метода аутентификации;
            \item Закрепление этапа разработки документацией.
        \end{enumerate}
    \end{enumerate}

    \item Проведение тестирования:
    \begin{enumerate}
        \item Выбор методики тестирования;
        \item Проведение тестирования;
        \item Обновление и дополнение документации проекта.
    \end{enumerate}

    \item Поддержка:
    \begin{enumerate}
        \item Мониторинг и анализ;
        \item Реакция на отклик пользователей;
        \item Внедрение изменений;
        \item Внедрение изменений в документацию.
    \end{enumerate}

    \item Маркетинг собственной фирмы:
    \begin{enumerate}
        \item Выбор аудитории;
        \item Проектирование стратегии;
        \item Закупка рекламы.
    \end{enumerate}

\end{enumerate}

\subsubsection{Сроки проекта}
\begin{enumerate}
    \item \textbf{MVP:} 2 месяца
    \item \textbf{Полнофункциональная платформа:} 1 год
\end{enumerate}

\subsubsection{Основные фазы}
\begin{enumerate}
    \item \textbf{Планирование и исследование:} 15 дней
    \item \textbf{Разработка:} 1.5 месяц
    \item \textbf{Тестирование:} 1 месяц
    \item \textbf{Развертывание и маркетинг:} текущее
\end{enumerate}

\subsubsection{Основные ограничения}
\begin{enumerate}
    \item \textbf{Бюджет:} Ограниченный
    \item \textbf{Ресурсы:} Небольшая команда разработчиков
    \item \textbf{Технические ограничения:} Должно работать на нескольких устройствах
\end{enumerate}

\subsection{Критерии и процедуры приемки}
\begin{enumerate}
    \item Система должна пройти функциональные тесты и приемочные испытания для пользователей.
\end{enumerate}


\section{Организационная структура проекта}
\subsection{Заинтересованные лица}
\begin{itemize}
\item \textbf{Заказчик:} Владелец сети "FitRU"
\item \textbf{Руководитель проекта:} Кисельков Денис Андреевич
\item \textbf{Пользователи:} Тренажерные залы, тренеры и спортсмены
\end{itemize}

\subsection{Роли участников}
\begin{itemize}
\item \textbf{Технический руководитель:} Малыгин Даниил Александрович
\item \textbf{Маркетолог:} Макаров Павел Михайлович
\item \textbf{Разработчики:} Альшаеб Басель, Малыгин Даниил Александрович
\end{itemize}

\subsection{Ресурсы проекта}
\subsubsection{Финансовые ресурсы}
\begin{itemize}
\item Бюджет: 5 000 000 рублей
\item Основные статьи расходов:
\begin{itemize}
\item Разработка: 2 500 000 рублей
\item Инфраструктура: 1 000 000 рублей
\item Маркетинг: 1 000 000 рублей
\item Непредвиденные расходы: 500 000 рублей
\end{itemize}
\end{itemize}

\subsubsection{Технические ресурсы}
\begin{itemize}
\item Серверное оборудование
\item Облачные сервисы
\item Инструменты аналитики
\end{itemize}

\section{Управление конфигурацией}
% Todo: fill later

\section{Основные процедуры управления проектом}
\begin{itemize}
    \item \textbf{Методы планирования:} Методология Agile с еженедельными спринтами
    \item \textbf{Процедура управления изменениями:} Любое изменение функций должно быть одобрено руководителем проекта и заинтересованными сторонами
    \item \textbf{Процедуры анализа хода проекта:} Еженедельные отчеты о ходе работ
\end{itemize}

\section{Управление коммуникациями}
\begin{itemize}
\item \textbf{Участники коммуникации:} Вся команда со всеми участниками проекта
\item \textbf{Каналы связи:} Slack, Zoom, Email
\item \textbf{Регулярность встреч:} раз в неделю
\item \textbf{Отчеты:} раз в две недели
\end{itemize}

\section{План разработки}
% Gantt Chart to be added

\section{Тестирование в проекте}
% Gantt Chart to be added

\section{Основные риски и управление ими}
\begin{itemize}
\item Задержка сроков --- \textbf{Митигирование:} внедрение Agile-методов
\item Проблемы с финансированием --- \textbf{Митигирование:} привлечение инвесторов
\item Технические ошибки --- \textbf{Митигирование:} тестирование на ранних этапах
\end{itemize}

\section{График передачи результатов}
\begin{itemize}
\item 1 апреля 2025 --- Начало проекта
\item 1 июня 2025 --- Первая версия
\item 30 июня 2025 --- MVP
\end{itemize}


\section{Заключение}
Проект MyGym направлен на создание удобной платформы для автоматизации управления фитнес-клубами. В документе изложены основные аспекты проекта, структура команды, сроки и критерии приемки. План будет обновляться по мере выполнения этапов.

\end{document}
