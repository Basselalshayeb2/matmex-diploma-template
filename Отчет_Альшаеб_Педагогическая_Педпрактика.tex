% !TEX TS-program = xelatex
% !BIB program = bibtex
% !TeX spellcheck = ru_RU

% About magic macros see also
% https://tex.stackexchange.com/questions/78101/

% По умолчанию используется шрифт 14 размера.
% Если Вы не влезаете в лимит страниц и нужен 12-й шрифт,
% то уберите опцию [14pt]

\documentclass[14pt, russian]{matmex-diploma-custom}
\usepackage{listings}
\usepackage{xcolor}

\newcommand{\graybox}[1]{%
  \colorbox{lightgray}{\strut #1}%
}


\input{preamble.tex}

\begin{document}

\input{отчет_Альшаеб_Педагогическая_Педпрактика_title.tex}
\maketitle
\setcounter{tocdepth}{2}

\pagebreak

В период прохождения педагогической практики я принимал участие в проведении семинарских занятий по дисциплинам, связанным с разработкой интернет-приложений, современными технологиями разработки бизнес-приложений и веб-приложений. Практика осуществлялась в учебных группах 22.Б07-мм, 22.Б10-мм, 22.Б11-мм, 22.Б15-мм и 24.М71-мм и была направлена на сопровождение учебного процесса в рамках текущего семестра.

Основное содержание практики включало участие в обсуждении студенческих докладов, посвящённых современным и перспективным технологиям, инструментам и подходам в области программной разработки. В ходе занятий я формулировал вопросы к представленным материалам, инициировал дискуссии по поводу новизны, практической применимости и целесообразности использования рассматриваемых технологий, а также давал рекомендации и экспертные комментарии. Особое внимание уделялось анализу актуальности выбранных решений и их соответствию современным требованиям индустрии. Практика способствовала развитию навыков педагогического взаимодействия, критической оценки технических решений и профессионального обсуждения инновационных технологий.
\end{document}
