% !TEX TS-program = xelatex
% !BIB program = bibtex
% !TeX spellcheck = ru_RU

% About magic macros see also
% https://tex.stackexchange.com/questions/78101/

% По умолчанию используется шрифт 14 размера.
% Если Вы не влезаете в лимит страниц и нужен 12-й шрифт,
% то уберите опцию [14pt]
\documentclass[14pt, russian]{matmex-diploma-custom}

% !TeX spellcheck = ru_RU
% !TEX root = vkr.tex
% Опциональные добавления используемых пакетов. Вполне может быть, что они вам не понадобятся, но в шаблоне приведены примеры их использования.
\usepackage{tikz} % Мощный пакет для создание рисунков, однако может очень сильно замедлять компиляцию
\usetikzlibrary{decorations.pathreplacing,calc,shapes,positioning,tikzmark}

% Библиотека для TikZ, которая генерирует отдельные файлы для каждого рисунка
% Позволяет ускорить компиляцию, однако имеет свои ограничения
% Например, ломает пример выделения кода в листинге из шаблона
% \usetikzlibrary{external}
% \tikzexternalize[prefix=figures/]

\newcounter{tmkcount}

\tikzset{
    use tikzmark/.style={
            remember picture,
            overlay,
            execute at end picture={
                    \stepcounter{tmkcount}
                },
        },
    tikzmark suffix={-\thetmkcount}
}

\usepackage{booktabs} % Пакет для верстки "более книжных" таблиц, вполне годится для оформления результатов
% В шаблоне есть команда \multirowcell, которой нужен этот пакет.
\usepackage{multirow}
\usepackage{siunitx} % для таблиц с единицами измерений

% Для названий стоит использовать \textsc{}
\newcommand{\OCaml}{\textsc{OCaml}}
\newcommand{\miniKanren}{\textsc{miniKanren}}
\newcommand{\BibTeX}{\textsc{BibTeX}}
\newcommand{\vsharp}{\textsc{V$\sharp$}}
\newcommand{\fsharp}{\textsc{F$\sharp$}}
\newcommand{\csharp}{\textsc{C\#}}
\newcommand{\GitHub}{\textsc{GitHub}}
\newcommand{\SMT}{\textsc{SMT}}

\definecolor{eclipseGreen}{RGB}{63,127,95}
% \lstdefinelanguage{ocaml}{
% keywords={@type, function, fun, let, in, match, with, when, class, type,
% nonrec, object, method, of, rec, repeat, until, while, not, do, done, as, val, inherit, and,
% new, module, sig, deriving, datatype, struct, if, then, else, open, private, virtual, include, success, failure,
% lazy, assert, true, false, end},
% sensitive=true,
% commentstyle=\small\itshape\ttfamily,
% keywordstyle=\ttfamily\bfseries, %\underbar,
% identifierstyle=\ttfamily,
% basewidth={0.5em,0.5em},
% columns=fixed,
% fontadjust=true,
% literate={->}{{$\to$}}3 {===}{{$\equiv$}}1 {=/=}{{$\not\equiv$}}1 {|>}{{$\triangleright$}}3 {\\/}{{$\vee$}}2 {/\\}{{$\wedge$}}2 {>=}{{$\ge$}}1 {<=}{{$\le$}} 1,
% morecomment=[s]{(*}{*)}
% }

\makeatletter
\@ifclassloaded{beamer}{
    %%% Обязательные пакеты
    %% Beamer
    \usepackage{beamerthemesplit}
    \usetheme{SPbGU}
    \beamertemplatenavigationsymbolsempty
    \usepackage{appendixnumberbeamer}

    %% Локализация
    \usepackage{fontspec}
    \setmainfont{CMU Serif}
    \setsansfont{CMU Sans Serif}
    \setmonofont{CMU Typewriter Text}
    %\setmonofont{Fira Code}[Contextuals=Alternate,Scale=0.9]
    %\setmonofont{Inconsolata}
    \usepackage{polyglossia}
    \setmainlanguage{russian}
    \setotherlanguage{english}

    %% Графика
    \usepackage{pdfpages} % Позволяет вставлять многостраничные pdf документы в текст

    % Математические окружения с русским названием
    \newtheorem{rutheorem}{Теорема}
    \newtheorem{ruproof}{Доказательство}
    \newtheorem{rudefinition}{Определение}
    \newtheorem{rulemma}{Лемма}
    \usepackage{fancyvrb}
}
{}
\makeatother

\usepackage[autostyle]{csquotes} % Правильные кавычки в зависимости от языка
\usepackage{totcount}
\usepackage{setspace}
\usepackage{amsmath, amsfonts, amssymb, amsthm, mathtools} % "Адекватная" работа с математикой в LaTeX



\begin{document}
% TODO: Formatting
% !TeX spellcheck = ru_RU
% !TEX root = vkr.tex

%% Если что-то забыли, при компиляции будут ошибки Undefined control sequence \my@title@<что забыли>@ru
%% Если англоязычная титульная страница не нужна, то ее можно просто удалить.
\filltitle{ru}{
    %% Актуально только для курсовых/практик. ВКР защищаются не на кафедре а в ГЭК по направлению,
    %%   и к моменту защиты вы будете уже не в группе.
    chair              = {Программная инженерия},
    group              = {24.М71-мм},
    %
    %% Макрос filltitle ненавидит пустые строки, поэтому обязателен хотя бы символ комментария на строке
    %% Актуально всем.
    title              = {Оптимизация и распределённое выполнение сетевой симуляции в Miminet с использованием Docker},
    %
    %% Здесь указывается тип работы. Возможные значения:
    %%   production - производственная практика;
    %%   coursework - отчёт по курсовой работе (ОБРАТИТЕ ВНИМАНИЕ, у техпрога и ПИ нет курсовых, только практики);
    %%   practice - отчёт по учебной практике;
    %%   prediploma - отчёт по преддипломной практике;
    %%   master - ВКР магистра;
    %%   bachelor - ВКР бакалавра.
    type               = {practice},
    %
    %% Здесь указывается вид работы. От вида работы зависят критерии оценивания.
    %%   solution - «Решение». Обучающемуся поручили найти способ решения проблемы в области разработки программного обеспечения или теоретической информатики с учётом набора ограничений.
    %%   experiment - «Эксперимент». Обучающемуся поручили изучить возможности, достоинства и недостатки новой технологии, платформы, языка и т. д. на примере какой-то задачи.
    %%   production - «Производственное задание». Автору поручили реализовать потенциально полезное программное обеспечение.
    %%   comparison - «Сравнение». Обучающемуся поручили сравнить несколько существующих продуктов и/или подходов.
    %%   theoretical - «Теоретическое исследование». Автору поручили доказать какое-то утверждение, исследовать свойства алгоритма и т.п., при этом не требуя написания кода.
    kind               = {solution},
    %
    author             = {Альшаеб Басель},
    %
    %% Актуально только для ВКР. Указывается код и название направления подготовки. Типичные примеры:
    %%   02.03.03 \enquote{Математическое обеспечение и администрирование информационных систем}
    %%   02.04.03 \enquote{Математическое обеспечение и администрирование информационных систем}
    %%   09.03.04 \enquote{Программная инженерия}
    %%   09.04.04 \enquote{Программная инженерия}
    %% Те, что с 03 в середине --- бакалавриат, с 04 --- магистратура.
    specialty          = {09.04.04 \enquote{Математическое обеспечение и администрирование информационных систем}},
    %
    %% Актуально только для ВКР. Указывается шифр и название образовательной программы. Типичные примеры:
    %%   СВ.5162.2020 \enquote{Технологии программирования}
    %%   СВ.5080.2020 \enquote{Программная инженерия}
    %%   ВМ.5665.2022 \enquote{Математическое обеспечение и администрирование информационных систем}
    %%   ВМ.5666.2022 \enquote{Программная инженерия}
    %% Шифр и название программы можно посмотреть в учебном плане, по которому вы учитесь.
    %% СВ.* --- бакалавриат, ВМ.* --- магистратура. В конце --- год поступления (не обязательно ваш, если вы были в академе/вылетали).
    programme          = {ВМ.5666.2024 \enquote{Технологии программирования}},
    %
    %% Актуально всем.
    %% Должно умещаться в одну строчку, допускается использование сокращений, но без переусердствования,
    %% короткая строка с большим количеством сокращений выглядит странно
    %supervisorPosition = {проф. кафeдры системного программирования, д.ф.-м.н.,}, % Терехов А. Н.
    %supervisorPosition = {ст. преподаватель кафедры ИАС, к.~ф.-м.~н. (если есть),}, % Смирнов К. К.
    supervisorPosition = {ст. преподаватель кафедры ИАС, к.~ф.-м.~н.},
    supervisor         = {И. В. Зеленчук.},
    %
    %% Актуально только для практик и курсовых. Если консультанта нет или он совпадает с научником, закомментировать или удалить вовсе.
    % consultantPosition = {должность, ООО \enquote{Место работы}, степень  (если есть),},
    % consultant         = {Консультант~К.~К.},
    %
    %% Актуально только для ВКР.
    % reviewerPosition   = {должность, ООО \enquote{Место работы}, степень (если есть),},
    % reviewer           = {Рецензент~Р.~Р.},
}

% Английский титульник нужен только для ВКР, остальные виды работ могут его смело игнорировать.
\filltitle{en}{
    chair              = {Advisor's chair},
    group              = {ХХ.BХХ-mm},
    title              = {Template for SPbU qualification works},
    type               = {bachelor},
    author             = {FirstName Surname},
    %
    %% Possible choices:
    %%   02.03.03 \foreignquote{english}{Software and Administration of Information Systems}
    %%   02.04.03 \foreignquote{english}{Software and Administration of Information Systems}
    %%   09.03.04 \foreignquote{english}{Software Engineering}
    %%   09.04.04 \foreignquote{english}{Software Engineering}
    %% Те, что с 03 в середине --- бакалавриат, с 04 --- магистратура.
    specialty          = {02.03.03 \foreignquote{english}{Software and Administration of Information Systems}},
    %
    %% Possible choices:
    %%   СВ.5162.2020 \foreignquote{english}{Programming Technologies}
    %%   СВ.5080.2020 \foreignquote{english}{Software Engineering}
    %%   ВМ.5665.2022 \foreignquote{english}{Software and Administration of Information Systems}
    %%   ВМ.5666.2022 \foreignquote{english}{Software Engineering}
    programme          = {СВ.5162.2020 \foreignquote{english}{Programming Technologies}},
    %
    %% Note that common title translations are:
    %%   кандидат наук --- C.Sc. (NOT Ph.D.)
    %%   доктор ... наук --- Sc.D.
    %%   доцент --- docent (NOT assistant/associate prof.)
    %%   профессор --- prof.
    supervisorPosition = {Sc.D, prof.},
    supervisor         = {S.S. Supervisor},
    %
    consultantPosition = {position at \foreignquote{english}{Company}, degree if present},
    consultant         = {C.C. Consultant},
    %
    reviewerPosition   = {position at \foreignquote{english}{Company}, degree if present},
    reviewer           = {R.R. Reviewer},
}

\maketitle
\setcounter{tocdepth}{2}
\tableofcontents

\pagebreak
\newfontfamily\myfont{CMU Sans Serif}
\begin{center}
    \hspace{0pt}
    \vfill
    {\Huge\myfont
        Текст ВКР или учебной практики пишется не ради зачета, а чтобы люди его прочитали, поняли как круто Вы все сделали, и могли продолжить с того места, где Вы остановились.
        \vspace{2em}

        Повторять эту страницу в тексте вашей работы \emph{нельзя}.}
    \vfill
    \hspace{0pt}
\end{center}
\pagebreak

% !TeX spellcheck = ru_RU
% !TEX root = vkr.tex

\section*{Введение}
\thispagestyle{withCompileDate}

Формат из 4х частей рекомендуется в курсе Д.~Кознова~\cite{koznov} по написанию текстов.

\begin{enumerate}
    \item Известная информация (background/обзор).
    \item Неизвестная информация (пробел в знаниях, <<Gap>>).
    \item Гипотезы, вопросы, цели --- <<что болит>>, что будет решать Ваша работа.
    \item Подход, план решения задачи, предлагаемое решение.
\end{enumerate}

Последний абзац должен читаться и быть понятен в отрыве от других трёх. Никакие абзацы нумеровать нельзя.

Части (абзацы) должны занять максимум две страницы, идеально уложиться в одну.

С.-П. Джонс~\cite{SPJGreatPaper} предлагает несколько другой формат написания введения.
Вполне возможно, что если Ваша работа про языки программирования, то его формат будет удачнее.

Введение и заключение читают чаще всего, поэтому они должны быть <<вылизаны>> до блеска.

\blfootnote{
    Иногда рецензенту полезно знать какого числа компилировался текст, чтобы оценить актуальность версии текста. В этом случае полезно вставлять в текст дату сборки. Для совсем официальных релизов документа это не вполне канон.\\
    Также здесь имеет смысл указать, если работа сделана на деньги, например, Российского Фонда Фундаментальных Исследований (РФФИ) по гранту номер такой-то, и т.п.}

% !TeX spellcheck = ru_RU
% !TEX root = vkr.tex

\section{Постановка задачи}
\label{sec:task}

Дословно \enquote{Целью работы является... Для её выполнения были постав\-лены следующие задачи:}
\begin{enumerate}
    \item реализовать это (раздел~\ref{subsec:task1});
    \item спроектировать то-то (раздел~\ref{subsec:task2}) наилучшим образом;
    \item протестировать на том-то (раздел~\ref{subsec:task3}) и обогнать тех-то;
    \item \sout{изучить язык \OCaml{}} писать тут не надо, так как тут должны быть задачи, выполнение которых можно проверить/оценить прочитав текст или выслушав доклад;
          (т.е. Ваши достижения должны быть опровержимы)
          \begin{itemize}
              \item это может вызвать сомнения по поводу обзора~--- \emph{выполнить обзор} писать можно и нужно, но защищаемым результатом будут не ваши знания, а текст обзора (то есть он должен иметь ценность сам по себе);
          \end{itemize}
    \item обязательна задача на валидацию результата, будь то эксперимент, апробация, внедрение~--- то есть доказательство того, что Вы сделали что-то, нужное пользователю.
          Не путайте с валидацией~--- доказательством того, что Вы сделали то, что хотели Вы (например, тесты~--- валидация результата, хорошо, но недостаточно).
\end{enumerate}

% !TeX spellcheck = ru_RU
% !TEX root = vkr.tex

\section{Обзор}
\label{sec:relatedworks}

В данном разделе нужно описать всё, что необходимо для понимания Вашей работы и что придумали не Вы. В дальнейших разделах нельзя прерывать повествование, например, для рассказа о деталях используемой технологии или архитектуре старой системы, потому что читателю будет трудно отличить Ваш вклад от не Вашего.

Любой обзор пишется с какой-то целью (обосновать актуальность, найти и описать интересные решения, сравнить и выбрать технологии) и по какой-то методике поиска материала (например, поиск N релевантных статей на таких-то сервисах). Не будет лишним это всё явно описать.

\subsection{Обзор существующих решений}

\emph{Обзор существующих решений должен быть.} Здесь нужно писать, что индустрия и наука уже сделали по вашей теме. Он нужен, чтобы Вы случайно не изобрели какой-нибудь велосипед.

По-английски называется related works или previous works.

Если Ваша работа является развитием предыдущей и плохо понима\-ема без неё, то обзор должен идти в начале. Если же Вы решаете некоторую задачу новым интересным способом, то если поставить обзор в начале, то читатель может устать, пока доберется до вашего решения. В этом случае уместней поставить обзор после описания Вашего подхода к проблеме\footnote{Такой подход рекомендуется в работе~\cite{SPJGreatPaper}.
    Вполне возможно, что Ваш реальный научный руководитель будет не согласен, и потребует, чтобы обзор был в начале.}.

В обзоре вам нужно рассказать про \emph{преимущества и недостатки} того, что было сделано до Вас.
Неправильным будет перечислять только недостатки, так как если Ваша работа хоть где-то хуже предыдущей, то рецензент будет радостно потирать руки и заваливать Вашу работу.
Гораздо лучше, если Вы честно признаетесь в этом сами.

\subsection{Обзор используемых технологий}

Для технических работ обзор может обозревать продукт, в рамках которого Вы выполняете задачу, другие продукты, где решалась схожая задача, а также используемые технологии с обоснованием выбора тех, которые Вы дальше используете. <<Выбор>> подразумевает наличие вариантов, поэтому опишите, из чего выбирали и почему выбрали то, что выбрали. Очень желательны чёткие критерии сравнения и сводная таблица в конце, где стоят плюсы и минусы рядом с каждым рассматриваемым вариантом.

В обзоре необходимо ссылаться на работы других людей. В данном шаблоне задумано, что литература будет указываться в файле \verb=vkr.bib=. В нём указываются пункты литературы в формате \BibTeX{}, а затем на них можно ссылаться с помощью \verb=\cite{...}=. Та литература, на которую Вы сошлетесь, попадет в список литературы в конце документа. Если не сошлетесь~---  не попадёт. Спецификацию в формате \BibTeX{} почти никогда (для второго курса~--- никогда), не нужно придумывать руками. Правильно: находить в интернете описание цитируемой статьи\footnote{Например, \url{https://dl.acm.org/doi/10.1145/3408995} (дата обращения: \DTMdate{2022-12-17}).},
копировать цитату с помощью кнопки \foreignquote{english}{Export Citation} и вставлять в \BibTeX{} файл. Так же умеет генерировать \BibTeX{}-описания и Google Scholar\footnote{Поисковая система для научных текстов Google Scholar, \url{https://scholar.google.com} (дата обращения: \DTMdate{2024-01-13})}.
Если так не делать, то оформление литературы будет обрастать ошибками.
Например, \BibTeX{} по особенному обрабатывает точ\-ки, запятые и \verb=and= в списке авторов, что позволяет ему самому понимать, сколько авторов у статьи, и что там фамилия, что~--- имя, а что~--- отчество. Google Scholar пытается генерировать описания автоматически, так что, возможно, потребуется ручная правка~--- обязательно проверьте свой список литературы.

В обзоре и в остальном тексте вы наверняка будете использовать названия продуктов или языков программирования.
Для них рекоменду\-ется (в файле \verb=preamble2.tex=) за\-дать специальные команды, чтобы писать сложные названия правильно и одинаково по всему доку\-менту.
Написать с ошибкой  название любимого языка программирова\-ния науч\-ного руко\-водителя~--- идеальный вариант его разозлить.

\subsection{Выводы}

Опишите явно, что читатель должен был вынести из обзора в отдельном подразделе.

% !TeX spellcheck = ru_RU
% !TEX root = vkr.tex

\section{Описание решения}
Реализация в широком смысле: что таки было сделано.
Часто это на самом деле несколько отдельных разделов, по одному на каждую \enquote{реализационную} задачу из постановки.
Например, \enquote{Архитектура} и \enquote{Особенности реализации}, либо по разделу на каждую крупную подзадачу.
Если разделы получились недостаточно эпичными (меньше пары страниц), сделайте их подразделами одного большого раздела, как тут.

Если программной реализации нет, тут пишутся Ваши теоретические наработки, если есть, хоть в каком-то виде, то опишите, даже если серьёзно кодом надо будет заняться только в следующей части.

Если работа на текущем этапе предполагает \emph{только} обзор, то
\begin{itemize}
    \item но всё равно же надо было попробовать воспроизвести чужие результаты, напишите про это сюда;
    \item если нет, то, наверное, обзор получился годным и ценным сам по себе, источников 40-50 было рассмотрено%
          \footnote{Это не шутка, в хороших работах, где целый семестр делался только обзор, оно примерно так и выходит.},
          тогда ладно, раздел \enquote{Описание решения} можно не делать.
\end{itemize}

Для понимания того, как отчёт по учебной практике должен писаться, можно посмотреть видео ниже.
Они про научные доклады и написание научных статей.
Учебные практики и ВКР отличаются тем, что тут есть требования отдельных глав (слайдов) со списком задач и списком результатов.
Но даже если Вы забьёте на требования, специфичные для ВКР, и соблюдете все рекомендации ниже, получившиеся скорее всего будет лучше, чем первая попытка 99\% ваших однокурсников.

\begin{enumerate}
    \item \enquote{Как сделать великолепный научный доклад} от Саймона Пейтона Джонса~\cite{SPJGreatTalk} (на английском).
    \item \enquote{Как написать великолепную научную статью} от Саймона Пейтона Джонса~\cite{SPJGreatPaper} (на английском).
    \item \enquote{Как писать статьи так, чтобы люди их смогли прочитать} от Дэрэка Драйера~\cite{DreyerYoutube2020} (на английском).
          По предыдущим ссылкам разбирается, что должно быть в статье, т.е. как она должна выгля\-деть на высоком уровне.
          Тут более низкоуровнево изложено, как должны быть устроены параграфы и т.п.
    \item Ещё можно посмотреть How to Design Talks~\cite{JhalaYoutube2020} от Ranjit Jhala, но мы думаем, что первых трёх ссылок всем хватит.
\end{enumerate}

\subsection{Первая задача}
\label{subsec:task1}

\subsection{Вторая задача}
\label{subsec:task2}

\subsection{Третья задача}
\label{subsec:task3}

\subsection{Некоторые типичные ошибки}
Здесь мы будем собирать основные ошибки, которые случаются при написании текстов.
В интернетах тоже можно найти коллекции типич\-ных косяков%
\footnote{\href{https://www.read.seas.harvard.edu/~kohler/latex.html}{https://www.read.seas.harvard.edu/\textasciitilde kohler/latex.html} (дата доступа: \DTMdate{2022-12-16}).}.

Рекомендуется по-умол\-ча\-нию использовать красивые греческие бук\-вы $\sigma$  и $\phi$, а именно $\phi$ вместо $\varphi$.
В данном шаблоне команды для этих букв переставлены местами по сравнению с ванильным \TeX'ом.

Также, если работа пишется на русском языке, необходимо, чтобы работа была написана на \textit{грамотном} русском языке даже если автор из ближнего зарубежья%
\footnote{Теоретически, возможен вариант написания текстов на английском языке, но это необходимо обсудить в первую очередь с научным руководителем.}.
Это включает в себя:
\begin{itemize}
    \item разделы должны оформляться с помощью \verb=\section{...}=, а также \verb=\subsection= и т.~п.; не нужно пытаться нумеровать вручную;
    \item точки после окончания предложений должны присутствовать;
    \item пробелы после запятых и точек, в конце слова перед скобкой;
    \item неразрывные пробелы там, где нужны пробелы, но переносить на другую строку нельзя, например \verb=т.~е.=, \verb=А.~Н.~Терехов=, \verb=что-то~\cite{?}=, \verb=что-то~\ref{?}=;
    \item дефис там, где нужен дефис (обозначается с помощью одиночного \enquote{минуса} (англ. dash) на клавиатуре);
    \item двойной дефис там, где он нужен; а именно  при указании проме\-жутка в цифрах: в 1900--1910 г.~г., стр. 150--154;
    \item тире (т.~е. \verb=---=~--- тройной минус) на месте тире, а не что-то другое; в русском языке тире не может \enquote{съезжать} на новую строку, поэтому стоит использовать такой синтаксис: \verb=До~--- после=;
    \item даты стоит писать везде одинаково; чтобы об этом не следить, можно пользоваться заклинанием \verb=\DTMdate{2022-12-16}=;
    \item правильные кавычки должны набираться с помощью пакета \texttt{csquotes}: для основного языка в \texttt{polyglossia} стоит использовать команду \verb=\enquote{текст}=, для второго языка стоит использовать \verb=\foreignquote{язык}{текст}=; правильные кавычки в русской типографии~--- \verb=<<ёлочки>>=, ни в коем случае не \verb="скандинавские лапки"=;
    \item все перечисления должны оформляться с помощью \verb=\enumerate= или \verb=\itemize=; пункты перечислений должны либо начинаться с заглавной буквы и заканчиваться точкой, либо начинаться со строчной и закачиваться точкой с запятой; последний пункт пере\-числений всегда заканчивается точкой.
    \item Перед выкладыванием финальной версии необходимо просмотреть лог \LaTeX'a, и обратить внимание на сообщения вида \emph{Overfull \textbackslash hbox (1.29312pt too wide) in paragraph}. Обычно, это означает, что текст выползает за поля, и надо подсказать, как правильно разделять слова на слоги, чтобы перенос произошел автоматически.
          Это делается, например, так: \verb=соломо\-волокуша=.
\end{itemize}

\subsection{Листинги, картинка и прочий \enquote{не текст}}

Различный \enquote{не текст} имеет свойство отображаться не там, где он написан в текстовом виде в \LaTeX{}, поэтому у него должна быть самодостаточ\-ная понятная подпись \verb=\caption{...}=, уникальная метка \verb=\label{...}=, чтобы на неё можно было бы ссылаться в тексте с помощью \verb=\ref{...}= (более того, ссылка из текста обязательна).
Ниже Вы сможете увидеть таблицу \ref{time_cmp_obj_func}, на которую мы сослались буквально только что.

\enquote{Не текста} может быть довольно много~--- чтобы не засорять корневую папку, хорошим решением будет складывать весь \enquote{не текст} в папку \texttt{figures}.
Заклинание \verb=\includegraphics{}= уже знает этот путь и будет искать там файлы без указания папки.
Команда \verb=\input{}= умеет ходит по путям, например \verb=\input{figures/my_awesome_table.tex}=.
Кроме того, листинги кода можно подтягивать из файла с помощью команды \verb=\inputminted{file}=.

%% Вставка кода с помощью listings
% \begin{lstlisting}[caption={Название для листинга кода. Достаточно длинное, чтобы люди, которые смотрят картинку сразу после названия статьи (т.~е. все люди), смогли разобраться и понять к чему в статье листинги, картинки и прочий \enquote{не текст}.}, language=Caml, frame=single]
%   let x = 5 in x + 1
% \end{lstlisting}

%% Вставка кода с помощью minted
\begin{listing}
    \caption{Название для листинга кода. Достаточно длинное, чтобы люди, которые смотрят картинку сразу после названия статьи (т.~е. все люди), смогли разобраться и понять к чему в статье листинги, картинки и прочий \enquote{не текст}.}
    \begin{minted}[frame=single]{ocaml}
    let x = 5 in x + 1
  \end{minted}
\end{listing}

\subsubsection{Выделение куска листинга с помощью tikz}
Это бывает полезно в текстах, а ещё чаще~--- в презентациях.
Пример сделан на основе вопроса на \textsc{StackExchange}%
\footnote{Вопрос про рамку вокруг листинга на StackExchange, \url{https://tex.stackexchange.com/questions/284311} (дата доступа: \DTMdate{2022-12-16}).}.
Заодно тут показывается альтернативный minted пакет, lstlistings, если не хотите ставить Python и пакет pygments.
В этом случае закомментируйте всё, что связано с minted, в matmex-diploma-custom.cls.

\begin{figure}
    % TODO(Kakadu): Сделать \lstset глобально, чтобы не выписывать все опции листингов каждый раз
    \begin{lstlisting}[ escapechar=!, keepspaces=true, extendedchars=\true, texcl=true
                      , basicstyle=\ttfamily, commentstyle=\color{eclipseGreen}\ttfamily\itshape, language=c ]
#include <stdio.h>
#include <math.h>

/** A comment in English */
int main(void)
{
  double c = -1;
  double z = 0;

  // Это комментарий на русском языке
  printf ("For c = %lf:\n", c);
  for (int i=0; i<10; i++ ) {
    printf ( !\tikzmark{a}!"z %d = %lf\n"!\tikzmark{b}!, i, z);
    z = pow(z, 2) + c;
  }
}
\end{lstlisting}

    \begin{tikzpicture}[use tikzmark]
        \draw[fill=gray,opacity=0.1]
        ([shift={(-3pt,2ex)}]pic cs:a)
        rectangle
        ([shift={(3pt,-0.65ex)}]pic cs:b);
    \end{tikzpicture}
    \caption{Пример листинга c помощью пакета \texttt{listings} и \textsc{TIKZ} декорации к нему, оформленные в окружении \texttt{figure}.
        Обратите внимание, что рисунок отображается не там, где он в документе, а может \enquote{плавать}.}
\end{figure}

\subsection{Некоторые детали описания реализации}
Описание реализации~--- очень важный раздел для будущих программных инженеров, т.е. почти для всех.
Важно иметь всегда, даже если Вы написали прототип на коленке или немного скриптов.

В процессе работы можно сделать огромное количество косяков, неполный список которых ниже.

\begin{enumerate}
    \item Реализация должна быть.
          На публично доступную реализацию обязательна ссылка (в заключении, но можно продублировать тут).
          Если код под \textsc{NDA}, то об этом, во-первых, должно быть сказано явно,
          во-вторых, на защиту должны выно\-ситься другие результаты (например, архитектура), чтобы комис\-сия имела возможность оценить хоть что-то,
          и, в третьих, должна быть справка от работодателя, что Вы правда что-то сделали.
          \begin{itemize}
              \item Рецензент обязан оценить код (о возможности должен побеспо\-коиться обучающийся).
          \end{itemize}
    \item Код реализации должен быть написан защищающимся целиком.
          \begin{itemize}
              \item Если проект групповой, то нужно явно выделить, какие части были модифицированы защищающимся.
                    Например, в преды\-дущих разделах на картинке архитектуры нужно выделить цветом то, что Вы модифицировали.
              \item Нельзя пускать в негрупповой проект коммиты от других людей, или людей не похожих на Вас.
                    Например, в 2022 году защищающийся-парень делал коммиты от сценического псев\-донима, который намекает на женский \enquote{гендер}.
                    (Нет, это не шутка.)
                    На тот момент в российской культуре это выглядело странно.
              \item Возможна ситуация, что вы используете конкретный ник в интернете уже лет пять, и желаете писать ВКР под этим ником на \GitHub{}.
                    В принципе, это допустимо, но если Вы встретите преподавателя, который считает наоборот, то Вам придется грамотно отмазы\-ваться.
                    В Вашу пользу могут сыграть те факты, что к нику на \GitHub{} у Вас приписаны настоящие имя и фамилия; что в репозитории у вас видна домашка за первый курс;
                    и что Ваш преподаватель практики сможет подтвердить, что Вы уже несколько лет используете это ник; и т.п.
          \end{itemize}
    \item Если Вы получаете диплом о присвоении квалификации программиста, код должен соответствовать.
          \begin{enumerate}
              \item Не стоит выкладывать код одним коммитом.
              \item Не стоит выкладывать код аккурат перед защитой.
              \item Лучше хоть какие-то тесты, чем совсем без них.
                    В идеале нужно предъявлять процент покрытия кода тестами.
              \item Лучше сделать \textsc{CI}, а также \textsc{CD}, если оно уместно в Вашем проекте.
              \item Не стоит демонстрировать на защите, что Вам даже не пришло в голову напустить на код линтеры и т.п.
          \end{enumerate}
    \item Если ваша реализация по сути является прохождением стандартного туториала,
          например, по отделению картинок кружек от котиков с помощью машинного обучения, то необходимо срочно сообщить об этом руководителю практики/ВКР,
          иначе Государственная Экзаменацион\-ная Комиссия \enquote{порвёт Вас как Тузик грелку}, поставит \enquote{единицу},
          а все остальные Ваши сокурсники получат оценку выше.
          (Это не шутка, а реальная история 2020 года.)
\end{enumerate}

\noindent Если Вам предстоит защищать учебную практику, а эти рекомендации видятся как более подходящие для защиты ВКР, то ... отмаза не засчиты\-вается, сразу учитесь делать нормально.

% !TeX spellcheck = ru_RU
% !TEX root = vkr.tex

\section{Эксперимент}
Как мы проверяем, что всё удачно получилось.
Если работа рассчитана на несколько семестров и в текущем до эксперимента дело не дошло, опишите максимально подробно, как он будет прово\-диться и на чём
(то, что называется \emph{дизайн эксперимента}~--- от того, что и как Вы будете проверять, очень сильно зависит, что Вы будете делать, так что это важно и делается \emph{не} после реализации).

\subsection{Условия эксперимента}
Железо (если актуально);
версии ОС, компиляторов и параметры командной строки;
почему мы выбрали именно эти тесты; входные дан\-ные, на которых проверяем наш подход, и почему мы выбрали именно их.

\subsection{Исследовательские вопросы }
По-английски называется \emph{research questions}, в тексте можно ссылаться на них как RQ1, RQ2, и т.~д.
Необходимо сформулировать, чего мы хотели бы добиться работой (2 пункта будет хорошо):

\begin{description}
    \item[RQ1]: правда ли предложенный в работе алгоритм лучше вот таких-то остальных?
    \item[RQ2]: насколько существенно каждая составляющая влияет на улучшения?
    (Если в подходе можно включать/выключать какие-то составляющие.)
    \item[RQ3]: насколько точны полученные приближения, если работа строит приближения каких-то штук?
    \item и т.п.
\end{description}

Иногда в работах это называют гипотезами, которые потом проверяют.
Далее в тексте можно ссылаться на исследовательские вопросы как \textsc{RQ}, это обще\-при\-нятое сокращение.

\subsection{Метрики}

Как мы сравниваем, что результаты двух подходов лучше или хуже:
\begin{itemize}
    \item Производительность.
    \item Строчки кода.
    \item Как часто алгоритм \enquote{угадывает} правильную класси\-фикацию входа.
\end{itemize}

\noindent Иногда метрики вырожденные (да/нет), это не очень хорошо, но если в области исследований так принято, то ладно.
Если метрики хитрые (даже IoU или $F_1$-меру можно считать хитрыми), разберите их в обзоре, пояснив, почему выбраны именно такие метрики.

\subsection{Результаты}
Результаты понятно что такое.
Тут всякие таблицы и графики, как в таблице \ref{time_cmp_obj_func}.
Обратите внимание, как цифры выровнены по правому краю, названия по центру, а разделители $\times$ и $\pm$ друг под другом.

Скорее всего Ваши измерения будут удовлетворять нормальному распределению, в идеале это надо проверять с помощью критерия Кол\-могорова и т.п.
Если критерий этого не подтверждает, то у Вас что-то сильно не так с измерениями, надо проверять кэши процессора, отключать Интернет во время измерений, подкручивать среду исполне\-ния (англ. runtime), что\-бы сборка мусора не вмешивалась и т.п.
Если критерий удовлетворён, то необходимо либо указать мат. ожидание и доверительный/предсказы\-вающий интервал, либо мат. ожидание и среднеквадратичное отклонение, либо, если совсем лень заморачиваться, написать, что все измерения проводились с погрешностью, например, в 5\%.
Не приводите слишком много значащих цифр (например, время работы в 239.1 секунды при среднеквадратичном отклонении в 50 секунд выглядит глупо, даже если ваш любимый бенчмарк так посчитал).

Замечание: если у вас получится улуч\-шение производительности в пределах погреш\-ности, то это обязательно вызовет вопросы.
Если погрешность получилась значительной (больше 10-15\% от среднего), это тоже вызовет вопросы, на которые надо ответить, либо разобравшись, что не так (первый подозреваемый~--- мультимодальное распределение), либо более глубоко статистически проанализировав результаты (например, привести гистограммы).

В этом разделе надо также явно ответить на Research Questions или как-то их прокомментировать.

\subsubsection{RQ1} Пояснения
\subsubsection{RQ2} Пояснения

\begin{table}
    \def\arraystretch{1.1}  % Растяжение строк в таблицах
    \setlength\tabcolsep{0.2em}
    \centering
    % \resizebox{\linewidth}{!}{%
    \caption{Производительность какого-то алгоритма при различных разрешениях картинок  (меньше~--- лучше), в мс.,  CI=0.95. За пример таблички кидаем чепчики в честь Я.~Кириленко}
    \begin{tabular}[C]{
            S[table-format=4.4,output-decimal-marker=\times]
            *4{S
                        [table-figures-uncertainty=2, separate-uncertainty=true, table-align-uncertainty=true,
                            table-figures-integer=3, table-figures-decimal=2, round-precision=2,
                            table-number-alignment=center]
                }
        }
        \toprule
        \multicolumn{1}{r}{Resolution} & \multicolumn{1}{r}{\textsc{TENG}} & \multicolumn{1}{r}{\textsc{LAPM}} &
        \multicolumn{1}{r}{\textsc{VOLL4}} \\ \midrule
        1920.1080 & 406.23 \pm 0.94 & 134.06 \pm 0.35 & 207.45 \pm 0.42  \\ \midrule
        1024.768  & 145.0 \pm 0.47  & 39.68 \pm 0.1   &  52.79  \pm 0.1 \\ \midrule
        464.848   & 70.57 \pm 0.2   & 19.86 \pm 0.01     & 32.75  \pm 0.04 \\ \midrule
        640.480   & 51.10 \pm 0.2   & 14.70 \pm 0.1 & 24  \pm 0.04 \\ \midrule
        160.120   & 2.4 \pm 0.02    & 0.67 \pm 0.01      & 0.92  \pm 0.01 \\
        \bottomrule
    \end{tabular}%
    %}
    \label{time_cmp_obj_func}
\end{table}

\clearpage
% !TeX spellcheck = ru_RU
% !TEX root = vkr.tex

\newcolumntype{C}{ >{\centering\arraybackslash} m{4cm} }
\newcommand\myvert[1]{\rotatebox[origin=c]{90}{#1}}
\newcommand\myvertcell[1]{\multirowcell{5}{\myvert{#1}}}
\newcommand\myvertcelll[1]{\multirowcell{4}{\myvert{#1}}}
\newcommand\myvertcellN[2]{\multirowcell{#1}{\myvert{#2}}}

\afterpage{%
    \clearpage% Flush earlier floats (otherwise order might not be correct)
    \thispagestyle{empty}% empty page style (?)
    \begin{landscape}% Landscape page
        \centering % Center table

        \begin{tabular}{|c|c|c|c|c|c|c|c|c|c|c|c|c|c|c|c|c|c|}\hline
            %& \multicolumn{17}{c|}{} \\ \hline
            \multirowcell{2}{Код модуля \\в составе \\ дисциплины,\\практики и т.п. }
            &\myvertcellN{2}{Трудоёмкость\quad}
            & \multicolumn{10}{c|}{\tiny{Контактная работа обучающихся с преподавателем}}
            & \multicolumn{5}{c|}{\tiny{Самостоятельная работа}}
            & \myvertcellN{2}{\tiny Объем активных и интерактивных\quad}
            \\ \cline{3-17}

            && \myvertcellN{2}{лекции\quad}
            &\myvertcellN{2}{семинары\quad}
            &\myvertcellN{2}{консультации\quad}
            &\myvertcellN{2}{\small практические  занятия\quad}
            &\myvertcellN{2}{\small лабораторные работы\quad}
            &\myvertcellN{2}{\small контрольные работы\quad}
            &\myvertcellN{2}{\small коллоквиумы\quad}
            &\myvertcellN{2}{\small текущий контроль\quad}
            &\myvertcellN{2}{\small промежуточная аттестация\quad}
            &\myvertcellN{2}{\small итоговая аттестация\quad}

            &\myvertcellN{2}{\tiny под руководством    преподавателя\quad}
            &\myvertcellN{2}{\tiny в присутствии     преподавателя\quad}
            &\myvertcellN{2}{\tiny с использованием    методических\quad}
            &\myvertcellN{2}{\small текущий контроль\quad}
            &\myvertcellN{2}{\makecell{\small промежуточная \\ аттестация}}
            &     \\
            && &&&&&&&&& &&&&&&\\
            && &&&&&&&&& &&&&&&\\
            && &&&&&&&&& &&&&&&\\
            &&&&&&&&&&& &&&&&&\\
            &&&&&&&&&&& &&&&&&\\
            &&&&&&&&&&& &&&&&&\\ \hline
            Семестр 3 & 2 &30  &&&&&&&&2   & &&&18 &&20 &10\\ \hline
            &   &2--42&&&&&&&&2--25& &&&1--1&&1--1&\\ \hline
            Итого     & 2 &30  &&&&&&&&2   & &&&18 &&20 &10\\ \hline
        \end{tabular}

        \captionof{table}{Если таблица очень большая, то можно её изобразить на отдельной портретной странице. Не забудьте подробное описание, чтобы содержимое таблицы можно было понять не читая весь текст.}
    \end{landscape}
    \clearpage% Flush page
}



\subsection{Обсуждение результатов}

Чуть более неформальное обсуждение, то, что сделано.
Например, почему метод работает лучше остальных?
Или, что делать со случаями, когда метод классифицирует вход некорректно.

\subsection{Угрозы нарушения корректности (опциональный)}

Если основная заслуга метода, это то, что он дает лучшие цифры, то стоит сказать, где мы могли облажаться, когда
\begin{enumerate}
    \item проводили численные замеры;
    \item выбирали тестовый набор (см. \emph{confirmation bias}).
\end{enumerate}

Например, если мы делали замер юзабилити нашего продукта по методике System Usability Scale, но в эксперименте участвовали Ваши друзья, которые хотят Вас порадовать, честно напишите об этом.
Честно напишите, какие меры были приняты для минимизации рисков валидности результатов эксперимента и, если это содержательно, почему их не удалось полностью исключить.

\subsection{Воспроизводимость эксперимента}

Это настолько важно, что заслуживает тут своего подраздела (в самой работе не надо, это должно естественно вытекать из разделов выше)~--- эксперимент должно быть можно повторить и получить примерно такие же результаты, как у Вас.
Поэтому выложите свой код, которым меряли.
Выложите данные, на которых меряли.
Или напишите, где эти данные взять.
Напишите, как конкретно запустить, в каком окружении, что надо дополнительно поставить и т.п.
Чтобы любой второкурсник мог выполнить все пункты и получить тот же график/таблицу, что у Вас.
Не обязательно это делать прямо в тексте, можете в README своего репозитория, но где-то надо.

% !TeX spellcheck = ru_RU
% !TEX root = vkr.tex

\section*{Заключение}
\textbf{Обязательно.} Список результатов, который будет либо один к одному соответствовать задачам из раздела~\ref{sec:task}, либо их уточнять (например, если было <<выбрать>>, то тут <<выбрано то-то>>).

\begin{itemize}
    \item Результат к задаче №1.
    \item Результат к задаче №2.
    \item и т.д.
\end{itemize}
\noindent Если работа на несколько семестров, отчитываетесь Вы только за текущий. Можно в свободной форме обрисовать планы продолжения работы, но не увлекайтесь --- если работа будет продолжена, по ней будет ещё один отчёт.

В заключении \emph{обязательна} ссылка на исходный код, если он выносится на защиту, либо явно напишите тут, что код закрыт. Если работа чисто теоретическая и это понятно из решённых задач, про код можно не писать. Обратите внимание, что ссылка на код должна быть именно в заключении, а не посреди раздела с реализацией, где её никто не найдёт.

Старайтесь оформить программные результаты работы так, чтобы это был один репозиторий или один пуллреквест, правильно оформленный --- комиссии тяжело будет собирать Ваши коммиты по всей истории. А если над проектом работало несколько человек и всё успело изрядно перемешаться, неизбежны вопросы о Вашем вкладе.

Заключение люди реально читают (ещё до \enquote{основных} разделов работы, чтобы понять, что же получилось и стоит ли вообще работу читать), так что оно должно быть вылизано до блеска.


\setmonofont{CMU Typewriter Text}
\bibliographystyle{ugost2008ls}
\bibliography{vkr}

\end{document}
