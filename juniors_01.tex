% !TEX TS-program = xelatex
% !BIB program = bibtex
% !TeX spellcheck = ru_RU

% About magic macros see also
% https://tex.stackexchange.com/questions/78101/

% По умолчанию используется шрифт 14 размера.
% Если Вы не влезаете в лимит страниц и нужен 12-й шрифт,
% то уберите опцию [14pt]

\documentclass[14pt, russian]{matmex-diploma-custom}


\input{preamble.tex}

\begin{document}

\input{junior_01_title.tex}
\maketitle

\section{Состав команды}
Команда состоит из 4-х человек:
\begin{itemize}
    \item Кисельков Денис Андреевич (Руководитель проекта)
    \item Альшаеб Басель (Разработчик)
    \item Макаров Павел Михайлович (Маркетолог)
    \item Малыгин Даниил Александрович (Разработчик)
\end{itemize}

\section{Выбор проекта и его обоснование}
Проектом выбран \textbf{MyGym} — это платформа для управления спортивными клубами и фитнес-центрами.\newline
\textbf{Почему это проект?} MyGym имеет четкую цель, уникальный результат и ограниченные ресурсы, что соответствует определению проекта.

\subsection{Цель проекта}
Создать удобную и функциональную платформу для автоматизации управления фитнес-центрами, включающую:
\begin{itemize}
    \item Запись клиентов на занятия.
    \item Управление абонементами.
    \item Финансовый учет и отчетность.
    \item Интеграцию с платежными системами.
\end{itemize}

\subsection{Ресурсы и ограничения}
\textbf{Ресурсы:}
\begin{itemize}
    \item \textbf{Человеческие ресурсы:}
    \begin{itemize}
        \item Команда разработчиков (бэкенд, фронтенд, DevOps, мобильные разработчики).
        \item Маркетологи для продвижения продукта.
        \item UX/UI-дизайнеры для улучшения пользовательского опыта.
        \item Аналитики данных для изучения поведения пользователей и улучшения бизнес-логики.
    \end{itemize}

    \item \textbf{Финансовые ресурсы:}
    \begin{itemize}
        \item Начальный бюджет: 5 000 000 рублей.
        \item План расходов:
            \begin{itemize}
                \item Разработка и поддержка платформы.
                \item Реклама и маркетинговые кампании.
                \item Операционные расходы (серверы, лицензии, юридическая поддержка).
            \end{itemize}
    \end{itemize}

    \item \textbf{Технические ресурсы:}
    \begin{itemize}
        \item Облачные серверы.
        \item Базы данных.
        \item Фреймворки и технологии.
        \item CI/CD-системы для автоматического развертывания.
        \item Система аналитики (Google Analytics).
    \end{itemize}

    \item \textbf{Организационные ресурсы:}
    \begin{itemize}
        \item Сетевые контакты и партнерства с фитнес-клубами.
    \end{itemize}
\end{itemize}


\textbf{Ограничения:}
\begin{itemize}
    \item Сроки: завершение MVP к 30 июня 2025 года.
    \item Бюджет не должен превышать 5 000 000 рублей.
    \item Функционал первой версии ограничен основными модулями.
\end{itemize}

\section{Устав проекта}
\subsection{Идентификация проекта}
\begin{itemize}
    \item \textbf{Заказчик:} Владелец сети фитнес-клубов "FitRU"
    \item \textbf{Цель:} Разработка MVP системы управления фитнес-клубами
    \item \textbf{Ограничения:} Бюджет, сроки, технологический стек
    \item \textbf{Ответственные лица:} Кисельков Денис Андреевич (PM), Малыгин Даниил Александрович (Tech Lead), Макаров Павел Михайлович (Marketing)
    \item \textbf{Бюджет:} 5 000 000 рублей
\end{itemize}

\subsection{Формальное подтверждение начала проекта}
Проект утвержден заказчиком, все участники команды подтверждают готовность к работе. Дата старта — 1 апреля 2025 года.

\subsection{Передача полномочий руководителю}
Руководителем проекта назначен Кисельков Денис Андреевич, который отвечает за контроль сроков, бюджета и общую координацию работ.

\section{Заключение}
Проект MyGym — это перспективное IT-решение для автоматизации управления фитнес-клубами. В рамках проекта будут разрабатываться и тестироваться основные модули платформы. Следующие этапы включают планирование задач и разработку MVP.

\end{document}
