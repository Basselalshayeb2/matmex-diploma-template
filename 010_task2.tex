% !TeX spellcheck = ru_RU
% !TEX root = vkr2.tex

\section{Постановка задачи}
\label{sec:task}

Целью работы является оптимизация выполнения сетевых эмуляций эмуляций в Miminet\cite{miminet} за счёт эффективного распределения нагрузки между несколькими контейнерами Docker.
Основной задачей является увеличение масштабируемости эмуляций путём равномерного распределения вычислительных ресурсов хоста без использования параллельных вычислений внутри контейнеров.

Для достижения данной цели были поставлены следующие задачи:
\begin{enumerate}
    \item Разработка архитектуры распределённого выполнения эмуляций.
        \begin{itemize}
            \item Реализовать механизм запуска и управления несколькими контейнерами, выполняющими эмуляции Miminet\cite{miminet}.
            \item Обеспечить взаимодействие между контейнерами и централизованный контроль выполнения задач.
        \end{itemize}
    \item Проектирование системы управления задачами и балансировки нагрузки.
        \begin{itemize}
            \item Выбрать и настроить механизм передачи задач (Celery \cite{celery} + RabbitMQ \cite{rabbitmq}).
            \item Разработать стратегию распределения эмуляций между контейнерами, учитывающую доступные вычислительные ресурсы.
            \item Обеспечить возможность динамического масштабирования количества контейнеров.
        \end{itemize}
    \item Экспериментальное тестирование эффективности распределённого подхода.
        \begin{itemize}
            \item Провести сравнение с традиционным подходом выполнения эмуляций в одном контейнере.
            \item Оценить влияние распределения эмуляций на загрузку процессора, время выполнения и потребление памяти.
        \end{itemize}
    \item Валидация результатов и демонстрация практической применимости.
        \begin{itemize}
            \item Провести апробацию предложенного метода на тестовых сетевых топологиях различной сложности.
            \item Оценить потенциальные сценарии использования в реальных задачах сетевого моделирования.
            \item Подготовить документацию и руководство по использованию разработанной системы.
        \end{itemize}
    \item Предложенный подход позволит более эффективно использовать вычислительные ресурсы хоста, избегая перегрузки одного контейнера и обеспечивая стабильную работу эмуляций при увеличении их количества.
\end{enumerate}
