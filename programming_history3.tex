% !TEX TS-program = xelatex
% !BIB program = bibtex
% !TeX spellcheck = ru_RU

% About magic macros see also
% https://tex.stackexchange.com/questions/78101/

% По умолчанию используется шрифт 14 размера.
% Если Вы не влезаете в лимит страниц и нужен 12-й шрифт,
% то уберите опцию [14pt]

\documentclass[14pt, russian]{matmex-diploma-custom}
\usepackage{listings}
\usepackage{xcolor}

\newcommand{\graybox}[1]{%
  \colorbox{lightgray}{\strut #1}%
}


\input{preamble.tex}

\begin{document}

\input{programming_history_title2.tex}
\maketitle
\setcounter{tocdepth}{2}

\pagebreak

\section{Изменения Perl 5 в версиях 5.20–5.42: Алексей Шлёнских}

\textbf{Оценка:} 5

\textbf{Отметка об оригинальности:} *

\textbf{Обоснование оценки:}
Работа демонстрирует высокий уровень аналитической проработки и глубокое понимание эволюции языка Perl. Автор не ограничивается перечислением изменений, а анализирует причины их появления, экспериментальный статус отдельных возможностей и влияние нововведений на экосистему и обратную совместимость. Материал изложен логично, примеры корректны, выводы обоснованы.

Отдельного внимания заслуживает рассмотрение внутренних процессов развития языка и роли сообщества, что придаёт работе исследовательский характер и отличает её от обзорных или учебных текстов.

\textbf{Обоснование отметки оригинальности:}
В работе преимущественно используются источники первого уровня.

% ==================== Work 2 ====================
\section{Десятилетие развития фреймворка тестирования JUnit (2015–2025): Муравьев Илья Владимирович}

\textbf{Оценка:} 4

\textbf{Отметка об оригинальности:} +

\textbf{Обоснование оценки:}
Работа отличается четкой структурой и последовательным изложением материала. Автор точно анализирует эволюцию JUnit, причины перехода на JUnit 5 и архитектурные изменения, повлиявшие на экосистему тестирования Java.

Автор часто следует источникам, а не формулирует собственную критическую позицию.

% ==================== Work 3 ====================
\section{Linux namespaces: Кисельков Денис}

\textbf{Оценка:} 3

\textbf{Отметка об оригинальности:} *

\textbf{Обоснование оценки:}
Работа содержит значительный объем технической информации и демонстрирует хорошее понимание механизмов изоляции Linux. Подробно описана эволюция пространств имен и их роль в развитии контейнерных технологий.

Изложение носит преимущественно описательный характер, и аналитические выводы можно улучшить.

\textbf{Обоснование отметки оригинальности:}
В статье широко используются первоисточники (документация ядра Linux, профильные технические публикации).

% ==================== Work 4 ====================
\section{История развития игровых движков: Ли Цзишэн}

\textbf{Оценка:} 2

\textbf{Отметка об оригинальности:} ?

\textbf{Обоснование оценки:}
Работа представляет собой хорошо написанный исторический и технический обзор, последовательно описывающий развитие игровых движков и ключевых платформ.

Однако аналитическая составляющая слаба: материал носит преимущественно описательный характер и напоминает обзор из учебника.

\textbf{Обоснование отметки оригинальности:}
Прямых заимствований или копирования текста не выявлено. Вместе с тем работа носит преимущественно описательный и реферативный характер, а структура и стиль изложения близки к учебным и обзорным источникам. Авторская аналитика выражена слабо.

% ==================== Work 5 ====================
\section{История вычислительной техники и программирования Эволюция фреймворка Spring (2003–2025): Юсуп Амин Турмуди}

\textbf{Оценка:} 1

\textbf{Отметка об оригинальности:} ?

\textbf{Обоснование оценки:}
Работа носит преимущественно абстрактный характер. Изложение материала близко к официальной документации. Собственный анализ автора минимален.

По сравнению с другими работами, вклад в исследование наименее заметен.

\textbf{Обоснование отметки оригинальности:}
Прямых заимствований не выявлено, однако работа практически полностью следует структуре и содержанию официальной документации Spring Framework. Авторский исследовательский вклад выражен минимально, что вызывает сомнения в степени оригинальности текста.

\setmonofont{CMU Typewriter Text}
% \bibliographystyle{ugost2008ls}
% \bibliography{programming_history2}
\end{document}
