% !TEX TS-program = xelatex
% !BIB program = bibtex
% !TeX spellcheck = ru_RU

% About magic macros see also
% https://tex.stackexchange.com/questions/78101/

% По умолчанию используется шрифт 14 размера.
% Если Вы не влезаете в лимит страниц и нужен 12-й шрифт,
% то уберите опцию [14pt]

\documentclass[14pt, russian]{matmex-diploma-custom}
\usepackage{listings}
\usepackage{xcolor}

\newcommand{\graybox}[1]{%
  \colorbox{lightgray}{\strut #1}%
}


\input{preamble.tex}

\begin{document}

\input{programming_history_title2.tex}
\maketitle
\setcounter{tocdepth}{2}

\pagebreak


\section{Итоговые решения по оригинальности спорных работ}

В соответствии с требованием задания, при наличии отметок «?» или «-»
я связался с авторами соответствующих работ и принял итоговое
решение по оригинальности.

\begin{itemize}
  \item \textbf{Ван Цзыхань:} после коммуникации выяснилось, что причины сомнений связаны
  преимущественно с недостаточным владением русским языком (стилистика/формулировки),
  а не с заимствованиями. \textbf{Итоговая отметка об оригинальности: «+».}

  \item \textbf{Альшаеб Басель (моя работа):} замечания, полученные в оценочном листе
  (орфография/порядок ссылок), не относятся к признакам неоригинальности аналитической работы
  и не подтверждают использование сторонней генерации содержания. \textbf{Итоговая отметка: «+».}

  \item \textbf{Юсуп Амин Турмуди:} автор предоставил комментарий в ответ на запрос,
  однако в своём сообщении сосредоточился преимущественно на пересказе содержания
  собственной работы и не дал пояснений относительно процесса её написания,
  использования источников или возможного применения внешних инструментов.
  \textbf{Итоговая отметка: «?».}

  \item \textbf{Азат Габдрахманова:} комментарий автора на момент сдачи отчёта не получен.
  \textbf{Временная отметка: «?».}
\end{itemize}

\vspace{0.8em}

\section{Ранжирование и оценка оценочных листов (1--5)}

\subsection*{Оценочный лист №1: \underline{Трефилов Степан} \hfill Оценка: 5}

Оценочный лист наиболее компетентен по совокупности критериев: автор приводит
конкретные замечания по тексту, логике и источникам, избегает голословных обвинений
и показывает понимание различия между «слабой аналитикой/структурой» и «неоригинальностью».

\subsection*{Оценочный лист №2: \underline{Альшаеб Басель} \hfill Оценка: 4}

Оценочный лист в целом выполнен на высоком уровне: оценки ранжированы последовательно,
обоснования понятны и соответствуют требованиям формата (оценка, отметка оригинальности,
краткое объяснение). В части оригинальности соблюдён осторожный тон и указаны признаки,
по которым ставится «?» без перехода к необоснованным обвинениям.


\subsection*{Оценочный лист №3: \underline{(Цзыхань Ван)} \hfill Оценка: 3}

Оценочный лист в целом корректен и содержит аргументацию, однако часть обоснований носит
слишком общий характер и недостаточно опирается на конкретные фрагменты текста (примеров
могло бы быть больше). Вопрос оригинальности рассматривается формально: присутствуют
выводы, но им не всегда хватает прозрачной связи с наблюдаемыми признаками.

\subsection*{Оценочный лист №4: \underline{(Киселёв Владимир)} \hfill Оценка: 1}

Автор оценочного листа продемонстрировал стремление к формальному и инструментальному
анализу работ, в том числе с использованием различных внешних средств и проверок.
Это показывает попытку подойти к задаче системно и выявить потенциальные проблемы
оригинальности.

\subsection*{Оценочный лист №5: \underline{Азат Габдрахманов} \hfill Оценка: ?}

Информация не предоставлена.

\setmonofont{CMU Typewriter Text}
% \bibliographystyle{ugost2008ls}
% \bibliography{programming_history2}
\end{document}
