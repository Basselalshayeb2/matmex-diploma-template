% !TEX TS-program = xelatex
% !BIB program = bibtex
% !TeX spellcheck = ru_RU

% About magic macros see also
% https://tex.stackexchange.com/questions/78101/

% По умолчанию используется шрифт 14 размера.
% Если Вы не влезаете в лимит страниц и нужен 12-й шрифт,
% то уберите опцию [14pt]

\documentclass[12pt, a4paper]{matmex-diploma-custom2}
\usepackage{microtype}
\usepackage{enumitem}
\setlist[itemize]{noitemsep,topsep=4pt,leftmargin=1.2cm}
\setlist[enumerate]{noitemsep,topsep=4pt,leftmargin=1.2cm}

\input{preamble.tex}
\usepackage[a4paper,margin=2.5cm]{geometry}

\begin{document}

\input{phd_title.tex}
\maketitle
\setcounter{tocdepth}{2}
\tableofcontents

\textbf{Ключевые слова:} распределённые системы, оптимизация, планирование, балансировка нагрузки, масштабируемость, отказоустойчивость, мониторинг, контейнеризация, микросервисная архитектура, компьютерные сети.

\section{Актуальность}
Распределённые вычислительные системы лежат в основе современных цифровых сервисов: облачных платформ, микросервисных приложений, корпоративных информационных систем, систем обработки больших данных и сетевых сервисов \cite{tanenbaum2002dspap,coulouris2011ds}. Рост требований к производительности, доступности и экономической эффективности приводит к необходимости оптимизации вычислительных и сетевых ресурсов: времени отклика, пропускной способности, стоимости инфраструктуры, энергопотребления, устойчивости к сбоям и эффективности масштабирования \cite{coulouris2011ds}.

Сложность оптимизации обусловлена динамической нагрузкой, неоднородностью ресурсов, распределённостью по узлам и площадкам, ограничениями сети (задержка, потери, пропускная способность), а также необходимостью обеспечения отказоустойчивости и предсказуемого качества обслуживания \cite{tanenbaum2002dspap}. Существенную роль также играют вопросы надёжности, согласованности и воспроизводимости поведения при сбоях и перегрузках \cite{kleppmann2017ddia}. На практике решения часто строятся на локальных эвристиках и разрозненных инструментах, без единой формальной модели качества и воспроизводимой методики оценки. Поэтому разработка математически обоснованных методов и программных средств оптимизации распределённых вычислительных систем является актуальной научно-практической задачей в рамках выбранного направления.

\section{Цель исследования}
Разработка и исследование методов и программных средств, повышающих эффективность функционирования распределённых вычислительных систем за счёт оптимального управления ресурсами, размещения вычислений и учёта сетевых характеристик \cite{coulouris2011ds}.

\section{Основные задачи}
\begin{enumerate}
  \item Проанализировать классы распределённых систем и типовые архитектуры (кластерные, облачные, микросервисные, контейнерные), а также ключевые метрики качества (латентность, throughput, SLA/SLO, стоимость, доступность, устойчивость) \cite{tanenbaum2002dspap,coulouris2011ds}.
  \item Построить формальную модель задачи оптимизации (например, многокритериальную или оптимизацию при ограничениях) с учётом вычислительных ресурсов и сетевых характеристик \cite{papadimitriou1982coac}.
  \item Разработать методы оптимизации, ориентированные на:
  \begin{itemize}
    \item планирование и распределение задач (scheduling),
    \item балансировку нагрузки,
    \item адаптивное масштабирование,
    \item повышение отказоустойчивости и контролируемую деградацию качества (graceful degradation) \cite{kleppmann2017ddia}.
  \end{itemize}
  \item Спроектировать программный прототип (инструмент/модуль) для проведения экспериментов и оценки предложенных методов в воспроизводимой среде (кластер/контейнерная среда/симулятор/тестовый стенд).
  \item Провести экспериментальное исследование на репрезентативных сценариях, сравнить результаты с базовыми стратегиями (round-robin, greedy, статическое распределение), оценить эффективность и ограничения.
\end{enumerate}

\section{Методы и подходы}
В исследовании предполагается использовать:
\begin{itemize}
  \item методы математического моделирования и оптимизации (линейная/целочисленная, многокритериальная оптимизация, эвристики и приближённые методы поиска) \cite{papadimitriou1982coac};
  \item модели компьютерных сетей и теорию графов (учёт топологии, задержек и пропускной способности) \cite{coulouris2011ds,tanenbaum2002dspap};
  \item имитационное моделирование и вычислительные эксперименты;
  \item практики инженерии распределённых систем (наблюдаемость: метрики/логи/трейсы, профилирование, нагрузочное тестирование);
  \item статистическую обработку результатов (доверительные интервалы, сравнение стратегий на наборах сценариев).
\end{itemize}

\section{Ожидаемые результаты}
\begin{enumerate}
  \item Формальная постановка и модель оптимизации управления распределёнными ресурсами \cite{papadimitriou1982coac}.
  \item Набор алгоритмов/стратегий, улучшающих ключевые метрики (время отклика, пропускная способность, стоимость, стабильность при сбоях) по сравнению с базовыми подходами.
  \item Программный прототип (исследовательский стенд/инструмент) для воспроизводимой оценки стратегий оптимизации в различных условиях нагрузки и сетевых ограничений.
  \item Практические рекомендации по применению результатов в системах с микросервисной и/или контейнерной архитектурой и в задачах моделирования сетевых вычислений \cite{kleppmann2017ddia}.
\end{enumerate}

\section{Научная новизна и практическая значимость}
Научная новизна может заключаться в интеграции сетевых характеристик и вычислительных ограничений в единую модель оптимизации распределённой системы, а также в разработке адаптивных методов, учитывающих динамику нагрузки и деградацию при отказах \cite{tanenbaum2002dspap,kleppmann2017ddia}. Практическая значимость состоит в возможности повышения эффективности распределённых сервисов и вычислительных платформ и создании инструментов поддержки принятия решений при конфигурировании и эксплуатации распределённой инфраструктуры \cite{coulouris2011ds}.

\section{Краткий план работ}
Обзор и постановка задачи $\rightarrow$ выбор метрик и модель $\rightarrow$ разработка алгоритмов $\rightarrow$ реализация прототипа $\rightarrow$ экспериментальная проверка $\rightarrow$ анализ результатов и оформление.

% \section{Примерный список литературы}
% \begin{enumerate}
%   \item Tanenbaum A., van Steen M. \textit{Distributed Systems}.
%   \item Coulouris G., Dollimore J., Kindberg T., Blair G. \textit{Distributed Systems: Concepts and Design}.
%   \item Kleppmann M. \textit{Designing Data-Intensive Applications}.
%   \item Hennessy J., Patterson D. \textit{Computer Architecture: A Quantitative Approach}.
%   \item Papadimitriou C., Steiglitz K. \textit{Combinatorial Optimization}.
% \end{enumerate}

\setmonofont{CMU Typewriter Text}
\bibliographystyle{ugost2008ls}
\bibliography{phd}
\end{document}
