% !TEX TS-program = xelatex
% !BIB program = bibtex
% !TeX spellcheck = ru_RU

% About magic macros see also
% https://tex.stackexchange.com/questions/78101/

% По умолчанию используется шрифт 14 размера.
% Если Вы не влезаете в лимит страниц и нужен 12-й шрифт,
% то уберите опцию [14pt]

\documentclass[14pt, russian]{matmex-diploma-custom}


\input{preamble.tex}
\usepackage[utf8]{inputenc}
\usepackage{xcolor}
\usepackage[margin=1in]{geometry}

\begin{document}

\input{junior_06_title.tex}
\maketitle

\section*{Задание}
Выполнить идентификацию, анализ и учет в планировании рисков в выбранном проекте

\section{Идентификация рисков}
\begin{itemize}
    \item Задержка в разработке ключевых модулей (frontend, backend).
    \item Ошибки интеграции с внешними сервисами (платежные системы).
    \item Перегрузка серверов из-за быстрого роста аудитории.
    \item Сложности в маркетинговом продвижении.
    \item Уход ключевых сотрудников в процессе реализации проекта.
    \item Непредвиденные дополнительные расходы.
    \item Юридические риски (например, нарушение законов о защите данных).
\end{itemize}

\section{Анализ рисков}

\begin{table}[h!]
    \centering
    \begin{tabular}{|p{7cm}|p{3cm}|p{3cm}|}
    \hline
    \textbf{Риск} & \textbf{Вероятность} & \textbf{Влияние} \\
    \hline
    Задержка в разработке & Средняя & Высокое \\
    \hline
    Ошибки интеграции с сервисами & Низкая & Среднее \\
    \hline
    Перегрузка серверов & Низкая & Высокое \\
    \hline
    Сложности в маркетинге & Средняя & Среднее \\
    \hline
    Уход ключевых сотрудников & Низкая & Высокое \\
    \hline
    Дополнительные расходы & Средняя & Среднее \\
    \hline
    Юридические риски & Низкая & Высокое \\
    \hline
    \end{tabular}
    \caption{Анализ основных рисков проекта}
    \end{table}

    \subsection{Матрица влияния}

    \begin{center}
        \begin{tikzpicture}[x=2cm, y=2cm]

        % Fill colors
        \fill[green!30]  (0,0) rectangle (1,1);
        \fill[green!30]  (1,0) rectangle (2,1);
        \fill[yellow!50] (2,0) rectangle (3,1);
        \fill[green!30]  (0,1) rectangle (1,2);
        \fill[yellow!50] (1,1) rectangle (2,2);
        \fill[yellow!50] (2,1) rectangle (3,2);
        \fill[yellow!50] (0,2) rectangle (1,3);
        \fill[yellow!50] (1,2) rectangle (2,3);
        \fill[red!60]    (2,2) rectangle (3,3);

        % Draw grid
        \draw[gray] (0,0) grid (3,3);

        % Risk entries (font=\scriptsize for better fitting)
        \node[align=center, font=\scriptsize] at (1.5,2.5) {Задержка\\в разработке};
        \node[align=center, font=\scriptsize] at (1.5,1.5) {Сложности\\в маркетинге};
        \node[align=center, font=\scriptsize] at (1.5,0.5) {Дополнительные\\расходы};

        \node[align=center, font=\scriptsize] at (0.5,1.5) {Ошибки\\интеграции};
        \node[align=center, font=\scriptsize] at (0.5,2.5) {};
        \node[align=center, font=\scriptsize] at (0.5,0.5) {};

        \node[align=center, font=\scriptsize] at (2.5,2.5) {Перегрузка\\серверов};
        \node[align=center, font=\scriptsize] at (2.5,0.5) {Уход\\ключевых\\сотрудников};
        \node[align=center, font=\scriptsize] at (2.5,1.5) {Юридические\\риски};

        % Axis titles
        \node[rotate=90] at (-0.5,1.5) {\textbf{Вероятность}};
        \node at (1.5,-0.5) {\textbf{Угроза}};

        % X axis labels
        \node at (0.5,-0.2) {Низкая};
        \node at (1.5,-0.2) {Средняя};
        \node at (2.5,-0.2) {Высокая};

        % Y axis labels
        \node[rotate=90] at (-0.2,0.5) {Низкое};
        \node[rotate=90] at (-0.2,1.5) {Среднее};
        \node[rotate=90] at (-0.2,2.5) {Высокое};

        \end{tikzpicture}
\end{center}

\section{Учет рисков в планировании}

\begin{itemize}
    \item Для снижения риска задержек — внедрение методологии Agile (спринты, регулярные ревью), контроль прогресса на еженедельной основе.
    \item Для ошибок интеграции — планирование отдельного этапа интеграционного тестирования до выхода в продакшн.
    \item Для предотвращения перегрузки серверов — проектирование масштабируемой архитектуры с использованием облачных технологий (AWS, Azure).
    \item Для решения маркетинговых сложностей — подготовка маркетинговой стратегии и резервного плана по продвижению.
    \item Для минимизации последствий ухода сотрудников — документирование процессов и формирование пула возможных замен.
    \item Для управления дополнительными расходами — выделение резервного бюджета в размере 10--15\% от общей сметы проекта.
    \item Для снижения юридических рисков — консультации с юристами по вопросам защиты персональных данных и подготовки пользовательских соглашений.
\end{itemize}



\end{document}
