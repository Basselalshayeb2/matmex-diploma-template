% !TEX TS-program = xelatex
% !BIB program = bibtex
% !TeX spellcheck = ru_RU

% About magic macros see also
% https://tex.stackexchange.com/questions/78101/

% По умолчанию используется шрифт 14 размера.
% Если Вы не влезаете в лимит страниц и нужен 12-й шрифт,
% то уберите опцию [14pt]

\documentclass[14pt, russian]{matmex-diploma-custom}

\usepackage{fontspec}
\usepackage{pgfplots}
\usepackage{pgfplotstable}
\usepackage{geometry}
\usepackage{booktabs}
\usepackage{caption}
\usepackage{xfp}
\geometry{margin=1in}
\setmainfont{Times New Roman}

\input{preamble.tex}

\begin{document}

\input{parallel_pro_title.tex}
\maketitle
\setcounter{tocdepth}{2}
\tableofcontents

\section*{Постановка задачи}
Необходимо реализовать программу перемножения двух нижне-треугольных матриц \( A \) и \( B \), результатом которой будет матрица \( C \), вычисляемая по формуле:
\begin{equation}
C = A \times B
\end{equation}

\[
A =
\begin{bmatrix}
A_{11} & 0       & 0       & \cdots & 0 \\
A_{21} & A_{22}  & 0       & \cdots & 0 \\
A_{31} & A_{32}  & A_{33}  & \cdots & 0 \\
\vdots& \vdots  & \vdots  & \ddots & \vdots \\
A_{n1} & A_{n2}  & A_{n3}  & \cdots & A_{nn}
\end{bmatrix}
\]

\[
B =
\begin{bmatrix}
B_{11} & 0       & 0       & \cdots & 0 \\
B_{21} & B_{22}  & 0       & \cdots & 0 \\
B_{31} & B_{32}  & B_{33}  & \cdots & 0 \\
\vdots& \vdots  & \vdots  & \ddots & \vdots \\
B_{n1} & B_{n2}  & B_{n3}  & \cdots & B_{nn}
\end{bmatrix}
\]

Каждая матрица имеет размерность \( N \times N \).

В отчёте рассматриваются три версии реализации на языке C:

\begin{enumerate}
    \item \textbf{Версия 1:} Матрицы хранятся в виде двумерных массивов, при этом нулевые элементы не участвуют в вычислениях.
    \item \textbf{Версия 2:} Аналогично первой версии, но с использованием блочной обработки (blocking), улучшающей эффективность кэширования.
    \item \textbf{Версия 3:} Матрицы хранятся в виде одномерных массивов: матрица \( A \) — в блочно-строчном порядке (row-major), а матрица \( B \) — в блочно-столбцовом порядке (column-major).
\end{enumerate}

Во всех версиях исключаются ненужные умножения на ноль, с учётом нижне-треугольной структуры исходных матриц.

\section{Введение}
В данной работе рассматривается задача умножения двух \textbf{нижне-треугольных квадратных матриц} $A$ и $B$ размером $N \times N$. Результатом операции является матрица $C$, также размером $N \times N$, вычисляемая по стандартной формуле матричного умножения:

\begin{equation}
C_{ij} = \sum_{k=0}^{N-1} A_{ik} \cdot B_{kj}
\end{equation}

Однако с учётом того, что $A$ и $B$ являются \textbf{нижне-треугольными}, значительное количество элементов в этих матрицах равно нулю. Поэтому, если использовать стандартный алгоритм, будет выполнено множество \textbf{избыточных операций умножения на ноль}, что снижает эффективность программы как по скорости, так и по использованию памяти.

Для оптимизации этой задачи были применены следующие подходы:

\subsection{Избежание вычислений с нулями}
Вместо полной обработки всех $N^3$ операций в формуле, программа учитывает только \textbf{ненулевые элементы}, опираясь на свойства нижне-треугольных матриц:

\begin{itemize}
    \item Для матрицы $A$: $A_{ik} = 0$, если $k > i$
    \item Для матрицы $B$: $B_{kj} = 0$, если $j > k$
\end{itemize}

Следовательно, при вычислении $C_{ij}$, где $j > i$, вся строка становится нулевой, и её можно не вычислять вообще. Сокращённая формула:
\begin{equation}
C_{ij} = \sum_{k=j}^{i} A_{ik} \cdot B_{kj}, \quad \text{только если } j \leq i
\end{equation}

\subsection{Эффективное хранение с использованием одномерных массивов}

Чтобы дополнительно \textbf{сэкономить память} и повысить \textbf{локальность доступа к данным}, нижне-треугольные матрицы $A$ и $B$ хранятся в виде \textbf{одномерных массивов}:

\begin{itemize}
  \item Матрица $A$ хранится в блочно-строчном порядке (block-row order)
  \item Матрица $B$ хранится в блочно-столбцовом порядке (block-column order)
\end{itemize}

Такой способ хранения позволяет избежать выделения памяти под нули и упростить блочную обработку, особенно при реализации кэш-оптимизированных и параллельных алгоритмов. Количество значащих (ненулевых) элементов в каждой из матриц равно:
\begin{equation}
\frac{N(N+1)}{2}
\end{equation}

Что даёт выигрыш по памяти почти в 2 раза по сравнению с полными $N^2$ элементами, особенно при больших $N$.






\section{Размещение матриц}
При реализации перемножения двух нижнетреугольных матриц мы использовали два различных способа хранения данных: двумерное (2D) и одномерное (1D) представления.
\subsection{Двумерный массив (2D array)}

Это традиционный способ, в котором вся матрица хранится как массив $N \times N$, включая нули выше главной диагонали:

\[
A =
\begin{bmatrix}
A_{11} & 0     & 0     & 0 \\
A_{21} & A_{22} & 0     & 0 \\
A_{31} & A_{32} & A_{33} & 0 \\
A_{41} & A_{42} & A_{43} & A_{44}
\end{bmatrix}
\]

В памяти такая матрица хранится в виде:

\[
\text{A11, 0, 0, 0, A21, A22, 0, 0, A31, A32, A33, 0, A41, A42, A43, A44}
\]

\begin{lstlisting}[language=C, caption={Инициализация нижнетреугольной матрицы в 2D массиве}]
    double **A = (double**)malloc(N * sizeof(double*));
    for (int i = 0; i < N; i++) {
        A[i] = (double*)malloc(N * sizeof(double));
    }
    for (int i = 0; i < N; i++) {
        for (int j = 0; j <= i; j++) {
            A[i][j] = ((double)rand() / RAND_MAX) * 10.0;
        }
    }
\end{lstlisting}

Хотя этот способ удобен, он неэффективен с точки зрения использования памяти, поскольку хранит ненужные нули.

\subsection{Одномерный массив (1D array)}

Чтобы избежать хранения нулей, мы применили два различных варианта размещения элементов нижнетреугольной матрицы в памяти:

\subsubsection{Матрица A — блочно-построчное хранение (block-row order)}

В этом представлении ненулевые элементы хранятся последовательно по строкам:

{\footnotesize
\begin{lstlisting}[language=C, caption={Инициализация матрицы A в 1D массиве}]
    #define EL (N * (N + 1) / 2)

    double* init_matrix_A() {
        double *A = (double*)malloc(EL * sizeof(double));
        for (int i = 0, k = 0; i < N; i++) {
            for (int j = 0; j <= i; j++, k++) {
                A[k] = ((double)rand() / RAND_MAX) * 10.0;
            }
        }
        return A;
    }
\end{lstlisting}
}

\subsubsection{Матрица B — блочно-по-столбцам (block-column order)}

Для лучшего кэширования при блочном перемножении матрица B хранится в колонном порядке:

{\footnotesize
\begin{lstlisting}[language=C, caption={Инициализация матрицы B в 1D массиве}]
#define EL (N * (N + 1) / 2)

double* init_matrix_B() {
    double *B = (double*)malloc(EL * sizeof(double));
    for (int i = 0, k = 0; i < N; i++) {
        for (int j = 0; j <= i; j++, k++) {
            B[(j * (2 * N - j + 1)) / 2 + (i - j)] = ((double)rand() / RAND_MAX) * 10.0;
        }
    }
    return B;
}
\end{lstlisting}
}

Такой подход позволяет повысить производительность за счёт упорядоченного обращения к памяти при блочной обработке.

\subsection{Порядок размещения блоков в памяти}

\begin{itemize}
    \item \textbf{Матрица A (строчный порядок)}: $A_{11}, A_{12}, A_{22}, A_{13}, A_{23}, A_{33}, \ldots, A_{1n_1}, A_{2n_1}, \ldots, A_{n_1n_1}$
    \item \textbf{Матрица B (столбцовый порядок)}: $B_{11}, B_{21}, B_{31}, \ldots, B_{n_1 1}, B_{22}, B_{32}, \ldots, B_{n_1 2}, B_{33}, \ldots$
\end{itemize}

\subsection{Сравнение подходов}

\begin{table}[h!]
\centering
\resizebox{\textwidth}{!}{
\begin{tabular}{|l|p{5cm}|p{5cm}|}
\hline
\textbf{Подход хранения} & \textbf{Преимущества} & \textbf{Недостатки} \\
\hline
Двумерный массив (2D) & Простая реализация, доступ по индексам & Хранит лишние нули, неэффективное использование памяти \\
\hline
1D (строчный порядок для A) & Эффективное хранение, меньше памяти & Сложность индексации \\
\hline
1D (столбцовый порядок для B) & Улучшение кэширования при блочной обработке & Более сложная адресация \\
\hline
\end{tabular}
}
\caption{Сравнение способов хранения матриц}
\end{table}


\section{Алгоритм Перемножения}

В данной работе рассматриваются три версии алгоритма перемножения двух нижнетреугольных матриц. Каждая версия реализует один и тот же базовый алгоритм, но с различиями в представлении и доступе к элементам матриц, а также в методе блочной оптимизации.

\subsection{Общий алгоритм перемножения нижнетреугольных матриц}
Пусть $A$ и $B$ — нижнетреугольные матрицы размера $N \times N$. Тогда их произведение $C = AB$ также будет нижнетреугольной матрицей. Формула для вычисления элемента $C_{ij}$:
\begin{equation}
C_{ij} = \sum_{k=j}^{i} A_{ik} \cdot B_{kj}, \quad \text{где } 0 \leq j \leq i < N
\end{equation}

\subsection{Версия 1: Двумерный массив без блочной оптимизации}
{\footnotesize
\begin{lstlisting}[language=C, caption={Алгоритм перемножения без блокировки для двумерных массивов}]
void multiply_blocked(double **A, double **B, double ***C) {
    for (int i = 0; i < N; i++) {
        for (int j = 0; j <= i; j++) {
            for (int k = j; k <= i; k++) {
                (*C)[i][j] += A[i][k] * B[k][j];
            }
        }
    }
}
\end{lstlisting}
}

\subsection{Версия 2: Двумерный массив с блочной оптимизацией}
{\footnotesize
\begin{lstlisting}[language=C, caption={Алгоритм перемножения с блокировкой для двумерных массивов}]
void multiply_blocked(double **A, double **B, double ***C) {
    for (int bi = 0; bi < N; bi += BLOCK_SIZE) {
        for (int bj = 0; bj <= bi; bj += BLOCK_SIZE) {
            for (int i = bi; i < bi + BLOCK_SIZE && i < N; i++) {
                for (int j = bj; j <= i && j < bj + BLOCK_SIZE; j++) {
                    for (int k = j; k <= i; k++) {
                        (*C)[i][j] += A[i][k] * B[k][j];
                    }
                }
            }
        }
    }
}
\end{lstlisting}
}

\subsection{Версия 3: Одномерные массивы с блочной оптимизацией и пользовательским доступом}
Для доступа к элементам A и B используются функции \texttt{get\_A} и \texttt{get\_B}, которые учитывают способ хранения (строчный и столбцовый порядки).

{\footnotesize
\begin{lstlisting}[language=C, caption={Алгоритм перемножения с блокировкой для 1D массивов}]
void multiply_blocked(double *A, double *B, double *C) {
    for (int i1 = 0; i1 < N; i1 += BLOCK_SIZE) {
        for (int j1 = 0; j1 <= i1; j1 += BLOCK_SIZE) {
            for (int k1 = j1; k1 <= i1; k1 += BLOCK_SIZE) {
                for (int i = i1; i < i1 + BLOCK_SIZE && i < N; i++) {
                    for (int j = j1; j <= i && j < j1 + BLOCK_SIZE; j++) {
                        for (int k = k1; k <= i && k < k1 + BLOCK_SIZE; k++) {
                            double a_val = get_A(A, i, k);
                            double b_val = get_B(B, k, j);
                            C[i * N + j] += a_val * b_val;
                        }
                    }
                }
            }
        }
    }
}

\end{lstlisting}
}

\subsection{Пример работы алгоритма}

Рассмотрим пример перемножения двух нижнетреугольных матриц размера $3 \times 3$:

\begin{equation*}
A = \begin{bmatrix}
a_{00} & 0       & 0 \\
a_{10} & a_{11} & 0 \\
a_{20} & a_{21} & a_{22}
\end{bmatrix}, \quad
B = \begin{bmatrix}
b_{00} & 0       & 0 \\
b_{10} & b_{11} & 0 \\
b_{20} & b_{21} & b_{22}
\end{bmatrix}
\end{equation*}

Тогда элемент $C_{21}$ вычисляется как:
\begin{equation*}
C_{21} = A_{20} B_{01} + A_{21} B_{11}
\end{equation*}

А элемент $C_{22}$:
\begin{equation*}
C_{22} = A_{20} B_{02} + A_{21} B_{12} + A_{22} B_{22}
\end{equation*}

\subsection{Вывод}
Блочная оптимизация позволяет повысить эффективность использования кэш-памяти за счёт улучшения локальности данных. Использование одномерных массивов усложняет доступ к данным, но даёт преимущества при более эффективном использовании памяти.

\section{Доступ к элементам 1D-массивов}

При хранении нижнетреугольных матриц в виде одномерных массивов важно использовать корректные формулы для доступа к элементам.

\subsection{Матрица A (строчное хранение)}

Матрица A хранится в памяти по строкам. Чтобы получить элемент $A[i][j]$ (если $i \geq j$), используется формула:
\begin{equation}
\text{index}_A = \frac{i(i+1)}{2} + j
\end{equation}
Если $i < j$, то $A[i][j] = 0$ (так как элемент вне нижнего треугольника).

\subsection{Матрица B (столбцовое хранение)}

Матрица B хранится по столбцам. Индекс для доступа к элементу $B[i][j]$ (если $i \geq j$):
\begin{equation}
\text{index}_B = \frac{j(2N - j + 1)}{2} + (i - j)
\end{equation}
Если $i < j$, то $B[i][j] = 0$.

\subsection{Код реализации}

{\footnotesize
\begin{lstlisting}[language=C, basicstyle=\ttfamily\footnotesize]
double get_A(double *A, int i, int j) {
    if (i < j) return 0.0;
    int index = (i * (i + 1)) / 2 + j;
    return A[index];
}

double get_B(double *B, int i, int j) {
    if (i < j) return 0.0;
    int index = (j * (2 * N - j + 1)) / 2 + (i - j);
    return B[index];
}
\end{lstlisting}
}

\section{Измерение производительности и выбор размера блока}

Для оценки производительности алгоритма перемножения блочных нижнетреугольных матриц мы использовали стандартную функцию \texttt{clock()} из библиотеки \texttt{time.h}. Все замеры производились при фиксированном размере матрицы $N = 2880$ и различных размерах блоков.

\subsection{Фрагмент кода измерения времени}

{\footnotesize
\begin{lstlisting}[language=C, basicstyle=\ttfamily\footnotesize]
clock_t start = clock();
multiply_blocked(...); // вызов нужной версии
clock_t end = clock();
double cpu_time_used = ((double) (end - start)) / CLOCKS_PER_SEC;
printf("\nExecution time: %f seconds\n", cpu_time_used);
\end{lstlisting}
}

\subsection{Характеристики и компилятор}
\begin{itemize}
    \item \textbf{Компилятор:} GCC v.13.2.0, compiler option: \texttt{-Ofast}
    \item \textbf{ОС:} Microsoft Windows 11 Pro (Version 10.0.22631 Build 22631)
    \item \textbf{Процессор:} Intel® Core™ i7-10750H CPU @ 2.60GHz (up to 4.50 GHz Turbo)
    \begin{itemize}
        \item 6 ядер и 12 потоков
        \item L2: 1.5 MB
        \item L3: 12 MB
    \end{itemize}
    \item \textbf{Оперативная память (RAM):} 16 GB LPDDR4, clock speed: 2933 MHz
\end{itemize}

\section{Таблица с результатами}
\begin{table}[h!]
    \centering
    \caption{Время выполнения (в секундах) для различных размеров блоков при $N = 2880$}
    \begin{tabular}{@{}cccc@{}}
    \toprule
    \textbf{Размер блока} & \textbf{2D без блокировки} & \textbf{2D с блокировкой} & \textbf{1D с блокировкой} \\ \midrule
    -     & 38.550  & -       & -       \\
    30    & -       & 24.383  & 14.793  \\
    60    & -       & 23.564  & 13.670  \\
    90    & -       & 24.054  & 13.054  \\
    120   & -       & 24.298  & 13.325  \\
    180   & -       & 23.070  & 12.763  \\
    240   & -       & \textbf{22.349}  & 12.538  \\
    360   & -       & 23.558  & 12.243  \\
    480   & -       & 24.411  & \textbf{12.195}  \\
    720   & -       & 29.213  & 12.390  \\
    1440  & -       & 35.138  & 12.722  \\
    \bottomrule
    \end{tabular}
\end{table}

\vspace{1cm}
\section{График сравнения производительности}

\begin{center}
\begin{tikzpicture}
\begin{axis}[
    width=14cm,
    height=9cm,
    xlabel={Размер блока},
    ylabel={Время выполнения (сек)},
    legend style={at={(0.5,-0.2)}, anchor=north, legend columns=3},
    ymajorgrids=true,
    grid style=dashed,
    xtick=data,
    xticklabel style={rotate=45, anchor=east},
    symbolic x coords={30,60,90,120,180,240,360,480,720,1440}
]

\addplot[color=red, mark=o] coordinates {
    (30,24.383) (60,23.564) (90,24.054) (120,24.298) (180,23.070)
    (240,22.349) (360,23.558) (480,24.411) (720,29.213) (1440,35.138)
};
\addlegendentry{2D с блочной оптимизацией}

\addplot[color=blue, mark=triangle*] coordinates {
    (30,14.793) (60,13.670) (90,13.054) (120,13.325) (180,12.763)
    (240,12.538) (360,12.243) (480,12.195) (720,12.390) (1440,12.722)
};
\addlegendentry{1D с блочной оптимизацией}

\addplot[color=black, dashed, thick] coordinates {
    (30,38.55) (1440,38.55)
};
\addlegendentry{2D без блочной оптимизации}

\end{axis}
\end{tikzpicture}
\end{center}

\section{Вывод}

Наилучшее время показал размер блока $64$, при котором достигнута оптимальная балансировка между числом операций и эффективным использованием кэш-памяти. При слишком малых блоках увеличивается накладная нагрузка на циклы, а при слишком больших — снижается локальность данных.


\end{document}
