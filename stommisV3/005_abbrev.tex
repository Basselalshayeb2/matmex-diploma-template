% !TeX spellcheck = ru_RU
% !TEX root = stommis.tex

\section*{Список использованных сокращений}
\thispagestyle{withCompileDate}

\begin{description}
    \item[API] \textit{Application Programming Interface} — программный интерфейс взаимодействия компонентов.
    \item[ASR] \textit{Automatic Speech Recognition} — автоматическое распознавание речи (преобразование аудио в текст).
    \item[CPU] \textit{Central Processing Unit} — центральный процессор (в работе подчёркивается возможность выполнения моделей на CPU без GPU).
    \item[DeepSeek] семейство больших языковых моделей (LLM), рассматриваемое как перспектива для экспериментов в задаче интерпретации команд.
    \item[DTW] \textit{Dynamic Time Warping} — метод сравнения последовательностей (используется в некоторых подходах keyword spotting).
    \item[FFmpeg] набор утилит для обработки мультимедиа (в работе используется для транскодирования аудио на сервере).
    \item[FP/FN] \textit{False Positive / False Negative} — ложное срабатывание / пропуск (например, при детекции wake word).
    \item[Intent] \textit{intent} — \textbf{намерение/тип команды} в задачах обработки естественного языка (класс действия, которое должна выполнить система).
    \item[JSON] \textit{JavaScript Object Notation} — формат обмена структурированными данными.
    \item[KWS] \textit{Keyword Spotting} — обнаружение ключевого слова (wake word).
    \item[LLM] \textit{Large Language Model} — большая языковая модель, используемая для семантической интерпретации команд.
    \item[MFCC] \textit{Mel-Frequency Cepstral Coefficients} — кепстральные коэффициенты, часто используемые как признаки в задачах аудио.
    \item[NLP] \textit{Natural Language Processing} — обработка естественного языка.
    \item[МИС] медицинская информационная система.
    \item[МИС «СТОММИС»] медицинская информационная система «СТОММИС» (объект внедрения в работе).
    \item[OGG] контейнерный формат Ogg (в работе используется Ogg Opus для унификации входа ASR).
    \item[Opus] аудиокодек Opus (в работе используется при транскодировании в Ogg Opus).
    \item[ONNX] \textit{Open Neural Network Exchange} — формат представления нейросетевых моделей.
    \item[ONNX Runtime] библиотека выполнения моделей в формате ONNX (используется в клиентском VAD).
    \item[RMS] \textit{Root Mean Square} — среднеквадратичная амплитуда сигнала (простой признак для пороговой детекции речи).
    \item[Slot] \textit{slot} — \textbf{параметр команды} (атрибут/значение, извлекаемое из текста команды; например номер зуба, диагноз, процедура).
    \item[VAD] \textit{Voice Activity Detection} — детекция речевой активности (определение участков речи и «тишины»).
    \item[VoIP] Voice over IP — передача голоса по IP-сетям (телефония через Интернет).
    \item[WASM] \textit{WebAssembly} — формат и среда выполнения кода в браузере (используется для ускорения вычислений на клиенте).
    \item[WebRTC] набор технологий реального времени в вебе; включает классические алгоритмы VAD и подавления эха.
    \item[wake word / trigger word] ключевое слово/фраза активации голосового режима.
\end{description}
