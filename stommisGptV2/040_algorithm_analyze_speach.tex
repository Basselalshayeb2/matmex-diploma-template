% !TeX spellcheck = ru_RU
% !TEX root = stommis.tex

% Этот абзац юридически корректен: ты не обещаешь «идеальное подавление шума», а говоришь «базовые методы» — научруки это ценят.

\section{Алгоритмы обработки речевого сигнала}

Обработка речевого сигнала в разрабатываемой системе осуществляется в несколько этапов, каждый из которых направлен на снижение вычислительной нагрузки, уменьшение количества ошибок распознавания и повышение удобства использования системы в реальных условиях медицинского приёма.

Основная идея алгоритмического подхода заключается в том, что непрерывный звуковой поток не передаётся целиком на сервер и не подвергается постоянному распознаванию. Вместо этого используется последовательная фильтрация аудиоданных, позволяющая выделять только информативные участки речи и игнорировать фоновые шумы и нерелевантные звуки.


\subsection{Захват и предварительная обработка аудиосигнала}

Захват аудиосигнала осуществляется на стороне клиента с использованием стандартных средств Web Audio API. Входной сигнал представляет собой непрерывный поток аудиоданных, поступающих с микрофона пользователя.

На этапе предварительной обработки к аудиосигналу применяются базовые методы улучшения качества записи, включая подавление шума, автоматическую регулировку усиления и компенсацию эха. Данные методы позволяют стабилизировать амплитуду сигнала и повысить устойчивость последующих этапов обработки к изменениям громкости речи и внешним акустическим условиям.

\subsection{Детекция речевой активности}

Для разделения непрерывного аудиопотока на участки речи и условной тишины в системе используется алгоритм детекции речевой активности (Voice Activity Detection, VAD). Задачей данного алгоритма является определение моментов начала и окончания речевых сегментов в звуковом сигнале.

Использование VAD позволяет отказаться от постоянной записи и передачи аудиоданных на сервер, что существенно снижает объём обрабатываемой информации и нагрузку на вычислительные ресурсы. Алгоритм VAD работает в режиме реального времени и принимает решение о наличии речи на основе анализа кратковременных характеристик сигнала.

\subsection{Выделение участков тишины}

Завершение речевой команды определяется на основе анализа продолжительности интервалов тишины между фрагментами речи. После обнаружения речевого сегмента система продолжает наблюдение за входным сигналом и фиксирует момент завершения команды при достижении заданной длительности условной тишины.

Использование интервалов тишины в качестве критерия завершения команды позволяет отказаться от явных маркеров окончания речи и делает взаимодействие с системой более естественным для пользователя. Такой подход особенно удобен в медицинской практике, где речь врача может сопровождаться паузами, не являющимися признаком завершения команды.

\subsection{Активация системы по ключевому слову}

Для предотвращения случайной активации системы и снижения количества ложных срабатываний используется механизм активации по ключевому слову. В неактивном состоянии система выполняет только детекцию речевой активности и анализ коротких речевых фрагментов.

При обнаружении короткого речевого сегмента он передаётся на сервер для проверки на соответствие ключевой фразе активации. Только после подтверждения ключевого слова система переходит в активный режим и начинает обработку основной команды врача.

Данный подход позволяет существенно сократить количество обращений к сервису распознавания речи и, как следствие, снизить эксплуатационные затраты и задержки обработки.

\subsection{Оптимизация вычислительных ресурсов}

Одним из ключевых требований к разрабатываемой системе является минимизация вычислительных затрат при сохранении приемлемой точности распознавания. Для достижения данной цели применяется многоуровневая фильтрация аудиосигнала.

На клиентской стороне выполняется первичная фильтрация и сегментация речи с использованием VAD. На сервер передаются только короткие аудиофрагменты, потенциально содержащие команды или ключевые слова. Полноценное распознавание речи и семантическая интерпретация выполняются исключительно для подтверждённых сегментов.

Таким образом, большая часть нерелевантных данных отбрасывается на ранних этапах обработки, что обеспечивает эффективное использование вычислительных ресурсов и повышает масштабируемость системы.

\subsection{Выводы по главе}

В данной главе был рассмотрен алгоритмический подход к обработке речевого сигнала, основанный на последовательной фильтрации аудиоданных и активации системы по ключевому слову. Предложенные алгоритмы позволяют обеспечить устойчивую работу системы в реальных условиях медицинского приёма при ограниченных вычислительных ресурсах.

Использование детекции речевой активности и анализа интервалов тишины делает взаимодействие с системой естественным и не требует от врача дополнительного обучения или изменения привычного рабочего процесса.
