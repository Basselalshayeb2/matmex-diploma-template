% !TeX spellcheck = ru_RU
% !TEX root = stommis.tex

\section*{Введение}
\thispagestyle{withCompileDate}

Цифровизация медицинских учреждений приводит к тому, что часть клинической работы переносится в информационные системы: врач фиксирует жалобы, анамнез, результаты осмотра и назначений в электронных формах.
В стоматологии это особенно заметно: итогом приёма является структурированная «карта пациента» (диагнозы, манипуляции, зубная формула, рекомендации), которая должна быть заполнена быстро и без потери качества.
В рамках эксплуатации медицинской информационной системы (МИС) «СТОММИС» было выявлено, что взаимодействие врача с компьютером непосредственно \emph{в процессе лечения} крайне неудобно: руки заняты инструментами, врач вынужден постоянно переключать внимание между пациентом и интерфейсом.
На практике это приводит к тому, что значимая часть заполнения карты откладывается на конец приёма, что увеличивает суммарное время приёма и повышает риск потери деталей из-за человеческого фактора.

Существующие решения на рынке нередко сводятся к диктовке «сплошного текста» в аудио/текстовый протокол или к полной записи консультации с последующей расшифровкой.
Подходы такого типа уменьшают объём ручного ввода, но не решают задачу \emph{структурирования} данных (заполнение конкретных полей карты) и создают дополнительную нагрузку: постоянное распознавание речи требует вычислительных ресурсов и, при использовании облачных сервисов, приводит к росту стоимости обработки и сетевого трафика.
Кроме того, при непрерывной обработке возрастают риски ложных срабатываний на разговоры пациента и фоновые звуки.

Целью данной работы является разработка и экспериментальная оценка прототипа подсистемы голосового управления для МИС «СТОММИС», обеспечивающей выделение участков речи и «тишины» в аудиопотоке, захватываемом в браузере, и последующую обработку голосовых команд из короткого списка.
При этом система должна быть ресурсоэффективной и не выполнять постоянное распознавание всего потока.
Ключевые подзадачи, сформулированные руководителем, включают:
\begin{enumerate}
    \item Определение участков с речью и без речи (условной «тишиной»)
    \item Выделение в речи команд из короткого списка (при наличии технической возможности и приемлемого качества)
    \item Внесение изменений в карту пациента в формате МИС «СТОММИС»
\end{enumerate}

Предлагаемый подход основан на многоэтапной обработке.
На стороне клиента в браузере непрерывно работает лёгкий модуль VAD, который определяет начало и конец речевого сегмента и завершает его по правилу «конец команды — 2 секунды тишины».
На сервер передаются только выделенные VAD сегменты речи. При этом на серверной стороне выполняется обнаружение и подтверждение ключевого слова активации (см. раздел 2.2). Такой подход что уменьшает сетевую нагрузку и стоимость распознавания.
На сервере аудиофрагмент при необходимости транскодируется, распознаётся с использованием сервиса Yandex SpeechKit (ASR), а затем интерпретируется как структурированная команда для заполнения полей карты пациента с применением правил и языковой модели (LLM) при строгом ограничении допустимых действий.
Прототип апробировался в том числе в условиях реальной стоматологической поликлиники СПб ГБУЗ «СП №8».
