% !TeX spellcheck = ru_RU
% !TEX root = stommis.tex
\clearpage
\section{Алгоритмы обработки команд и автоматического заполнения медицинской карты}
\label{sec:command_processing}

После завершения этапа распознавания речи система получает текстовую транскрипцию голосовой команды врача. Однако прямое использование данного текста для заполнения медицинской документации является неэффективным и может приводить к ошибкам, поскольку речь врача часто содержит уточнения, комментарии и фразы, не предназначенные для прямого сохранения в медицинской карте.

В связи с этим в разрабатываемой системе реализован отдельный этап интерпретации команд, целью которого является преобразование неструктурированной текстовой информации в формализованное представление, пригодное для автоматизированного внесения данных в электронную медицинскую карту пациента.


\subsection{Разделение речевых фрагментов и управляющих команд}

Речь врача в процессе приёма пациента может содержать как управляющие команды, адресованные системе, так и пояснительные комментарии, не имеющие прямого отношения к заполнению медицинской карты. Для корректной обработки голосового ввода необходимо различать данные типы речевых фрагментов.

В предлагаемом подходе речевая команда рассматривается как управляющее высказывание, содержащее указание на конкретное действие системы, например выбор диагноза, добавление симптома или заполнение определённого поля медицинской карты. Все остальные речевые фрагменты рассматриваются как нерелевантные для автоматической обработки и отбрасываются на данном этапе.


\subsection{Формирование структурного представления команды}

Для обеспечения однозначной интерпретации команд врача используется структурное представление команды, включающее тип действия и набор параметров. Каждая команда преобразуется в формализованную структуру следующего вида:

\begin{center}
\textit{(тип\_команды, объект, параметры)}
\end{center}

Тип команды определяет общее направление действия, например добавление записи, выбор значения или подтверждение состояния пациента. Объект команды указывает на элемент медицинской карты, к которому относится действие, а параметры содержат дополнительные данные, необходимые для выполнения команды.

\subsection{Использование языковой модели для интерпретации команд}

Для преобразования текстовой транскрипции в структурированное представление используется языковая модель, способная выполнять семантический анализ естественного языка. Языковая модель получает на вход текст команды и контекст текущего состояния медицинской карты пациента.

На основе анализа входных данных модель определяет наиболее вероятный тип команды и соответствующие параметры. Использование языковой модели позволяет учитывать вариативность формулировок команд, характерную для естественной речи, и снижает зависимость системы от строго фиксированного набора шаблонов.

\subsection{Ограничение пространства решений и фильтрация команд}

Для повышения надёжности работы системы и предотвращения некорректных действий языковая модель не имеет доступа к произвольному пространству решений. Вместо этого интерпретация команд выполняется в рамках заранее определённого множества допустимых действий и объектов.

В частности, выбор диагнозов, симптомов и процедур осуществляется исключительно из предопределённых медицинских справочников и классификаторов. Такой подход позволяет снизить вероятность логических ошибок и обеспечивает соответствие результатов обработки требованиям медицинских информационных систем.

\subsection{Особенности заполнения стоматологической карты}

Стоматологическая медицинская карта обладает высокой степенью структурированности и включает множество взаимосвязанных параметров, таких как состояние отдельных зубов, наличие патологий и выполненные процедуры. Ручной выбор соответствующих значений из иерархических списков является трудоёмкой задачей и отвлекает врача от лечебного процесса.

В разрабатываемой системе голосовые команды используются для навигации по структуре стоматологической карты и выбора необходимых элементов без прямого взаимодействия с пользовательским интерфейсом. Это позволяет врачу сосредоточиться на пациенте и существенно ускоряет процесс заполнения документации.

\subsection{Интеграция с медицинской информационной системой}

После формирования структурированного представления команды выполняется её интеграция с медицинской информационной системой. На данном этапе проверяется корректность параметров команды и соответствие текущему состоянию медицинской карты пациента.

При успешной проверке система формирует запрос на изменение соответствующих полей и применяет его к медицинской карте \textbf{посредством REST API МИС «СТОММИС»}. В случае возникновения неоднозначностей или ошибок команда отклоняется, а врач получает уведомление о необходимости уточнения запроса.

\subsection{Сравнение с традиционным подходом}

В традиционных системах голосового ввода распознанная речь сохраняется в медицинской карте в виде текстового комментария без дальнейшей обработки. Такой подход не позволяет автоматизировать работу с данными и требует последующего ручного анализа.

В отличие от этого, предлагаемый в работе метод ориентирован на интерпретацию речи как источника управляющих команд. Результатом обработки является не текст, а структурированные изменения в медицинской карте, что обеспечивает более высокий уровень автоматизации и снижает нагрузку на медицинский персонал.

\subsection{Выводы по главе}

В данной главе был рассмотрен алгоритмический подход к интерпретации голосовых команд врача и автоматизированному заполнению медицинской карты пациента. Использование языковой модели в сочетании с ограничением пространства допустимых решений позволяет обеспечить надёжную и контролируемую обработку команд в медицинской информационной системе.
