% !TeX spellcheck = ru_RU
% !TEX root = vkr.tex

\section*{Введение}
\thispagestyle{withCompileDate}

Формат из 4х частей рекомендуется в курсе Д.~Кознова~\cite{koznov} по написанию текстов.

\begin{enumerate}
    \item Известная информация (background/обзор).
    \item Неизвестная информация (пробел в знаниях, \enquote{Gap}).
    \item Гипотезы, вопросы, цели~--- \enquote{что болит}, что будет решать Ваша работа.
    \item Подход, план решения задачи, предлагаемое решение.
\end{enumerate}

Последний абзац должен читаться и быть понятен в отрыве от других трёх. Никакие абзацы нумеровать нельзя.

Части (абзацы) должны занять максимум две страницы, идеально уложиться в одну.

С.-П. Джонс~\cite{SPJGreatPaper} предлагает несколько другой формат написания введения.
Вполне возможно, что если Ваша работа про языки программирования, то его формат будет удачнее.

Введение и заключение читают чаще всего, поэтому они должны быть \enquote{вылизаны} до блеска.

\blfootnote{
    Иногда рецензенту полезно знать какого числа компилировался текст, чтобы оценить актуальность версии текста. В этом случае полезно вставлять в текст дату сборки. Для совсем официальных релизов документа это не вполне канон.\\
    Также здесь имеет смысл указать, если работа сделана на деньги, например, Российского Фонда Фундаментальных Исследований (РФФИ) по гранту номер такой-то, и т.п.}
