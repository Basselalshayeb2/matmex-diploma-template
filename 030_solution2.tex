% !TeX spellcheck = ru_RU
% !TEX root = vkr.tex

\section{Описание решения}
\label{sec:solution}

В рамках данной работы была реализована архитектура распределённого выполнения сетевых эмуляций в рамках проекта Miminet\cite{miminet} с использованием Docker и системы управления задачами Celery + RabbitMQ\cite{rabbitmq}.
Это позволило повысить масштабируемость платформы и сократить время ожидания запуска эмуляции для студентов.

\subsection{Разработка архитектуры распределённого выполнения эмуляций}
\label{subsec:task1}

Изначально все эмуляции в Miminet\cite{miminet} выполнялись в одном контейнере, что ограничивало масштабируемость: увеличение числа одновременных задач приводило к росту времени ожидания и перегрузке одного ядра CPU.

Предложенное решение включает запуск нескольких изолированных Docker-контейнеров, каждый из которых может выполнять одну или несколько эмуляций. Контейнеры работают независимо и получают задания из общей очереди RabbitMQ\cite{rabbitmq}, а изоляция каждой эмуляции реализована с помощью Mininet\cite{mininet}.

Каждый контейнер содержит воркер Celery\cite{celery}, получающий задачи из общей очереди. Такая схема позволяет масштабировать систему горизонтально — за счёт увеличения количества рабочих контейнеров без изменения основной логики.

\begin{figure}[H]
  \centering
  \includegraphics[width=0.95\textwidth]{figures/architecture_single_vs_multi.png}
  \caption{Сравнение архитектур: исходная (один контейнер) и новая (множество контейнеров)}
  \label{fig:arch_compare}
\end{figure}

\begin{figure}[H]
  \centering
  \includegraphics[width=0.95\textwidth]{figures/architecture_single_vs_multi.png}
  \caption{Сравнение архитектур: исходная (один контейнер) и новая (множество контейнеров)}
  \label{fig:arch_compare}
\end{figure}

\subsection{Проектирование и внедрение системы управления задачами}
\label{subsec:task2}

Для распределения задач по контейнерам была использована связка Celery\cite{celery} + RabbitMQ\cite{miminet}.
Celery\cite{celery} позволяет организовать очередь заданий и обеспечивает параллельное выполнение с учётом текущей загрузки.

В рамках данной работы была выполнена:
\begin{itemize}
  \item настройка очереди заданий для передачи задач воркерам;
  \item параметризация контейнеров (например, через переменные окружения) для запуска с разными конфигурациями;
  \item создание логики отправки результата выполнения эмуляции.
\end{itemize}

Каждый контейнер при старте подключается к общей очереди RabbitMQ\cite{rabbitmq} и ожидает задания. Как только задача поступает — она исполняется и возвращает результат в центральную систему.

\subsection{Экспериментальное тестирование и валидация}
\label{subsec:task3}

Для валидации решения были проведены сравнительные тесты двух версий платформы:
\begin{itemize}
  \item одноконтейнерной (исходная реализация Miminet);
  \item многоконтейнерной (предложенная в рамках данной работы).
\end{itemize}

Основные метрики:
\begin{itemize}
  \item среднее количество одновременно выполняемых задач;
  \item загрузка CPU хоста;
  \item время ожидания до начала эмуляции;
\end{itemize}

Эксперименты показали, что система с несколькими контейнерами позволяет запускать больше эмуляций одновременно и равномерно распределяет нагрузку между ядрами процессора.

Это, в свою очередь, приводит к сокращению времени ожидания и повышению эффективности обучения.

\subsection{Итоги реализации}
\label{subsec:summary}

Предложенное решение было внедрено в существующий код проекта Miminet и протестировано в условиях, приближённых к реальному использованию студентами. Архитектура допускает дальнейшее масштабирование и автоматизацию (например, динамическое добавление контейнеров на основе текущей загрузки).

Исходный код решения доступен в открытом репозитории: \url{https://github.com/mimi-net/miminet}
