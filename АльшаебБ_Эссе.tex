% !TEX TS-program = xelatex
% !BIB program = bibtex
% !TeX spellcheck = ru_RU

% About magic macros see also
% https://tex.stackexchange.com/questions/78101/

% По умолчанию используется шрифт 14 размера.
% Если Вы не влезаете в лимит страниц и нужен 12-й шрифт,
% то уберите опцию [14pt]

\documentclass[12pt, russian]{matmex-diploma-custom}


\input{preamble.tex}

\usepackage{fontspec}
\setmainfont{Times New Roman}
\usepackage{geometry}
\geometry{a4paper, margin=1in}
\linespread{1.5}

\begin{document}

\input{АльшаебБ_Эссе - title.tex}
\maketitle
\section*{}
За годы работы в различных компаниях я накопил ценный профессиональный опыт. Мне удалось увидеть, как организуются процессы, как принимаются решения, и как ошибки в управлении могут повлиять на успех проекта. Когда я решил запустить собственный стартап, я осознал: для построения эффективной команды недостаточно просто знать, что работает хорошо — необходимо также уметь предвидеть организационные и операционные сложности.

Моя мотивация стать руководителем проекта выросла из желания исправить те недостатки, с которыми я сталкивался ранее, и применить на практике лучшие решения, которые я наблюдал. Я хочу строить процессы осознанно, учитывая не только технические аспекты, но и особенности командной работы.

В процессе подготовки к запуску стартапа я также понял, что люди по-разному воспринимают задачи и обладают разным уровнем знаний. Эффективное руководство требует не только технической грамотности, но и умения выстраивать коммуникацию, объяснять цели, поддерживать и развивать участников команды.

Моя цель как руководителя проекта — создать среду, где каждый член команды чувствует свою значимость, понимает общие задачи и стремится к общему успеху. Я верю, что грамотная организация процессов и внимательное отношение к людям помогают достигать выдающихся результатов.

Управление проектами для меня — это не только про выполнение задач в срок, но и про развитие команды, личную ответственность и стремление к постоянному улучшению.

\end{document}
