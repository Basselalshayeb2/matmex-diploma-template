% !TEX TS-program = xelatex
% !BIB program = bibtex
% !TeX spellcheck = ru_RU

% About magic macros see also
% https://tex.stackexchange.com/questions/78101/

% По умолчанию используется шрифт 14 размера.
% Если Вы не влезаете в лимит страниц и нужен 12-й шрифт,
% то уберите опцию [14pt]

\documentclass[14pt, russian]{matmex-diploma-custom}


\input{preamble.tex}

\begin{document}

\input{АльшаебБ_Эссе - title.tex}
\maketitle
\section*{Задание}
\begin{itemize}
    \item Выбрать книгу по теме курса (до 3 человек на одну книгу), зафиксировать свой выбор публично (в топике Отчет о книге), прочесть и написать отчет по книге. Это задание необходимо выполнять персонально, в том числе если несколько человек читали одну книгу, отчеты необходимо делать самостоятельно.
    \item План отчета (все пункты обязательны, особое внимание пунктам 3 - 5):
    \begin{enumerate}
        \item Что понравилось и почему.
        \item Что не понравилось и почему.
        \item Что будете использовать в своем проекте (2-3 наиболее интересных идеи).
        \item План внедрения новых знаний.
        \item Критерий оценки эффективности внедрения.
    \end{enumerate}
\end{itemize}

\section{Что понравилось и почему}

Я обожал книгу. Она иллюстрирует в нехудожественной форме, что бывает с ИТ, если там хаос, и там никакой коммуникации, и всё валится само собой.
Через героя Билла и эту историю, показывается, что бывает, если нету DevOps, если не видно, что работаешь и во власти потока книга легкая, поэтому усваивается и понимается с лёгкостью и симпатией.

Пока я читал книгу, описания хаоса и потрясений в компании вызывали у меня сильный стресс.
В итоге мне показалось, насколько важны важна четкость процессов и слаженная работа команды.

\section{Что не понравилось и почему}
книга больше посвящена компьютерному цеху, а про работу с клиентами или конечными пользователями пишется еще меньше.

\section{Что буду использовать в своём проекте}

\begin{itemize}
    \item Практика ограничения незавершённой работы (WIP limits) для повышения прозрачности и эффективности.
    \item Использование трёх путей DevOps (flow, feedback, continuous learning) как фреймворка для улучшения процессов.
\end{itemize}

\section{План внедрения новых знаний}

\begin{itemize}
    \item Провести аудит текущего состояния командных процессов (особенно Dev → Ops).
    \item Визуализировать поток задач (например: через \textbf{Kanban-доску}) и ввести ограничение по WIP.
    \item Установить цикл непрерывной обратной связи: после каждого релиза — внутренний мини-ретроспективный анализ.
    \item Периодически проводить встречи по обмену знаниями между командами (Dev, QA, Ops).
\end{itemize}

\section{Критерий оценки эффективности внедрения}

\begin{itemize}
    \item Снижение количества инцидентов на продакшене \textbf{метрика}: количество багов в месяц.
    \item Увеличение частоты релизов \textbf{метрика}: deployments per week/month.
    \item Повышение удовлетворённости команды \textbf{метрика}: опросы.
\end{itemize}

\end{document}
