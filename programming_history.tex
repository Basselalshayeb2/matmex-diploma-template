% !TEX TS-program = xelatex
% !BIB program = bibtex
% !TeX spellcheck = ru_RU

% About magic macros see also
% https://tex.stackexchange.com/questions/78101/

% По умолчанию используется шрифт 14 размера.
% Если Вы не влезаете в лимит страниц и нужен 12-й шрифт,
% то уберите опцию [14pt]

\documentclass[14pt, russian]{matmex-diploma-custom}


\input{preamble.tex}

\begin{document}

\input{programming_history_title.tex}
\maketitle
\setcounter{tocdepth}{2}
\tableofcontents

\pagebreak

\section*{Введение}
\thispagestyle{withCompileDate}
PHP традиционно играет важную роль в динамичном развитии веб-разработки в течение почти трех десятилетий.
Согласно актуальной статистике W3Techs\cite{W3Techs} по состоянию на 1 ноября 2025 года, PHP используется в 72.9\% всех веб-сайтов с известным серверным языком.

PHP\cite{phplatest} стал основой для многих систем управления контентом и прикладных веб-решений благодаря низкому порогу входа и широкой доступности хостинга.

Тем не менее, устойчивость языка и его экосистемы на протяжении столь длительного периода в значительной степени связана с развитием общих архитектурных практик и стандартов, которые определили индустрию PHP\cite{phplatest}-разработки.

В начале 2000-х годов появились первые PHP\cite{phplatest}-фреймворки. CakePHP (2005)\cite{cakephp2005announce}, Symfony (2005)\cite{symfony2005release} и CodeIgniter (2006)\cite{codeigniter2006announce} поставили основы MVC-подхода и организовали разработку веб-приложений без стандартов.

Тем не менее, по мере роста сложности веб-приложений стало очевидно, что архитектура 2000-х годов сталкивалась с важными ограничениями, такими как непрозрачная архитектура, несовместимость компонентов, отсутствие унифицированных интерфейсов и низкая тестируемость.

В течение следующих десяти лет (2005–2015) произошли значительные изменения в PHP\cite{phplatest}-стеке. Это было связано с разработкой PHP-FIG \cite{phpfig2010charter} и ряда стандартов PSR \cite{psr0standard}, появлением Composer \cite{seldaek2012composer}, унификацией HTTP-модели\cite{psr7standard}, распространением принципов DI \cite{psr11standard} и архитектурой middleware \cite{psr15standard}.

Тем не менее, между 2015 и 2025 годами произошли наиболее значительные структурные изменения в PHP\cite{phplatest}-фреймворке. Эти изменения радикально изменили архитектурные решения, принципы проектирования и практики масштабирования. Эти изменения включали выход PHP 7\cite{php7rfc2015}, массовую типизацию и строгую модель ошибок, переход Symfony на компонентный подход, стремительный рост Laravel\cite{laravel2011announce}, падение Zend Framework и его перерождение в Laminas\cite{laminas2016announce}.

Цель этой работы состоит в том, чтобы проанализировать основные архитектурные и технологические изменения в PHP\cite{phplatest}-фреймворках за 2015–2025 годы, определить причины этих изменений, оценить эффективность принятых решений и определить, какие идеи были отвергнуты или изменились в ходе обсуждений.
В анализе используются материалы рассылок PHP-FIG\cite{phpfig2010charter}, предложения RFC Internals PHP, официальные публикации Symfony\cite{symfony2005release} и Laravel\cite{laravel2011announce}, дискуссии GitHub о стандартах Composer и PSR, а также выступления разработчиков на конференциях, таких как SymfonyCon\cite{symfonycon} и Laracon\cite{laracon}.

Таким образом, в статье рассматривается не только развитие определенных фреймворков, но и более широкий процесс создания профессиональной экосистемы PHP, в которой технологические решения являются основной движущей силой прогресса.


\setmonofont{CMU Typewriter Text}
\bibliographystyle{ugost2008ls}
\bibliography{programming_history}

\end{document}
