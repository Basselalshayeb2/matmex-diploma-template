% !TEX TS-program = xelatex
% !BIB program = bibtex
% !TeX spellcheck = ru_RU

% About magic macros see also
% https://tex.stackexchange.com/questions/78101/

% По умолчанию используется шрифт 14 размера.
% Если Вы не влезаете в лимит страниц и нужен 12-й шрифт,
% то уберите опцию [14pt]

\documentclass[14pt, russian]{matmex-diploma-custom}


% !TeX spellcheck = ru_RU
% !TEX root = vkr.tex
% Опциональные добавления используемых пакетов. Вполне может быть, что они вам не понадобятся, но в шаблоне приведены примеры их использования.
\usepackage{tikz} % Мощный пакет для создание рисунков, однако может очень сильно замедлять компиляцию
\usetikzlibrary{decorations.pathreplacing,calc,shapes,positioning,tikzmark}

% Библиотека для TikZ, которая генерирует отдельные файлы для каждого рисунка
% Позволяет ускорить компиляцию, однако имеет свои ограничения
% Например, ломает пример выделения кода в листинге из шаблона
% \usetikzlibrary{external}
% \tikzexternalize[prefix=figures/]

\newcounter{tmkcount}

\tikzset{
    use tikzmark/.style={
            remember picture,
            overlay,
            execute at end picture={
                    \stepcounter{tmkcount}
                },
        },
    tikzmark suffix={-\thetmkcount}
}

\usepackage{booktabs} % Пакет для верстки "более книжных" таблиц, вполне годится для оформления результатов
% В шаблоне есть команда \multirowcell, которой нужен этот пакет.
\usepackage{multirow}
\usepackage{siunitx} % для таблиц с единицами измерений

% Для названий стоит использовать \textsc{}
\newcommand{\OCaml}{\textsc{OCaml}}
\newcommand{\miniKanren}{\textsc{miniKanren}}
\newcommand{\BibTeX}{\textsc{BibTeX}}
\newcommand{\vsharp}{\textsc{V$\sharp$}}
\newcommand{\fsharp}{\textsc{F$\sharp$}}
\newcommand{\csharp}{\textsc{C\#}}
\newcommand{\GitHub}{\textsc{GitHub}}
\newcommand{\SMT}{\textsc{SMT}}

\definecolor{eclipseGreen}{RGB}{63,127,95}
% \lstdefinelanguage{ocaml}{
% keywords={@type, function, fun, let, in, match, with, when, class, type,
% nonrec, object, method, of, rec, repeat, until, while, not, do, done, as, val, inherit, and,
% new, module, sig, deriving, datatype, struct, if, then, else, open, private, virtual, include, success, failure,
% lazy, assert, true, false, end},
% sensitive=true,
% commentstyle=\small\itshape\ttfamily,
% keywordstyle=\ttfamily\bfseries, %\underbar,
% identifierstyle=\ttfamily,
% basewidth={0.5em,0.5em},
% columns=fixed,
% fontadjust=true,
% literate={->}{{$\to$}}3 {===}{{$\equiv$}}1 {=/=}{{$\not\equiv$}}1 {|>}{{$\triangleright$}}3 {\\/}{{$\vee$}}2 {/\\}{{$\wedge$}}2 {>=}{{$\ge$}}1 {<=}{{$\le$}} 1,
% morecomment=[s]{(*}{*)}
% }

\makeatletter
\@ifclassloaded{beamer}{
    %%% Обязательные пакеты
    %% Beamer
    \usepackage{beamerthemesplit}
    \usetheme{SPbGU}
    \beamertemplatenavigationsymbolsempty
    \usepackage{appendixnumberbeamer}

    %% Локализация
    \usepackage{fontspec}
    \setmainfont{CMU Serif}
    \setsansfont{CMU Sans Serif}
    \setmonofont{CMU Typewriter Text}
    %\setmonofont{Fira Code}[Contextuals=Alternate,Scale=0.9]
    %\setmonofont{Inconsolata}
    \usepackage{polyglossia}
    \setmainlanguage{russian}
    \setotherlanguage{english}

    %% Графика
    \usepackage{pdfpages} % Позволяет вставлять многостраничные pdf документы в текст

    % Математические окружения с русским названием
    \newtheorem{rutheorem}{Теорема}
    \newtheorem{ruproof}{Доказательство}
    \newtheorem{rudefinition}{Определение}
    \newtheorem{rulemma}{Лемма}
    \usepackage{fancyvrb}
}
{}
\makeatother

\usepackage[autostyle]{csquotes} % Правильные кавычки в зависимости от языка
\usepackage{totcount}
\usepackage{setspace}
\usepackage{amsmath, amsfonts, amssymb, amsthm, mathtools} % "Адекватная" работа с математикой в LaTeX



\begin{document}

% !TeX spellcheck = ru_RU
% !TEX root = vkr.tex


%% Если что-то забыли, при компиляции будут ошибки Undefined control sequence \my@title@<что забыли>@ru
%% Если англоязычная титульная страница не нужна, то ее можно просто удалить.
\filltitle{ru}{
    date = {2 5 2025},
    %% Актуально только для курсовых/практик. ВКР защищаются не на кафедре а в ГЭК по направлению,
    %%   и к моменту защиты вы будете уже не в группе.
    chair              = {Программная инженерия},
    group              = {24.М71-мм},
    %
    %% Макрос filltitle ненавидит пустые строки, поэтому обязателен хотя бы символ комментария на строке
    %% Актуально всем.
    title              = {Марксизм: теория и практика},
    %
    %% Здесь указывается тип работы. Возможные значения:
    %%   production - производственная практика;
    %%   coursework - отчёт по курсовой работе (ОБРАТИТЕ ВНИМАНИЕ, у техпрога и ПИ нет курсовых, только практики);
    %%   practice - отчёт по учебной практике;
    %%   prediploma - отчёт по преддипломной практике;
    %%   master - ВКР магистра;
    %%   bachelor - ВКР бакалавра.
    type               = {practice},
    %
    %% Здесь указывается вид работы. От вида работы зависят критерии оценивания.
    %%   solution - «Решение». Обучающемуся поручили найти способ решения проблемы в области разработки программного обеспечения или теоретической информатики с учётом набора ограничений.
    %%   experiment - «Эксперимент». Обучающемуся поручили изучить возможности, достоинства и недостатки новой технологии, платформы, языка и т. д. на примере какой-то задачи.
    %%   production - «Производственное задание». Автору поручили реализовать потенциально полезное программное обеспечение.
    %%   comparison - «Сравнение». Обучающемуся поручили сравнить несколько существующих продуктов и/или подходов.
    %%   theoretical - «Теоретическое исследование». Автору поручили доказать какое-то утверждение, исследовать свойства алгоритма и т.п., при этом не требуя написания кода.
    kind               = {none},
    %
    author             = {Альшаеб Басель},
    %
    %% Актуально только для ВКР. Указывается код и название направления подготовки. Типичные примеры:
    %%   02.03.03 \enquote{Математическое обеспечение и администрирование информационных систем}
    %%   02.04.03 \enquote{Математическое обеспечение и администрирование информационных систем}
    %%   09.03.04 \enquote{Программная инженерия}
    %%   09.04.04 \enquote{Программная инженерия}
    %% Те, что с 03 в середине --- бакалавриат, с 04 --- магистратура.
    specialty          = {09.04.04 \enquote{Математическое обеспечение и администрирование информационных систем}},
    %
    %% Актуально только для ВКР. Указывается шифр и название образовательной программы. Типичные примеры:
    %%   СВ.5162.2020 \enquote{Технологии программирования}
    %%   СВ.5080.2020 \enquote{Программная инженерия}
    %%   ВМ.5665.2022 \enquote{Математическое обеспечение и администрирование информационных систем}
    %%   ВМ.5666.2022 \enquote{Программная инженерия}
    %% Шифр и название программы можно посмотреть в учебном плане, по которому вы учитесь.
    %% СВ.* --- бакалавриат, ВМ.* --- магистратура. В конце --- год поступления (не обязательно ваш, если вы были в академе/вылетали).
    programme          = {ВМ.5666.2024 \enquote{Технологии программирования}},
    %
    %% Актуально всем.
    %% Должно умещаться в одну строчку, допускается использование сокращений, но без переусердствования,
    %% короткая строка с большим количеством сокращений выглядит странно
    %supervisorPosition = {проф. кафeдры системного программирования, д.ф.-м.н.,}, % Терехов А. Н.
    %supervisorPosition = {ст. преподаватель кафедры ИАС, к.~ф.-м.~н. (если есть),}, % Смирнов К. К.
    supervisorPosition = {},
    supervisor         = {},
    teacher            = {Секацкий Александр}
    %
    %% Актуально только для практик и курсовых. Если консультанта нет или он совпадает с научником, закомментировать или удалить вовсе.
    % consultantPosition = {должность, ООО \enquote{Место работы}, степень  (если есть),},
    % consultant         = {Консультант~К.~К.},
    %
    %% Актуально только для ВКР.
    % reviewerPosition   = {должность, ООО \enquote{Место работы}, степень (если есть),},
    % reviewer           = {Рецензент~Р.~Р.},
}

% Английский титульник нужен только для ВКР, остальные виды работ могут его смело игнорировать.
\filltitle{en}{
    chair              = {Advisor's chair},
    group              = {ХХ.BХХ-mm},
    title              = {Template for SPbU qualification works},
    type               = {bachelor},
    author             = {FirstName Surname},
    %
    %% Possible choices:
    %%   02.03.03 \foreignquote{english}{Software and Administration of Information Systems}
    %%   02.04.03 \foreignquote{english}{Software and Administration of Information Systems}
    %%   09.03.04 \foreignquote{english}{Software Engineering}
    %%   09.04.04 \foreignquote{english}{Software Engineering}
    %% Те, что с 03 в середине --- бакалавриат, с 04 --- магистратура.
    specialty          = {02.03.03 \foreignquote{english}{Software and Administration of Information Systems}},
    %
    %% Possible choices:
    %%   СВ.5162.2020 \foreignquote{english}{Programming Technologies}
    %%   СВ.5080.2020 \foreignquote{english}{Software Engineering}
    %%   ВМ.5665.2022 \foreignquote{english}{Software and Administration of Information Systems}
    %%   ВМ.5666.2022 \foreignquote{english}{Software Engineering}
    programme          = {СВ.5162.2020 \foreignquote{english}{Programming Technologies}},
    %
    %% Note that common title translations are:
    %%   кандидат наук --- C.Sc. (NOT Ph.D.)
    %%   доктор ... наук --- Sc.D.
    %%   доцент --- docent (NOT assistant/associate prof.)
    %%   профессор --- prof.
    supervisorPosition = {Sc.D, prof.},
    supervisor         = {S.S. Supervisor},
    %
    consultantPosition = {position at \foreignquote{english}{Company}, degree if present},
    consultant         = {C.C. Consultant},
    %
    reviewerPosition   = {position at \foreignquote{english}{Company}, degree if present},
    reviewer           = {R.R. Reviewer},
}

\maketitle

\section{Intro}
Marxism originated in the mid-19th century, primarily developed by Karl Marx (1818–1883) and Friedrich Engels (1820–1895).
It emerged as a revolutionary critique of capitalism and a theoretical foundation for socialism and communism.
It stands as a seminal framework in understanding societal structures, economic systems.
Over the decades, it has evolved from a theoretical construct to a practical guide influencing revolutions, policies, and academic discourses worldwide.

\section{Theoretical Foundations of Marxism}
Central to Marxist theory is the concept of historical materialism, which posits that material conditions and economic activities are the primary drivers of societal change.
Marx argued that the mode of production in any given society fundamentally shapes its social structures and relations. This perspective emphasizes that the economic base influences the cultural and political superstructures.

Another pivotal element is the class struggle. Marx believed that history is marked by conflicts between oppressor and oppressed classes.
In capitalist societies, this manifests as a struggle between the bourgeoisie (owners of the means of production) and the proletariat (working class).
The bourgeoisie exploits the labor of the proletariat, leading to inherent tensions and eventual revolutionary change.

Furthermore, Marx introduced the concept of alienation, where workers become estranged from their labor, the products they produce, and their own human potential.
This alienation arises because workers have little control over the production process and are often reduced to mere cogs in the capitalist machine.

\section{Intellectual Influences}

\begin{itemize}
    \item \textbf{German Philosophy:} Marx was influenced by \textbf{G.W.F. Hegel’s dialectics} (the idea that history progresses through contradictions) but rejected Hegel’s idealism, instead developing dialectical materialism (focusing on material conditions rather than ideas).
    \item \textbf{French Socialism & Revolution:} Early socialist thinkers like \textbf{Henri de Saint-Simon} and \textbf{Charles Fourier}, along with the radical politics of the \textbf{French Revolution}, shaped Marx’s vision of class struggle.
\end{itemize}

\section{Evolution into Marxist Movements}
After Marx’s death, Engels and later thinkers (like Lenin, Rosa Luxemburg, and Trotsky) adapted Marxism into different strands:

\begin{itemize}
    \item \textbf{Orthodox Marxism:} Strict adherence to Marx’s theories.
    \item \textbf{Leninism:} Revolutionary vanguard party leading the proletariat (basis for Soviet communism).
    \item \textbf{Western Marxism:} Focus on culture and ideology (e.g., Gramsci, Frankfurt School).
\end{itemize}

\section*{Marxism in Practice}
The transition from Marxist theory to practice has been both transformative and contentious. The 20th century witnessed several revolutions inspired by Marxist principles, notably in Russia, China, and Cuba. These movements aimed to dismantle capitalist structures and establish socialist states.

In the Soviet Union, the Bolshevik Revolution of 1917 led to the establishment of a socialist state under Lenin's leadership. While the revolution aimed to realize Marxist ideals, the ensuing regime faced challenges, including authoritarianism and economic difficulties. Similarly, China's Maoist revolution sought to adapt Marxist principles to Chinese conditions, leading to significant social and economic reforms but also periods of turmoil, such as the Cultural Revolution.

These implementations highlight the complexities of applying Marxist theory in diverse socio-political contexts. While they achieved certain objectives, they also deviated from Marx's original vision, leading to debates about the authenticity and efficacy of these practices.

\section{Criticisms}

Marxist theory, despite its substantial impact on political and economic thought, has faced considerable criticism from various academic and ideological perspectives.

From a political standpoint, critics argue that Marxism lacks a coherent framework for political organization and conflict resolution after the overthrow of capitalism. While Marx detailed the processes leading to revolution, he provided relatively little guidance on the structure of governance and policy in a post-capitalist society. This has led to a wide range of interpretations and implementations—some of which, like those in the Soviet Union or Maoist China, resulted in authoritarian regimes far removed from Marx's original ideals.

Economically, the resilience of capitalism has challenged Marx’s predictions. Contrary to Marx's expectation that capitalism would collapse under the weight of its internal contradictions, the system has demonstrated adaptability through reforms, welfare mechanisms, and technological innovation. Capitalist economies have proven more resistant to crises than Marx anticipated, and some argue that his failure to foresee the dynamism of capitalism undermines the predictive power of his theory.

From a scientific perspective, Marxism has been criticized for its methodological approach. It does not conform to the standards of falsifiability and empirical testing associated with scientific theories. Instead, its reliance on dialectical reasoning—a method that seeks to understand change through contradictions—is viewed by critics as vague and incompatible with the rigor of scientific methodology. This has led some scholars to classify Marxism more as a philosophy of history than a scientific theory.

Philosophically, Marxism’s strict materialism and atheism have been points of contention. Critics argue that such a worldview cannot adequately explain non-material human values such as compassion, morality, or altruism. These critics assert that by reducing human behavior to class interest and material conditions, Marxism neglects the spiritual and ethical dimensions of life that often play critical roles in human society.

In sum, while Marxism provides powerful tools for analyzing class and economic inequality, its theoretical and practical limitations have prompted significant debate. These criticisms do not render the theory obsolete but rather invite continuous reflection, adaptation, and critical engagement.

\section{Contemporary Relevance of Marxism}
Despite the challenges faced in its practical applications, Marxism remains a vital analytical tool in contemporary discourse. The global financial crises, rising income inequalities, and debates over labor rights have reignited interest in Marxist critiques of capitalism. Scholars and activists utilize Marxist frameworks to analyze modern phenomena, from gig economies to globalization.
\section{Conclusion}
Marxism, as both a theory and practice, offers profound insights into the workings of societies and economies. While its practical implementations have faced criticisms and challenges, the core principles of Marxist thought continue to resonate in contemporary analyses of social injustices and economic disparities. As the world grapples with evolving challenges, Marxism provides a framework to question existing structures and envision equitable alternatives.

\bibliographystyle{plain}
\bibliography{philosophy_bib}

\end{document}
