% !TEX TS-program = xelatex
% !BIB program = bibtex
% !TeX spellcheck = ru_RU

% About magic macros see also
% https://tex.stackexchange.com/questions/78101/

% По умолчанию используется шрифт 14 размера.
% Если Вы не влезаете в лимит страниц и нужен 12-й шрифт,
% то уберите опцию [14pt]

\documentclass[14pt, russian]{matmex-diploma-custom}


\input{preamble.tex}

\begin{document}

\input{philosophy_title.tex}
\maketitle

\section{Intro}
Marxism originated in the mid-19th century, primarily developed by Karl Marx (1818–1883) and Friedrich Engels (1820–1895).
It emerged as a revolutionary critique of capitalism and a theoretical foundation for socialism and communism.
It stands as a seminal framework in understanding societal structures, economic systems.
Over the decades, it has evolved from a theoretical construct to a practical guide influencing revolutions, policies, and academic discourses worldwide.

\section{Theoretical Foundations of Marxism}
Central to Marxist theory is the concept of historical materialism, which posits that material conditions and economic activities are the primary drivers of societal change. Marx argued that the mode of production in any given society fundamentally shapes its social structures and relations. This perspective emphasizes that the economic base influences the cultural and political superstructures.

Another pivotal element is the class struggle. Marx believed that history is marked by conflicts between oppressor and oppressed classes. In capitalist societies, this manifests as a struggle between the bourgeoisie (owners of the means of production) and the proletariat (working class). The bourgeoisie exploits the labor of the proletariat, leading to inherent tensions and eventual revolutionary change.

Furthermore, Marx introduced the concept of alienation, where workers become estranged from their labor, the products they produce, and their own human potential. This alienation arises because workers have little control over the production process and are often reduced to mere cogs in the capitalist machine.

\subsection{Intellectual Influences}

\begin{itemize}
    \item \textbf{German Philosophy:} Marx was influenced by \textbf{G.W.F. Hegel’s dialectics} (the idea that history progresses through contradictions) but rejected Hegel’s idealism, instead developing dialectical materialism (focusing on material conditions rather than ideas).
    \item \textbf{French Socialism & Revolution:} Early socialist thinkers like \textbf{Henri de Saint-Simon} and \textbf{Charles Fourier}, along with the radical politics of the \textbf{French Revolution}, shaped Marx’s vision of class struggle.
\end{itemize}

\subsection{Evolution into Marxist Movements}
After Marx’s death, Engels and later thinkers (like Lenin, Rosa Luxemburg, and Trotsky) adapted Marxism into different strands:

\begin{itemize}
    \item \textbf{Orthodox Marxism:} Strict adherence to Marx’s theories.
    \item \textbf{Leninism:} Revolutionary vanguard party leading the proletariat (basis for Soviet communism).
    \item \textbf{Western Marxism:} Focus on culture and ideology (e.g., Gramsci, Frankfurt School).
\end{itemize}

\section{Marxism in Practice}

\section{Criticisms}
\section{Modern Relevance}
\section{Conclusion}


\end{document}
