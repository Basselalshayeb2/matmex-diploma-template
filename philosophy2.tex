% !TEX TS-program = xelatex
% !BIB program = bibtex
% !TeX spellcheck = ru\_RU

\documentclass[14pt, russian]{matmex-diploma-custom}

% !TeX spellcheck = ru_RU
% !TEX root = vkr.tex
% Опциональные добавления используемых пакетов. Вполне может быть, что они вам не понадобятся, но в шаблоне приведены примеры их использования.
\usepackage{tikz} % Мощный пакет для создание рисунков, однако может очень сильно замедлять компиляцию
\usetikzlibrary{decorations.pathreplacing,calc,shapes,positioning,tikzmark}

% Библиотека для TikZ, которая генерирует отдельные файлы для каждого рисунка
% Позволяет ускорить компиляцию, однако имеет свои ограничения
% Например, ломает пример выделения кода в листинге из шаблона
% \usetikzlibrary{external}
% \tikzexternalize[prefix=figures/]

\newcounter{tmkcount}

\tikzset{
    use tikzmark/.style={
            remember picture,
            overlay,
            execute at end picture={
                    \stepcounter{tmkcount}
                },
        },
    tikzmark suffix={-\thetmkcount}
}

\usepackage{booktabs} % Пакет для верстки "более книжных" таблиц, вполне годится для оформления результатов
% В шаблоне есть команда \multirowcell, которой нужен этот пакет.
\usepackage{multirow}
\usepackage{siunitx} % для таблиц с единицами измерений

% Для названий стоит использовать \textsc{}
\newcommand{\OCaml}{\textsc{OCaml}}
\newcommand{\miniKanren}{\textsc{miniKanren}}
\newcommand{\BibTeX}{\textsc{BibTeX}}
\newcommand{\vsharp}{\textsc{V$\sharp$}}
\newcommand{\fsharp}{\textsc{F$\sharp$}}
\newcommand{\csharp}{\textsc{C\#}}
\newcommand{\GitHub}{\textsc{GitHub}}
\newcommand{\SMT}{\textsc{SMT}}

\definecolor{eclipseGreen}{RGB}{63,127,95}
% \lstdefinelanguage{ocaml}{
% keywords={@type, function, fun, let, in, match, with, when, class, type,
% nonrec, object, method, of, rec, repeat, until, while, not, do, done, as, val, inherit, and,
% new, module, sig, deriving, datatype, struct, if, then, else, open, private, virtual, include, success, failure,
% lazy, assert, true, false, end},
% sensitive=true,
% commentstyle=\small\itshape\ttfamily,
% keywordstyle=\ttfamily\bfseries, %\underbar,
% identifierstyle=\ttfamily,
% basewidth={0.5em,0.5em},
% columns=fixed,
% fontadjust=true,
% literate={->}{{$\to$}}3 {===}{{$\equiv$}}1 {=/=}{{$\not\equiv$}}1 {|>}{{$\triangleright$}}3 {\\/}{{$\vee$}}2 {/\\}{{$\wedge$}}2 {>=}{{$\ge$}}1 {<=}{{$\le$}} 1,
% morecomment=[s]{(*}{*)}
% }

\makeatletter
\@ifclassloaded{beamer}{
    %%% Обязательные пакеты
    %% Beamer
    \usepackage{beamerthemesplit}
    \usetheme{SPbGU}
    \beamertemplatenavigationsymbolsempty
    \usepackage{appendixnumberbeamer}

    %% Локализация
    \usepackage{fontspec}
    \setmainfont{CMU Serif}
    \setsansfont{CMU Sans Serif}
    \setmonofont{CMU Typewriter Text}
    %\setmonofont{Fira Code}[Contextuals=Alternate,Scale=0.9]
    %\setmonofont{Inconsolata}
    \usepackage{polyglossia}
    \setmainlanguage{russian}
    \setotherlanguage{english}

    %% Графика
    \usepackage{pdfpages} % Позволяет вставлять многостраничные pdf документы в текст

    % Математические окружения с русским названием
    \newtheorem{rutheorem}{Теорема}
    \newtheorem{ruproof}{Доказательство}
    \newtheorem{rudefinition}{Определение}
    \newtheorem{rulemma}{Лемма}
    \usepackage{fancyvrb}
}
{}
\makeatother

\usepackage[autostyle]{csquotes} % Правильные кавычки в зависимости от языка
\usepackage{totcount}
\usepackage{setspace}
\usepackage{amsmath, amsfonts, amssymb, amsthm, mathtools} % "Адекватная" работа с математикой в LaTeX



\begin{document}

% !TeX spellcheck = ru_RU
% !TEX root = vkr.tex


%% Если что-то забыли, при компиляции будут ошибки Undefined control sequence \my@title@<что забыли>@ru
%% Если англоязычная титульная страница не нужна, то ее можно просто удалить.
\filltitle{ru}{
    date = {2 5 2025},
    %% Актуально только для курсовых/практик. ВКР защищаются не на кафедре а в ГЭК по направлению,
    %%   и к моменту защиты вы будете уже не в группе.
    chair              = {Программная инженерия},
    group              = {24.М71-мм},
    %
    %% Макрос filltitle ненавидит пустые строки, поэтому обязателен хотя бы символ комментария на строке
    %% Актуально всем.
    title              = {Марксизм: теория и практика},
    %
    %% Здесь указывается тип работы. Возможные значения:
    %%   production - производственная практика;
    %%   coursework - отчёт по курсовой работе (ОБРАТИТЕ ВНИМАНИЕ, у техпрога и ПИ нет курсовых, только практики);
    %%   practice - отчёт по учебной практике;
    %%   prediploma - отчёт по преддипломной практике;
    %%   master - ВКР магистра;
    %%   bachelor - ВКР бакалавра.
    type               = {practice},
    %
    %% Здесь указывается вид работы. От вида работы зависят критерии оценивания.
    %%   solution - «Решение». Обучающемуся поручили найти способ решения проблемы в области разработки программного обеспечения или теоретической информатики с учётом набора ограничений.
    %%   experiment - «Эксперимент». Обучающемуся поручили изучить возможности, достоинства и недостатки новой технологии, платформы, языка и т. д. на примере какой-то задачи.
    %%   production - «Производственное задание». Автору поручили реализовать потенциально полезное программное обеспечение.
    %%   comparison - «Сравнение». Обучающемуся поручили сравнить несколько существующих продуктов и/или подходов.
    %%   theoretical - «Теоретическое исследование». Автору поручили доказать какое-то утверждение, исследовать свойства алгоритма и т.п., при этом не требуя написания кода.
    kind               = {none},
    %
    author             = {Альшаеб Басель},
    %
    %% Актуально только для ВКР. Указывается код и название направления подготовки. Типичные примеры:
    %%   02.03.03 \enquote{Математическое обеспечение и администрирование информационных систем}
    %%   02.04.03 \enquote{Математическое обеспечение и администрирование информационных систем}
    %%   09.03.04 \enquote{Программная инженерия}
    %%   09.04.04 \enquote{Программная инженерия}
    %% Те, что с 03 в середине --- бакалавриат, с 04 --- магистратура.
    specialty          = {09.04.04 \enquote{Математическое обеспечение и администрирование информационных систем}},
    %
    %% Актуально только для ВКР. Указывается шифр и название образовательной программы. Типичные примеры:
    %%   СВ.5162.2020 \enquote{Технологии программирования}
    %%   СВ.5080.2020 \enquote{Программная инженерия}
    %%   ВМ.5665.2022 \enquote{Математическое обеспечение и администрирование информационных систем}
    %%   ВМ.5666.2022 \enquote{Программная инженерия}
    %% Шифр и название программы можно посмотреть в учебном плане, по которому вы учитесь.
    %% СВ.* --- бакалавриат, ВМ.* --- магистратура. В конце --- год поступления (не обязательно ваш, если вы были в академе/вылетали).
    programme          = {ВМ.5666.2024 \enquote{Технологии программирования}},
    %
    %% Актуально всем.
    %% Должно умещаться в одну строчку, допускается использование сокращений, но без переусердствования,
    %% короткая строка с большим количеством сокращений выглядит странно
    %supervisorPosition = {проф. кафeдры системного программирования, д.ф.-м.н.,}, % Терехов А. Н.
    %supervisorPosition = {ст. преподаватель кафедры ИАС, к.~ф.-м.~н. (если есть),}, % Смирнов К. К.
    supervisorPosition = {},
    supervisor         = {},
    teacher            = {Секацкий Александр}
    %
    %% Актуально только для практик и курсовых. Если консультанта нет или он совпадает с научником, закомментировать или удалить вовсе.
    % consultantPosition = {должность, ООО \enquote{Место работы}, степень  (если есть),},
    % consultant         = {Консультант~К.~К.},
    %
    %% Актуально только для ВКР.
    % reviewerPosition   = {должность, ООО \enquote{Место работы}, степень (если есть),},
    % reviewer           = {Рецензент~Р.~Р.},
}

% Английский титульник нужен только для ВКР, остальные виды работ могут его смело игнорировать.
\filltitle{en}{
    chair              = {Advisor's chair},
    group              = {ХХ.BХХ-mm},
    title              = {Template for SPbU qualification works},
    type               = {bachelor},
    author             = {FirstName Surname},
    %
    %% Possible choices:
    %%   02.03.03 \foreignquote{english}{Software and Administration of Information Systems}
    %%   02.04.03 \foreignquote{english}{Software and Administration of Information Systems}
    %%   09.03.04 \foreignquote{english}{Software Engineering}
    %%   09.04.04 \foreignquote{english}{Software Engineering}
    %% Те, что с 03 в середине --- бакалавриат, с 04 --- магистратура.
    specialty          = {02.03.03 \foreignquote{english}{Software and Administration of Information Systems}},
    %
    %% Possible choices:
    %%   СВ.5162.2020 \foreignquote{english}{Programming Technologies}
    %%   СВ.5080.2020 \foreignquote{english}{Software Engineering}
    %%   ВМ.5665.2022 \foreignquote{english}{Software and Administration of Information Systems}
    %%   ВМ.5666.2022 \foreignquote{english}{Software Engineering}
    programme          = {СВ.5162.2020 \foreignquote{english}{Programming Technologies}},
    %
    %% Note that common title translations are:
    %%   кандидат наук --- C.Sc. (NOT Ph.D.)
    %%   доктор ... наук --- Sc.D.
    %%   доцент --- docent (NOT assistant/associate prof.)
    %%   профессор --- prof.
    supervisorPosition = {Sc.D, prof.},
    supervisor         = {S.S. Supervisor},
    %
    consultantPosition = {position at \foreignquote{english}{Company}, degree if present},
    consultant         = {C.C. Consultant},
    %
    reviewerPosition   = {position at \foreignquote{english}{Company}, degree if present},
    reviewer           = {R.R. Reviewer},
}

\maketitle

Немногие политические теории оказали на современную историю столь глубокое и одновременно противоречивое влияние, как марксизм. От революций до ожесточённых академических споров — его наследие остаётся актуальным. Сформировавшись в середине XIX века, марксизм, прежде всего, был разработан Карлом Марксом (1818–1883) и Фридрихом Энгельсом (1820–1895). Он возник как революционная критика капитализма и стал теоретическим основанием для социализма и коммунизма. Со временем марксизм перешёл от абстрактной теории к практическому руководству, повлиявшему на революции, политику и академические исследования по всему миру.

Теоретические основы марксизма заключаются, прежде всего, в историческом материализме \cite{marx1845}. Согласно этой концепции, именно материальные условия и экономическая деятельность служат основными двигателями общественного развития. Маркс утверждал, что способ производства в любом обществе определяет его социальные структуры \cite[Гл.~1]{marx1867} и отношения. Экономический базис, таким образом, оказывает определяющее влияние на культурную и политическую надстройки.

Другой ключевой элемент марксистской теории — классовая борьба. Маркс считал, что история человечества \cite[с.~79]{marx1848} — это история борьбы между угнетателями и угнетёнными. В капиталистическом обществе это выражается в противостоянии буржуазии (владельцев средств производства) и пролетариата (рабочего класса). Буржуазия эксплуатирует труд пролетариата, что порождает внутренние противоречия и ведёт к неизбежной революции.

Также Маркс ввёл понятие отчуждения — состояния, при котором рабочий становится чуждым своему труду, его продуктам и самому себе. Это отчуждение проистекает из отсутствия контроля над производственным процессом и превращения человека в "винтик" капиталистической системы.

Философия Маркса не возникла в вакууме. Её истоки уходят в немецкую классическую философию, особенно в диалектику Г. В. Ф. Гегеля  \cite{hegel1807} — идею, что развитие происходит через противоречия. Однако Маркс отверг идеализм Гегеля, развив вместо него диалектический материализм, сосредоточенный на материальных условиях как источнике изменений. Французский социализм и политическое наследие Великой французской революции также оказали влияние на акцент Маркса на классовой борьбе и социальной трансформации.

После смерти Маркса его идеи не остались неизменными. Энгельс, а затем такие мыслители, как Ленин, Роза Люксембург и Троцкий, адаптировали и развивали марксистскую теорию. Это привело к формированию различных направлений:
\begin{itemize}
    \item Ортодоксальный марксизм, строго следовавший текстам Маркса;
    \item Ленинизм \cite{lenin1902}, предложивший идею авангардной партии, ведущей пролетариат к революции и ставшей основой советского коммунизма;
    \item Западный марксизм, более сосредоточенный на культуре и идеологии, в лице таких фигур, как Антонио Грамши и представители Франкфуртской школы \cite{horkheimer1947}.
\end{itemize}

Переход марксистской теории к практике был как преобразующим, так и противоречивым. XX век ознаменовался революциями, вдохновлёнными марксистскими принципами — в России, Китае и на Кубе. Эти движения стремились разрушить капиталистические структуры и построить социалистические государства. В Советском Союзе революция 1917 года привела к созданию социалистического государства под руководством Ленина. Однако реализация марксистских идеалов столкнулась с проблемами авторитаризма и экономических трудностей. В Китае маоистская революция адаптировала марксизм к местным условиям, приведя к радикальным реформам, но и к потрясениям — в частности, во время Культурной революции. Эти примеры иллюстрируют сложности применения марксистской теории в различных социально-политических условиях.

Несмотря на влияние, марксизм подвергался и продолжает подвергаться критике. В политическом плане ему вменяют отсутствие ясной модели управления после свержения капитализма. Маркс подробно описал предпосылки революции, но уделил меньше внимания построению нового общества. Это привело к разнообразным интерпретациям и результатам, включая авторитарные режимы.

С экономической точки зрения капитализм оказался более устойчивым, чем предсказывал Маркс. Он адаптировался через реформы, технологии и глобализацию. В научном контексте марксизм критикуется за отсутствие эмпирических методов и опору на диалектическую логику, которую некоторые считают неопределённой. С философской позиции его материализм и атеизм рассматриваются как редуцирующие человеческие ценности к экономическим факторам, игнорируя мораль, духовность и индивидуальность.

Тем не менее марксизм остаётся важным аналитическим инструментом. Современные кризисы, неравенство и нестабильность труда вновь актуализируют его критику капитализма. Его идеи применяются для анализа таких феноменов, как платформа занятости, глобализация и цифровой капитализм.

В заключение, марксизм предлагает глубокое понимание социальной и экономической динамики. Хотя его реализация в истории вызвала споры и разочарования, теоретические основы продолжают быть актуальными в борьбе с несправедливостью. В условиях глобального неравенства, технологических изменений и экологических вызовов марксизм не теряет своей значимости — как минимум, как вызов существующему порядку.


\setmonofont{CMU Typewriter Text}
\bibliographystyle{ugost2008ls}
\bibliography{philosophy_bib}

\end{document}
