% !TEX TS-program = xelatex
% !BIB program = bibtex
% !TeX spellcheck = ru\_RU

\documentclass[14pt, russian]{matmex-diploma-custom}

\input{preamble.tex}

\begin{document}

\input{philosophy_title.tex}
\maketitle

Немногие политические теории оказали на современную историю столь глубокое и одновременно противоречивое влияние, как марксизм. От революций до ожесточённых академических споров — его наследие остаётся актуальным. Сформировавшись в середине XIX века, марксизм, прежде всего, был разработан Карлом Марксом (1818–1883) и Фридрихом Энгельсом (1820–1895). Он возник как революционная критика капитализма и стал теоретическим основанием для социализма и коммунизма. Со временем марксизм перешёл от абстрактной теории к практическому руководству, повлиявшему на революции, политику и академические исследования по всему миру.

Теоретические основы марксизма заключаются, прежде всего, в историческом материализме \cite{marx1845}. Согласно этой концепции, именно материальные условия и экономическая деятельность служат основными двигателями общественного развития. Маркс утверждал, что способ производства в любом обществе определяет его социальные структуры \cite[Гл.~1]{marx1867} и отношения. Экономический базис, таким образом, оказывает определяющее влияние на культурную и политическую надстройки.

Другой ключевой элемент марксистской теории — классовая борьба. Маркс считал, что история человечества \cite[с.~79]{marx1848} — это история борьбы между угнетателями и угнетёнными. В капиталистическом обществе это выражается в противостоянии буржуазии (владельцев средств производства) и пролетариата (рабочего класса). Буржуазия эксплуатирует труд пролетариата, что порождает внутренние противоречия и ведёт к неизбежной революции.

Также Маркс ввёл понятие отчуждения — состояния, при котором рабочий становится чуждым своему труду, его продуктам и самому себе. Это отчуждение проистекает из отсутствия контроля над производственным процессом и превращения человека в "винтик" капиталистической системы.

Философия Маркса не возникла в вакууме. Её истоки уходят в немецкую классическую философию, особенно в диалектику Г. В. Ф. Гегеля  \cite{hegel1807} — идею, что развитие происходит через противоречия. Однако Маркс отверг идеализм Гегеля, развив вместо него диалектический материализм, сосредоточенный на материальных условиях как источнике изменений. Французский социализм и политическое наследие Великой французской революции также оказали влияние на акцент Маркса на классовой борьбе и социальной трансформации.

После смерти Маркса его идеи не остались неизменными. Энгельс, а затем такие мыслители, как Ленин, Роза Люксембург и Троцкий, адаптировали и развивали марксистскую теорию. Это привело к формированию различных направлений:
\begin{itemize}
    \item Ортодоксальный марксизм, строго следовавший текстам Маркса;
    \item Ленинизм \cite{lenin1902}, предложивший идею авангардной партии, ведущей пролетариат к революции и ставшей основой советского коммунизма;
    \item Западный марксизм, более сосредоточенный на культуре и идеологии, в лице таких фигур, как Антонио Грамши и представители Франкфуртской школы \cite{horkheimer1947}.
\end{itemize}

Переход марксистской теории к практике был как преобразующим, так и противоречивым. XX век ознаменовался революциями, вдохновлёнными марксистскими принципами — в России, Китае и на Кубе. Эти движения стремились разрушить капиталистические структуры и построить социалистические государства. В Советском Союзе революция 1917 года привела к созданию социалистического государства под руководством Ленина. Однако реализация марксистских идеалов столкнулась с проблемами авторитаризма и экономических трудностей. В Китае маоистская революция адаптировала марксизм к местным условиям, приведя к радикальным реформам, но и к потрясениям — в частности, во время Культурной революции. Эти примеры иллюстрируют сложности применения марксистской теории в различных социально-политических условиях.

Несмотря на влияние, марксизм подвергался и продолжает подвергаться критике. В политическом плане ему вменяют отсутствие ясной модели управления после свержения капитализма. Маркс подробно описал предпосылки революции, но уделил меньше внимания построению нового общества. Это привело к разнообразным интерпретациям и результатам, включая авторитарные режимы.

С экономической точки зрения капитализм оказался более устойчивым, чем предсказывал Маркс. Он адаптировался через реформы, технологии и глобализацию. В научном контексте марксизм критикуется за отсутствие эмпирических методов и опору на диалектическую логику, которую некоторые считают неопределённой. С философской позиции его материализм и атеизм рассматриваются как редуцирующие человеческие ценности к экономическим факторам, игнорируя мораль, духовность и индивидуальность.

Тем не менее марксизм остаётся важным аналитическим инструментом. Современные кризисы, неравенство и нестабильность труда вновь актуализируют его критику капитализма. Его идеи применяются для анализа таких феноменов, как платформа занятости, глобализация и цифровой капитализм.

В заключение, марксизм предлагает глубокое понимание социальной и экономической динамики. Хотя его реализация в истории вызвала споры и разочарования, теоретические основы продолжают быть актуальными в борьбе с несправедливостью. В условиях глобального неравенства, технологических изменений и экологических вызовов марксизм не теряет своей значимости — как минимум, как вызов существующему порядку.


\setmonofont{CMU Typewriter Text}
\bibliographystyle{ugost2008ls}
\bibliography{philosophy_bib}

\end{document}
