% !TeX spellcheck = ru_RU
% !TEX root = vkr.tex

\section*{Введение}
\thispagestyle{withCompileDate}
Современные сети являются ключевым элементом информационных технологий, обеспечивая обмен данными между устройствами и сервисами.
Для тестирования и исследования сетевых топологий широко используются программные симуляторы, такие как Miminet\cite{miminet}, разработанный на базе Mininet\cite{mininet}.

Они позволяют эмулировать сети на одном хосте, создавая виртуальные узлы и соединения.
Однако с увеличением сложности и размеров симуляций возрастают требования к вычислительным ресурсам, что приводит к снижению производительности и ограничению масштабируемости.
Традиционный подход к запуску Miminet неэффективно использует ресурсы хоста.
В частности, вычислительная нагрузка концентрируется в одном процессе, что не позволяет использовать все доступные ядра процессора.
Существующие решения не уделяют должного внимания вопросу распределения симуляций по нескольким контейнерам с целью более равномерного использования вычислительной мощности хоста.

В отличие от подходов, ориентированных на параллельные вычисления, основной целью является эффективное использование вычислительных ресурсов хоста путем распределения отдельных симуляций по разным контейнерам.
Это позволит избежать перегрузки одного процесса и обеспечить более стабильную работу по мере увеличения количества моделируемых узлов.
