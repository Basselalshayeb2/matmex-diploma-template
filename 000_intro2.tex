% !TeX spellcheck = ru_RU
% !TEX root = vkr.tex

\section*{Введение}
\thispagestyle{withCompileDate}
% in matmex the course is being course. to complete the practical section we should add like
% connecting routers with small networks and siwtches - to buy showing examples of switches routers - packet racing - virtual networking but it's hard to maintain.
% We should talk about our university and matmex.
% and it is used mainly in matmex.

Курс по компьютерным сетям\cite{stepik} является обязательным в рамках учебной программы Математико-механического факультета, так как он играет ключевую роль в подготовке специалистов по программной инженерии, системному администрированию и других направлений, связанных с разработкой и эксплуатацией информационных систем.
Этот курс включает в себя различные разделы, которые должны быть продемонстрированы на практике с использованием реальных примеров и ситуаций.
Однако для эффективного освоения материала необходимо не только изучение теории, но и практическая работа с сетевыми конфигурациями.

Для того чтобы реализовать практическую часть курса, можно приобрести физическое оборудование, такое как коммутаторы и маршрутизаторы.
Например, коммутатор Cisco\cite{cisco} Catalyst 3500/3700 серии, который используется в образовательных целях, может стоить до 2000 долларов США за один блок. Вдобавок, для создания полноценной лаборатории потребуется достаточно большое количество оборудования, что приводит к дополнительным расходам на кабели, сетевые карты и другие элементы, которые могут стоить тысячи долларов.
Например, качественные Ethernet-кабели категории 6 с длиной 10 метров стоят около 10-20 долларов за штуку.
Чтобы организовать несколько лабораторий с необходимым оборудованием, затраты на оборудование и инфраструктуру могут легко превысить десятки тысяч долларов.

Однако, в 2022 году компания Cisco Systems покинула российский рынок, что значительно усложнило процесс приобретения оборудования, сделав его не только более дорогим, но и труднодоступным.
Это создало дополнительные сложности для образовательных учреждений, в том числе для Математико-механического факультета, где курс по компьютерным сетям является обязательным для студентов.
Возникла необходимость находить альтернативные решения, которые позволили бы проводить практические занятия по сетям, не полагаясь на дорогие физические устройства.

Для решения этой проблемы был выбран эмулятор Miminet, который был разработан на основе Mininet и позволяет создавать, конфигурировать и тестировать сети в виртуальной среде.
Этот эмулятор предоставляет возможность проведения практических занятий с минимальными затратами и делает курс доступным и удобным для образовательных целей.
Благодаря эмуляции, студенты могут работать с сетями в виртуальной среде, не требуя физического оборудования.
Это позволяет значительно снизить расходы на материально-техническое обеспечение курса и предоставляет возможность масштабировать учебный процесс, не ограничиваясь количеством доступных физических устройств.

Однако, несмотря на явные преимущества эмуляторов, проблема с долгим временем ожидания студентов всё ещё остаётся актуальной. Когда большое количество студентов одновременно выполняет задания в эмуляторе, каждый из них должен ждать своей очереди, что существенно снижает эффективность обучения и замедляет процесс усвоения материала.

В связи с этим возникает необходимость разработки решения для улучшения производительности и масштабируемости Miminet\cite{miminet}.
Для этого будет реализован механизм распределения нагрузки между несколькими контейнерами, что позволит значительно уменьшить время ожидания студентов иповысить эффективность выполнения симуляций.
В частности, система будет использовать динамическое распределение ресурсов и многопроцессорную обработку, что обеспечит более гибкое управление и оптимизацию работы эмулятора.
Эти улучшения позволят создать более стабильную и эффективную среду для обучения сетевым технологиям.
