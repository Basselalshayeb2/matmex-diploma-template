% !TeX spellcheck = ru_RU
% !TEX root = vkr.tex

\section*{Введение}
\thispagestyle{withCompileDate}
% 1- training in univer hard with buying equep and cisco went out
% 2- course with practical with link
% 3- people teaching this yandex metric : link
% 4- proplem inside the lecture all student test the lecture and students wait too much time



Обучение компьютерным сетям играет ключевую роль в подготовке специалистов по программной инженерии, системному администрированию и других направлений, связанных с разработкой и эксплуатацией информационных систем.
Для эффективного освоения этой области необходимо не только изучение теории, но и практическая работа с сетевыми конфигурациями.
Однако использование физического оборудования для обучения зачастую связано с высокими затратами и сложностью организации.

Для решения этой проблемы многие университеты используют эмуляторы, которые позволяют моделировать сетевые взаимодействия и проводить практические занятия без необходимости в реальном оборудовании.
Одним из таких инструментов является Miminet\cite{miminet}, разработанный на базе Mininet\cite{mininet}. Miminet\cite{miminet} позволяет создавать, конфигурировать и тестировать сети в виртуальной среде, что делает его доступным и удобным для образовательных целей.

Несмотря на явные преимущества использования эмуляторов, в Miminet\cite{miminet} существует проблема с долгим временем ожидания, которая ограничивает эффективность обучения. Когда большое количество студентов одновременно выполняет задания в эмуляторе, каждый из них должен ждать своей очереди, что существенно замедляет процесс обучения.
Эта проблема усугубляется ограничением вычислительных ресурсов, так как текущая архитектура эмулятора использует один контейнер для всех симуляций, что приводит к перегрузке системы.

В связи с этим возникает необходимость разработки решения для улучшения производительности и масштабируемости Miminet\cite{miminet}.
Для этого будет реализован механизм распределения нагрузки между несколькими контейнерами, что позволит значительно уменьшить время ожидания студентов иповысить эффективность выполнения симуляций.
В частности, система будет использовать динамическое распределение ресурсов и многопроцессорную обработку, что обеспечит более гибкое управление и оптимизацию работы эмулятора.
Эти улучшения позволят создать более стабильную и эффективную среду для обучения сетевым технологиям.
