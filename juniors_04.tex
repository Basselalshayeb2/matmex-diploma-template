% !TEX TS-program = xelatex
% !BIB program = bibtex
% !TeX spellcheck = ru_RU

% About magic macros see also
% https://tex.stackexchange.com/questions/78101/

% По умолчанию используется шрифт 14 размера.
% Если Вы не влезаете в лимит страниц и нужен 12-й шрифт,
% то уберите опцию [14pt]

\documentclass[14pt, russian]{matmex-diploma-custom}


\input{preamble.tex}

\begin{document}

\input{junior_04_title.tex}
\maketitle
\section*{Задание}
    Сделайте технико-коммерческое предложение (ТКП) на выполнение вашего проекта (вы - подрядчик).
    Примерный состав:
    \begin{itemize}
        \item Информация о компании
        \item Рамки проекта
        \item Описание решения
        \item Процедуры управления проектом
        \item Команда
        \item Оценка проекта
        \item Стоимость и условия оплаты
\end{itemize}

\section{Информация о компании}

\subsection{Общие сведения}

\paragraph{Полное наименование компании.}
Общество с ограниченной ответственностью «Gym Tech»

\paragraph{Юридический адрес.}
123456, г. Москва, ул. Примерная, д. 10, офис 101

\paragraph{Контактная информация.}
Телефон: +7 (495) 123-45-67 \\
Email: \texttt{contact@gym.tech} \\
Веб-сайт: \texttt{https://gym.tech}

\paragraph{Основной вид деятельности.}
Разработка программного обеспечения для автоматизации бизнес-процессов в сфере фитнеса и здравоохранения.
Основные направления включают:
\begin{itemize}[noitemsep]
  \item Создание цифровых платформ для управления фитнес-клубами;
  \item Разработка клиентских и администраторских интерфейсов;
  \item Интеграция с платёжными и аналитическими сервисами;
  \item Разработка мобильных приложений.
\end{itemize}

\subsection{Структура собственности}

Компания зарегистрирована в форме общества с ограниченной ответственностью (ООО). Учредителем и генеральным директором является Кисельков Денис Андреевич.

Структура собственности — 100\% частная.

\subsection{Финансово-экономические показатели}

По состоянию на 2024 год:

\begin{itemize}[noitemsep]
  \item Годовой оборот: 12 000 000 руб.
  \item Чистая прибыль: 3 000 000 руб.
  \item Среднегодовая численность персонала: 15 человек
  \item Устойчивый рост: 15–20\% в год
\end{itemize}

\subsection{Система менеджмента качества}

Компания внедрила внутреннюю систему контроля качества, основанную на принципах ISO 9001:
\begin{itemize}[noitemsep]
  \item Чек-листы код-ревью и тестирования;
  \item CI/CD процессы с автоматизированной сборкой и тестированием;
  \item Документирование требований и технических решений;
  \item Обратная связь от заказчиков после завершения этапов.
\end{itemize}

\subsection{Процедуры управления проектами}

Проекты ведутся по гибкой методологии Scrum. Основные практики:
\begin{itemize}[noitemsep]
  \item Планирование спринтов длительностью 2 недели
  \item Ежедневные стендапы
  \item Промежуточные демонстрации результатов
  \item Использование Jira для трекинга задач
\end{itemize}

\subsection{Квалификация и численность персонала}

В компании трудятся 4 специалистов:
\begin{itemize}[noitemsep]
  \item Руководители проектов — 1
  \item Разработчики (backend/frontend) — 2
  \item Маркетинг/аналитика — 1
\end{itemize}

Средний стаж в ИТ — 4 года. Все сотрудники прошли профильное обучение и имеют опыт работы с крупными ИТ-системами.

\subsection{Квалификация по рынкам}

Основной рынок компании — цифровизация и автоматизация фитнес-индустрии

Компания адаптирует решения под российский рынок с учётом локальных требований по безопасности и законодательству.

\subsection{Ключевые заказчики}

\begin{itemize}[noitemsep]
  \item Сеть фитнес-клубов \textbf{FitLine} — CRM
  \item Онлайн-платформа \textbf{TrainMe} — разработка мобильного приложения
  \item Частные клиенты — \textbf{MVP}-проекты и консалтинг
\end{itemize}

\section{Рамки проекта}

\begin{itemize}
    \item \textbf{Проект:} Разработка и внедрение системы управления фитнес-клубом «MyGym» \\
    \item \textbf{Цель:} Повышение эффективности работы клуба за счёт цифровизации процессов, в том числе расписаний, подписок, аналитики и взаимодействия с клиентами. \\
    \item \textbf{Срок выполнения:} 2 месяца \\
    \item \textbf{Форма сдачи:} Веб-приложение + мобильное приложение, документация, Первоначальный маркетинг
\end{itemize}

\section{Описание решения}

Будет реализована платформа, включающая:

\begin{itemize}[noitemsep]
    \item Клиентский портал: регистрация, расписание, оплата, онлайн-занятия
    \item Админ-панель: управление залом, тренерами, расписаниями, клиентами
    \item Модуль аналитики: загрузка и анализ метрик посещаемости, доходов, занятости
    \item Интеграция с платёжными системами
    \item Мобильное приложение (PWA или нативное)
\end{itemize}

\subsection{Технологии}

\begin{itemize}[noitemsep]
    \item Backend: PHP / Laravel
    \item Frontend: Vue.js / Nuxt.js
    \item База данных: PostgreSQL
    \item CI/CD, облачный хостинг
\end{itemize}


\section{Процедуры управления проектом}

\begin{itemize}[noitemsep]
    \item Методология: Agile / Scrum
    \item Спринты: каждые 2 недели
    \item Отчётность: еженедельные демо и отчёты
    \item GitHub с системой код-ревью
    \item Инструменты управления: Jira, Notion
    \item Управление рисками и изменениями — через Change Request
\end{itemize}


\section{Команда проекта}

\begin{longtable}{|p{5cm}|p{5cm}|p{4cm}|}
    \hline
    \textbf{Роль} & \textbf{Имя} & \textbf{Опыт} \\
    \hline
    Руководитель проекта & Иванов И.И. & 5+ лет в ИТ-проектах \\
    \hline
    Разработчик Backend & Смирнов С.С. & Django, API, Docker \\
    \hline
    Разработчик Frontend & Петрова А.А. & Vue.js, UI/UX \\
    \hline
    QA / Тестировщик & Козлова Е.Е. & Ручное и автотестирование \\
    \hline
    DevOps & Григорьев Д.Д. & CI/CD, Docker, VPS \\
    \hline
    \end{longtable}


\section{Оценка проекта}

\subsection*{Экспертная оценка}

\begin{tabular}{|p{10cm}|p{4cm}|}
\hline
\textbf{Этап} & \textbf{Оценка (чел-дней)} \\
\hline
Планирование & 5 \\
\hline
Анализ и проектирование & 10 \\
\hline
Разработка (backend + frontend) & 40 \\
\hline
Тестирование & 10 \\
\hline
Развертывание и поддержка & 5 \\
\hline
\textbf{Итого} & \textbf{70 чел-дней} \\
\hline
\end{tabular}

\bigskip

Метод Use Case Points (UCP) может быть предоставлен по запросу.


\section{Стоимость и условия оплаты}

\textbf{Общая стоимость проекта:} 700 000 руб \\
(расчёт: 10 000 руб / чел-день × 70 чел-дней)

\bigskip

\textbf{Условия оплаты:}
\begin{itemize}[noitemsep]
    \item 30\% — предоплата (210 000 руб)
    \item 40\% — после завершения основного функционала
    \item 30\% — после сдачи и развёртывания проекта
\end{itemize}

Возможна поэтапная сдача по актам приёмки.

\end{document}
