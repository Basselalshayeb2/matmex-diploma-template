% !TEX TS-program = xelatex
% !BIB program = bibtex
% !TeX spellcheck = ru_RU

% About magic macros see also
% https://tex.stackexchange.com/questions/78101/

% По умолчанию используется шрифт 14 размера.
% Если Вы не влезаете в лимит страниц и нужен 12-й шрифт,
% то уберите опцию [14pt]

\documentclass[14pt, russian]{matmex-diploma-custom}


% !TeX spellcheck = ru_RU
% !TEX root = vkr.tex
% Опциональные добавления используемых пакетов. Вполне может быть, что они вам не понадобятся, но в шаблоне приведены примеры их использования.
\usepackage{tikz} % Мощный пакет для создание рисунков, однако может очень сильно замедлять компиляцию
\usetikzlibrary{decorations.pathreplacing,calc,shapes,positioning,tikzmark}

% Библиотека для TikZ, которая генерирует отдельные файлы для каждого рисунка
% Позволяет ускорить компиляцию, однако имеет свои ограничения
% Например, ломает пример выделения кода в листинге из шаблона
% \usetikzlibrary{external}
% \tikzexternalize[prefix=figures/]

\newcounter{tmkcount}

\tikzset{
    use tikzmark/.style={
            remember picture,
            overlay,
            execute at end picture={
                    \stepcounter{tmkcount}
                },
        },
    tikzmark suffix={-\thetmkcount}
}

\usepackage{booktabs} % Пакет для верстки "более книжных" таблиц, вполне годится для оформления результатов
% В шаблоне есть команда \multirowcell, которой нужен этот пакет.
\usepackage{multirow}
\usepackage{siunitx} % для таблиц с единицами измерений

% Для названий стоит использовать \textsc{}
\newcommand{\OCaml}{\textsc{OCaml}}
\newcommand{\miniKanren}{\textsc{miniKanren}}
\newcommand{\BibTeX}{\textsc{BibTeX}}
\newcommand{\vsharp}{\textsc{V$\sharp$}}
\newcommand{\fsharp}{\textsc{F$\sharp$}}
\newcommand{\csharp}{\textsc{C\#}}
\newcommand{\GitHub}{\textsc{GitHub}}
\newcommand{\SMT}{\textsc{SMT}}

\definecolor{eclipseGreen}{RGB}{63,127,95}
% \lstdefinelanguage{ocaml}{
% keywords={@type, function, fun, let, in, match, with, when, class, type,
% nonrec, object, method, of, rec, repeat, until, while, not, do, done, as, val, inherit, and,
% new, module, sig, deriving, datatype, struct, if, then, else, open, private, virtual, include, success, failure,
% lazy, assert, true, false, end},
% sensitive=true,
% commentstyle=\small\itshape\ttfamily,
% keywordstyle=\ttfamily\bfseries, %\underbar,
% identifierstyle=\ttfamily,
% basewidth={0.5em,0.5em},
% columns=fixed,
% fontadjust=true,
% literate={->}{{$\to$}}3 {===}{{$\equiv$}}1 {=/=}{{$\not\equiv$}}1 {|>}{{$\triangleright$}}3 {\\/}{{$\vee$}}2 {/\\}{{$\wedge$}}2 {>=}{{$\ge$}}1 {<=}{{$\le$}} 1,
% morecomment=[s]{(*}{*)}
% }

\makeatletter
\@ifclassloaded{beamer}{
    %%% Обязательные пакеты
    %% Beamer
    \usepackage{beamerthemesplit}
    \usetheme{SPbGU}
    \beamertemplatenavigationsymbolsempty
    \usepackage{appendixnumberbeamer}

    %% Локализация
    \usepackage{fontspec}
    \setmainfont{CMU Serif}
    \setsansfont{CMU Sans Serif}
    \setmonofont{CMU Typewriter Text}
    %\setmonofont{Fira Code}[Contextuals=Alternate,Scale=0.9]
    %\setmonofont{Inconsolata}
    \usepackage{polyglossia}
    \setmainlanguage{russian}
    \setotherlanguage{english}

    %% Графика
    \usepackage{pdfpages} % Позволяет вставлять многостраничные pdf документы в текст

    % Математические окружения с русским названием
    \newtheorem{rutheorem}{Теорема}
    \newtheorem{ruproof}{Доказательство}
    \newtheorem{rudefinition}{Определение}
    \newtheorem{rulemma}{Лемма}
    \usepackage{fancyvrb}
}
{}
\makeatother

\usepackage[autostyle]{csquotes} % Правильные кавычки в зависимости от языка
\usepackage{totcount}
\usepackage{setspace}
\usepackage{amsmath, amsfonts, amssymb, amsthm, mathtools} % "Адекватная" работа с математикой в LaTeX



\begin{document}

% !TeX spellcheck = ru_RU
% !TEX root = vkr.tex


%% Если что-то забыли, при компиляции будут ошибки Undefined control sequence \my@title@<что забыли>@ru
%% Если англоязычная титульная страница не нужна, то ее можно просто удалить.
\filltitle{ru}{
    date = {27 марта 2025},
    %% Актуально только для курсовых/практик. ВКР защищаются не на кафедре а в ГЭК по направлению,
    %%   и к моменту защиты вы будете уже не в группе.
    chair              = {Программная инженерия},
    group              = {24.М71-мм},
    %
    %% Макрос filltitle ненавидит пустые строки, поэтому обязателен хотя бы символ комментария на строке
    %% Актуально всем.
    title              = {ТКП},
    %
    %% Здесь указывается тип работы. Возможные значения:
    %%   production - производственная практика;
    %%   coursework - отчёт по курсовой работе (ОБРАТИТЕ ВНИМАНИЕ, у техпрога и ПИ нет курсовых, только практики);
    %%   practice - отчёт по учебной практике;
    %%   prediploma - отчёт по преддипломной практике;
    %%   master - ВКР магистра;
    %%   bachelor - ВКР бакалавра.
    type               = {groupwork},
    %
    %% Здесь указывается вид работы. От вида работы зависят критерии оценивания.
    %%   solution - «Решение». Обучающемуся поручили найти способ решения проблемы в области разработки программного обеспечения или теоретической информатики с учётом набора ограничений.
    %%   experiment - «Эксперимент». Обучающемуся поручили изучить возможности, достоинства и недостатки новой технологии, платформы, языка и т. д. на примере какой-то задачи.
    %%   production - «Производственное задание». Автору поручили реализовать потенциально полезное программное обеспечение.
    %%   comparison - «Сравнение». Обучающемуся поручили сравнить несколько существующих продуктов и/или подходов.
    %%   theoretical - «Теоретическое исследование». Автору поручили доказать какое-то утверждение, исследовать свойства алгоритма и т.п., при этом не требуя написания кода.
    kind               = {none},
    %
    author             = {Juniors},
    %
    %% Актуально только для ВКР. Указывается код и название направления подготовки. Типичные примеры:
    %%   02.03.03 \enquote{Математическое обеспечение и администрирование информационных систем}
    %%   02.04.03 \enquote{Математическое обеспечение и администрирование информационных систем}
    %%   09.03.04 \enquote{Программная инженерия}
    %%   09.04.04 \enquote{Программная инженерия}
    %% Те, что с 03 в середине --- бакалавриат, с 04 --- магистратура.
    specialty          = {09.04.04 \enquote{Математическое обеспечение и администрирование информационных систем}},
    %
    %% Актуально только для ВКР. Указывается шифр и название образовательной программы. Типичные примеры:
    %%   СВ.5162.2020 \enquote{Технологии программирования}
    %%   СВ.5080.2020 \enquote{Программная инженерия}
    %%   ВМ.5665.2022 \enquote{Математическое обеспечение и администрирование информационных систем}
    %%   ВМ.5666.2022 \enquote{Программная инженерия}
    %% Шифр и название программы можно посмотреть в учебном плане, по которому вы учитесь.
    %% СВ.* --- бакалавриат, ВМ.* --- магистратура. В конце --- год поступления (не обязательно ваш, если вы были в академе/вылетали).
    programme          = {ВМ.5666.2024 \enquote{Технологии программирования}},
    %
    %% Актуально всем.
    %% Должно умещаться в одну строчку, допускается использование сокращений, но без переусердствования,
    %% короткая строка с большим количеством сокращений выглядит странно
    %supervisorPosition = {проф. кафeдры системного программирования, д.ф.-м.н.,}, % Терехов А. Н.
    %supervisorPosition = {ст. преподаватель кафедры ИАС, к.~ф.-м.~н. (если есть),}, % Смирнов К. К.
    supervisorPosition = {},
    supervisor         = {},
    teacher            = {Тимохин Д. В.}
    %
    %% Актуально только для практик и курсовых. Если консультанта нет или он совпадает с научником, закомментировать или удалить вовсе.
    % consultantPosition = {должность, ООО \enquote{Место работы}, степень  (если есть),},
    % consultant         = {Консультант~К.~К.},
    %
    %% Актуально только для ВКР.
    % reviewerPosition   = {должность, ООО \enquote{Место работы}, степень (если есть),},
    % reviewer           = {Рецензент~Р.~Р.},
}

% Английский титульник нужен только для ВКР, остальные виды работ могут его смело игнорировать.
\filltitle{en}{
    chair              = {Advisor's chair},
    group              = {ХХ.BХХ-mm},
    title              = {Template for SPbU qualification works},
    type               = {bachelor},
    author             = {FirstName Surname},
    %
    %% Possible choices:
    %%   02.03.03 \foreignquote{english}{Software and Administration of Information Systems}
    %%   02.04.03 \foreignquote{english}{Software and Administration of Information Systems}
    %%   09.03.04 \foreignquote{english}{Software Engineering}
    %%   09.04.04 \foreignquote{english}{Software Engineering}
    %% Те, что с 03 в середине --- бакалавриат, с 04 --- магистратура.
    specialty          = {02.03.03 \foreignquote{english}{Software and Administration of Information Systems}},
    %
    %% Possible choices:
    %%   СВ.5162.2020 \foreignquote{english}{Programming Technologies}
    %%   СВ.5080.2020 \foreignquote{english}{Software Engineering}
    %%   ВМ.5665.2022 \foreignquote{english}{Software and Administration of Information Systems}
    %%   ВМ.5666.2022 \foreignquote{english}{Software Engineering}
    programme          = {СВ.5162.2020 \foreignquote{english}{Programming Technologies}},
    %
    %% Note that common title translations are:
    %%   кандидат наук --- C.Sc. (NOT Ph.D.)
    %%   доктор ... наук --- Sc.D.
    %%   доцент --- docent (NOT assistant/associate prof.)
    %%   профессор --- prof.
    supervisorPosition = {Sc.D, prof.},
    supervisor         = {S.S. Supervisor},
    %
    consultantPosition = {position at \foreignquote{english}{Company}, degree if present},
    consultant         = {C.C. Consultant},
    %
    reviewerPosition   = {position at \foreignquote{english}{Company}, degree if present},
    reviewer           = {R.R. Reviewer},
}

\maketitle
\section*{Задание}
    Сделайте технико-коммерческое предложение (ТКП) на выполнение вашего проекта (вы - подрядчик).
    Примерный состав:
    \begin{itemize}
        \item Информация о компании
        \item Рамки проекта
        \item Описание решения
        \item Процедуры управления проектом
        \item Команда
        \item Оценка проекта
        \item Стоимость и условия оплаты
\end{itemize}

\section{Информация о компании}

\subsection{Общие сведения}

\paragraph{Полное наименование компании.}
Общество с ограниченной ответственностью «Gym Tech»

\paragraph{Юридический адрес.}
123456, г. Москва, ул. Примерная, д. 10, офис 101

\paragraph{Контактная информация.}
Телефон: +7 (495) 123-45-67 \\
Email: \texttt{contact@gym.tech} \\
Веб-сайт: \texttt{https://gym.tech}

\paragraph{Основной вид деятельности.}
Разработка программного обеспечения для автоматизации бизнес-процессов в сфере фитнеса и здравоохранения.
Основные направления включают:
\begin{itemize}[noitemsep]
  \item Создание цифровых платформ для управления фитнес-клубами;
  \item Разработка клиентских и администраторских интерфейсов;
  \item Интеграция с платёжными и аналитическими сервисами;
  \item Разработка мобильных приложений.
\end{itemize}

\subsection{Структура собственности}

Компания зарегистрирована в форме общества с ограниченной ответственностью (ООО). Учредителем и генеральным директором является Кисельков Денис Андреевич.

Структура собственности — 100\% частная.

\subsection{Финансово-экономические показатели}

По состоянию на 2024 год:

\begin{itemize}[noitemsep]
  \item Годовой оборот: 12 000 000 руб.
  \item Чистая прибыль: 3 000 000 руб.
  \item Среднегодовая численность персонала: 15 человек
  \item Устойчивый рост: 15–20\% в год
\end{itemize}

\subsection{Система менеджмента качества}

Компания внедрила внутреннюю систему контроля качества, основанную на принципах ISO 9001:
\begin{itemize}[noitemsep]
  \item Чек-листы код-ревью и тестирования;
  \item CI/CD процессы с автоматизированной сборкой и тестированием;
  \item Документирование требований и технических решений;
  \item Обратная связь от заказчиков после завершения этапов.
\end{itemize}

\subsection{Процедуры управления проектами}

Проекты ведутся по гибкой методологии Scrum. Основные практики:
\begin{itemize}[noitemsep]
  \item Планирование спринтов длительностью 2 недели
  \item Ежедневные стендапы
  \item Промежуточные демонстрации результатов
  \item Использование Jira для трекинга задач
\end{itemize}

\subsection{Квалификация и численность персонала}

В компании трудятся 4 специалистов:
\begin{itemize}[noitemsep]
  \item Руководители проектов — 1
  \item Разработчики (backend/frontend) — 2
  \item Маркетинг/аналитика — 1
\end{itemize}

Средний стаж в ИТ — 4 года. Все сотрудники прошли профильное обучение и имеют опыт работы с крупными ИТ-системами.

\subsection{Квалификация по рынкам}

Основной рынок компании — цифровизация и автоматизация фитнес-индустрии

Компания адаптирует решения под российский рынок с учётом локальных требований по безопасности и законодательству.

\subsection{Ключевые заказчики}

\begin{itemize}[noitemsep]
  \item Сеть фитнес-клубов \textbf{FitLine} — CRM
  \item Онлайн-платформа \textbf{TrainMe} — разработка мобильного приложения
  \item Частные клиенты — \textbf{MVP}-проекты и консалтинг
\end{itemize}

\section{Рамки проекта}

\begin{itemize}
    \item \textbf{Проект:} Разработка и внедрение системы управления фитнес-клубом «MyGym» \\
    \item \textbf{Цель:} Повышение эффективности работы клуба за счёт цифровизации процессов, в том числе расписаний, подписок, аналитики и взаимодействия с клиентами. \\
    \item \textbf{Срок выполнения:} 2 месяца \\
    \item \textbf{Форма сдачи:} Веб-приложение + мобильное приложение, документация, Первоначальный маркетинг
\end{itemize}

\section{Описание решения}

Будет реализована платформа, включающая:

\begin{itemize}[noitemsep]
    \item Клиентский портал: регистрация, расписание, оплата, онлайн-занятия
    \item Админ-панель: управление залом, тренерами, расписаниями, клиентами
    \item Модуль аналитики: загрузка и анализ метрик посещаемости, доходов, занятости
    \item Интеграция с платёжными системами
    \item Мобильное приложение (PWA или нативное)
\end{itemize}

\subsection{Технологии}

\begin{itemize}[noitemsep]
    \item Backend: PHP / Laravel
    \item Frontend: Vue.js / Nuxt.js
    \item База данных: PostgreSQL
    \item CI/CD, облачный хостинг
\end{itemize}


\section{Процедуры управления проектом}

\begin{itemize}[noitemsep]
    \item Методология: Agile / Scrum
    \item Спринты: каждые 2 недели
    \item Отчётность: еженедельные демо и отчёты
    \item GitHub с системой код-ревью
    \item Инструменты управления: Jira, Notion
    \item Управление рисками и изменениями — через Change Request
\end{itemize}


\section{Команда проекта}

{\footnotesize
\begin{longtable}
    \centering
    \begin{tabular}{|l|c|c|}
    \hline
    \textbf{Роль} & \textbf{Имя} & \textbf{Опыт} \\
    \hline
    Руководитель проекта & Кисельков Денис Андреевич & 5+ лет в ИТ-проектах - ИП \\
    \hline
    Разработчик Backend & Альшаеб Басель & 2+ г Django, API, Docker - ИП \\
    \hline
    Разработчик Frontend & Малыгин Даниил Александрович & 2+ г Vue.js, UI/UX - ИП \\
    \hline
    Маркетолог & Макаров Павел Михайлович & 6 лет - МДА \\
    \hline
    \end{tabular}

\end{longtable}
}

\section{Оценка проекта}

\subsection{Экспертная оценка}

\begin{tabular}{|p{10cm}|p{4cm}|}
\hline
\textbf{Этап} & \textbf{Оценка (чел-часов)} \\
    \hline
    Планирование & 70 \\
    \hline
    Технические задачи & 150 \\
    \hline
    Размещение & 80 \\
    \hline
    Организационные задачи & 160 \\
    \hline
    Поддержка и поддержка & 100 \\
    \hline
    Маркетинговые задачи & 260 \\
    \hline
    \textbf{Итого} & \textbf{820 чел-часов} \\
    \hline
\end{tabular}

\section{Стоимость и условия оплаты}

\textbf{Общая стоимость проекта:} \textbf{\₽1,640,000} \\
(расчёт: 2000 руб / чел-часов × 820 чел-дней)

\bigskip

\textbf{Условия оплаты:}
\begin{itemize}[noitemsep]
    \item 30\% — предоплата (492 000 руб)
    \item 40\% — после завершения основного функционала
    \item 30\% — после сдачи и развёртывания проекта
\end{itemize}

Возможна поэтапная сдача по актам приёмки.

\end{document}
