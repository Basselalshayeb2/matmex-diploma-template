% !TEX TS-program = xelatex
% !BIB program = bibtex
% !TeX spellcheck = ru_RU

% About magic macros see also
% https://tex.stackexchange.com/questions/78101/

% По умолчанию используется шрифт 14 размера.
% Если Вы не влезаете в лимит страниц и нужен 12-й шрифт,
% то уберите опцию [14pt]

\documentclass[14pt, russian]{matmex-diploma-custom}


% !TeX spellcheck = ru_RU
% !TEX root = vkr.tex
% Опциональные добавления используемых пакетов. Вполне может быть, что они вам не понадобятся, но в шаблоне приведены примеры их использования.
\usepackage{tikz} % Мощный пакет для создание рисунков, однако может очень сильно замедлять компиляцию
\usetikzlibrary{decorations.pathreplacing,calc,shapes,positioning,tikzmark}

% Библиотека для TikZ, которая генерирует отдельные файлы для каждого рисунка
% Позволяет ускорить компиляцию, однако имеет свои ограничения
% Например, ломает пример выделения кода в листинге из шаблона
% \usetikzlibrary{external}
% \tikzexternalize[prefix=figures/]

\newcounter{tmkcount}

\tikzset{
    use tikzmark/.style={
            remember picture,
            overlay,
            execute at end picture={
                    \stepcounter{tmkcount}
                },
        },
    tikzmark suffix={-\thetmkcount}
}

\usepackage{booktabs} % Пакет для верстки "более книжных" таблиц, вполне годится для оформления результатов
% В шаблоне есть команда \multirowcell, которой нужен этот пакет.
\usepackage{multirow}
\usepackage{siunitx} % для таблиц с единицами измерений

% Для названий стоит использовать \textsc{}
\newcommand{\OCaml}{\textsc{OCaml}}
\newcommand{\miniKanren}{\textsc{miniKanren}}
\newcommand{\BibTeX}{\textsc{BibTeX}}
\newcommand{\vsharp}{\textsc{V$\sharp$}}
\newcommand{\fsharp}{\textsc{F$\sharp$}}
\newcommand{\csharp}{\textsc{C\#}}
\newcommand{\GitHub}{\textsc{GitHub}}
\newcommand{\SMT}{\textsc{SMT}}

\definecolor{eclipseGreen}{RGB}{63,127,95}
% \lstdefinelanguage{ocaml}{
% keywords={@type, function, fun, let, in, match, with, when, class, type,
% nonrec, object, method, of, rec, repeat, until, while, not, do, done, as, val, inherit, and,
% new, module, sig, deriving, datatype, struct, if, then, else, open, private, virtual, include, success, failure,
% lazy, assert, true, false, end},
% sensitive=true,
% commentstyle=\small\itshape\ttfamily,
% keywordstyle=\ttfamily\bfseries, %\underbar,
% identifierstyle=\ttfamily,
% basewidth={0.5em,0.5em},
% columns=fixed,
% fontadjust=true,
% literate={->}{{$\to$}}3 {===}{{$\equiv$}}1 {=/=}{{$\not\equiv$}}1 {|>}{{$\triangleright$}}3 {\\/}{{$\vee$}}2 {/\\}{{$\wedge$}}2 {>=}{{$\ge$}}1 {<=}{{$\le$}} 1,
% morecomment=[s]{(*}{*)}
% }

\makeatletter
\@ifclassloaded{beamer}{
    %%% Обязательные пакеты
    %% Beamer
    \usepackage{beamerthemesplit}
    \usetheme{SPbGU}
    \beamertemplatenavigationsymbolsempty
    \usepackage{appendixnumberbeamer}

    %% Локализация
    \usepackage{fontspec}
    \setmainfont{CMU Serif}
    \setsansfont{CMU Sans Serif}
    \setmonofont{CMU Typewriter Text}
    %\setmonofont{Fira Code}[Contextuals=Alternate,Scale=0.9]
    %\setmonofont{Inconsolata}
    \usepackage{polyglossia}
    \setmainlanguage{russian}
    \setotherlanguage{english}

    %% Графика
    \usepackage{pdfpages} % Позволяет вставлять многостраничные pdf документы в текст

    % Математические окружения с русским названием
    \newtheorem{rutheorem}{Теорема}
    \newtheorem{ruproof}{Доказательство}
    \newtheorem{rudefinition}{Определение}
    \newtheorem{rulemma}{Лемма}
    \usepackage{fancyvrb}
}
{}
\makeatother

\usepackage[autostyle]{csquotes} % Правильные кавычки в зависимости от языка
\usepackage{totcount}
\usepackage{setspace}
\usepackage{amsmath, amsfonts, amssymb, amsthm, mathtools} % "Адекватная" работа с математикой в LaTeX



\begin{document}

% !TeX spellcheck = ru_RU
% !TEX root = vkr.tex


%% Если что-то забыли, при компиляции будут ошибки Undefined control sequence \my@title@<что забыли>@ru
%% Если англоязычная титульная страница не нужна, то ее можно просто удалить.
\filltitle{ru}{
    date = {27 марта 2025},
    %% Актуально только для курсовых/практик. ВКР защищаются не на кафедре а в ГЭК по направлению,
    %%   и к моменту защиты вы будете уже не в группе.
    chair              = {Программная инженерия},
    group              = {24.М71-мм},
    %
    %% Макрос filltitle ненавидит пустые строки, поэтому обязателен хотя бы символ комментария на строке
    %% Актуально всем.
    title              = {Вакансия и план собеседования},
    %
    %% Здесь указывается тип работы. Возможные значения:
    %%   production - производственная практика;
    %%   coursework - отчёт по курсовой работе (ОБРАТИТЕ ВНИМАНИЕ, у техпрога и ПИ нет курсовых, только практики);
    %%   practice - отчёт по учебной практике;
    %%   prediploma - отчёт по преддипломной практике;
    %%   master - ВКР магистра;
    %%   bachelor - ВКР бакалавра.
    type               = {groupwork},
    %
    %% Здесь указывается вид работы. От вида работы зависят критерии оценивания.
    %%   solution - «Решение». Обучающемуся поручили найти способ решения проблемы в области разработки программного обеспечения или теоретической информатики с учётом набора ограничений.
    %%   experiment - «Эксперимент». Обучающемуся поручили изучить возможности, достоинства и недостатки новой технологии, платформы, языка и т. д. на примере какой-то задачи.
    %%   production - «Производственное задание». Автору поручили реализовать потенциально полезное программное обеспечение.
    %%   comparison - «Сравнение». Обучающемуся поручили сравнить несколько существующих продуктов и/или подходов.
    %%   theoretical - «Теоретическое исследование». Автору поручили доказать какое-то утверждение, исследовать свойства алгоритма и т.п., при этом не требуя написания кода.
    kind               = {none},
    %
    author             = {Juniors},
    %
    %% Актуально только для ВКР. Указывается код и название направления подготовки. Типичные примеры:
    %%   02.03.03 \enquote{Математическое обеспечение и администрирование информационных систем}
    %%   02.04.03 \enquote{Математическое обеспечение и администрирование информационных систем}
    %%   09.03.04 \enquote{Программная инженерия}
    %%   09.04.04 \enquote{Программная инженерия}
    %% Те, что с 03 в середине --- бакалавриат, с 04 --- магистратура.
    specialty          = {09.04.04 \enquote{Математическое обеспечение и администрирование информационных систем}},
    %
    %% Актуально только для ВКР. Указывается шифр и название образовательной программы. Типичные примеры:
    %%   СВ.5162.2020 \enquote{Технологии программирования}
    %%   СВ.5080.2020 \enquote{Программная инженерия}
    %%   ВМ.5665.2022 \enquote{Математическое обеспечение и администрирование информационных систем}
    %%   ВМ.5666.2022 \enquote{Программная инженерия}
    %% Шифр и название программы можно посмотреть в учебном плане, по которому вы учитесь.
    %% СВ.* --- бакалавриат, ВМ.* --- магистратура. В конце --- год поступления (не обязательно ваш, если вы были в академе/вылетали).
    programme          = {ВМ.5666.2024 \enquote{Технологии программирования}},
    %
    %% Актуально всем.
    %% Должно умещаться в одну строчку, допускается использование сокращений, но без переусердствования,
    %% короткая строка с большим количеством сокращений выглядит странно
    %supervisorPosition = {проф. кафeдры системного программирования, д.ф.-м.н.,}, % Терехов А. Н.
    %supervisorPosition = {ст. преподаватель кафедры ИАС, к.~ф.-м.~н. (если есть),}, % Смирнов К. К.
    supervisorPosition = {},
    supervisor         = {},
    teacher            = {Тимохин Д. В.}
    %
    %% Актуально только для практик и курсовых. Если консультанта нет или он совпадает с научником, закомментировать или удалить вовсе.
    % consultantPosition = {должность, ООО \enquote{Место работы}, степень  (если есть),},
    % consultant         = {Консультант~К.~К.},
    %
    %% Актуально только для ВКР.
    % reviewerPosition   = {должность, ООО \enquote{Место работы}, степень (если есть),},
    % reviewer           = {Рецензент~Р.~Р.},
}

% Английский титульник нужен только для ВКР, остальные виды работ могут его смело игнорировать.
\filltitle{en}{
    chair              = {Advisor's chair},
    group              = {ХХ.BХХ-mm},
    title              = {Template for SPbU qualification works},
    type               = {bachelor},
    author             = {FirstName Surname},
    %
    %% Possible choices:
    %%   02.03.03 \foreignquote{english}{Software and Administration of Information Systems}
    %%   02.04.03 \foreignquote{english}{Software and Administration of Information Systems}
    %%   09.03.04 \foreignquote{english}{Software Engineering}
    %%   09.04.04 \foreignquote{english}{Software Engineering}
    %% Те, что с 03 в середине --- бакалавриат, с 04 --- магистратура.
    specialty          = {02.03.03 \foreignquote{english}{Software and Administration of Information Systems}},
    %
    %% Possible choices:
    %%   СВ.5162.2020 \foreignquote{english}{Programming Technologies}
    %%   СВ.5080.2020 \foreignquote{english}{Software Engineering}
    %%   ВМ.5665.2022 \foreignquote{english}{Software and Administration of Information Systems}
    %%   ВМ.5666.2022 \foreignquote{english}{Software Engineering}
    programme          = {СВ.5162.2020 \foreignquote{english}{Programming Technologies}},
    %
    %% Note that common title translations are:
    %%   кандидат наук --- C.Sc. (NOT Ph.D.)
    %%   доктор ... наук --- Sc.D.
    %%   доцент --- docent (NOT assistant/associate prof.)
    %%   профессор --- prof.
    supervisorPosition = {Sc.D, prof.},
    supervisor         = {S.S. Supervisor},
    %
    consultantPosition = {position at \foreignquote{english}{Company}, degree if present},
    consultant         = {C.C. Consultant},
    %
    reviewerPosition   = {position at \foreignquote{english}{Company}, degree if present},
    reviewer           = {R.R. Reviewer},
}

\maketitle

\section*{Задание}
\begin{itemize}
    \item Опишите вакансию для выбранного проекта и план распространения информации о вакансии.
    \item Опишите план собеседования для предложенной вакансии.
\end{itemize}

\section{Вакансия: Backend-разработчик}

\subsection*{Информация о компании}

\subsubsection*{Полное наименование}
\textbf{Общество с ограниченной ответственностью «Gym Tech».}

\subsubsection*{Основной вид деятельности}
Разработка программного обеспечения для автоматизации бизнес-процессов в сфере фитнеса и здравоохранения.

\textbf{Основные направления:}
\begin{itemize}[leftmargin=1.5em]
    \item Создание цифровых платформ для управления фитнес-клубами.
    \item Разработка клиентских и администраторских интерфейсов.
    \item Интеграция с платёжными и аналитическими сервисами.
    \item Разработка мобильных приложений.
\end{itemize}

\subsection*{Описание проекта}

Проект направлен на создание web-платформы, позволяющей организовать онлайн- и офлайн-тренировки, вести расписание, управлять абонементами, а также наладить коммуникацию между клиентами и тренерами.

\subsection*{Основные обязанности}

\begin{itemize}[leftmargin=1.5em]
    \item Разработка и поддержка бизнес-логики backend-системы.
    \item Проектирование и реализация базы данных (PostgreSQL).
    \item Интеграция с внешними API.
    \item Участие в технических обсуждениях и планировании спринтов.
    \item Поддержка внутренней документации проекта.
\end{itemize}

\subsection*{Требования к кандидату}

\begin{itemize}[leftmargin=1.5em]
    \item Уверенное знание PHP.
    \item Опыт работы с Laravel (от 1 года).
    \item Опыт проектирования и оптимизации баз данных (PostgreSQL).
    \item Понимание REST API и принципов безопасности.
    \item Приветствуется опыт мобильной разработки и знание Docker.
\end{itemize}

\subsection*{Условия}
\begin{itemize}[leftmargin=1.5em]
    \item Гибкий график (удалённо или в офисе).
    \item Оформление по договору ГПХ.
    \item Зарплата: от 200\,000 рублей в месяц.
    \item Испытательный срок — 1 месяц.
\end{itemize}

\subsection*{Контактная информация}

\begin{itemize}[leftmargin=1.5em]
    \item Телефон: +7 (495) 123-45-67
    \item Email: \texttt{contact@gym.tech}
    \item Веб-сайт: \url{https://gym.tech}
\end{itemize}

% \subsection{Информация о компании}

% \subsubsection{Полное наименование}
% Общество с ограниченной ответственностью «Gym Tech»

% \subsubsection{Юридический адрес}
% 123456, г. Москва, ул. Примерная, д. 10, офис 101

% \subsubsection{Основной вид деятельности}
% Разработка программного обеспечения для автоматизации бизнес-процессов в сфере фитнеса и здравоохранения.
% Основные направления:
% \begin{itemize}
%     \item Создание цифровых платформ для управления фитнес-клубами
%     \item Разработка клиентских и администраторских интерфейсов
%     \item Интеграция с платёжными и аналитическими сервисами
%     \item Разработка мобильных приложений
% \end{itemize}

% \subsection{Описание проекта}
% Проект направлен на создание web-платформы, позволяющей организовать онлайн- и офлайн-тренировки, вести расписание, управлять абонементами, а также наладить коммуникацию между клиентами и тренерами

% \subsection{Основные обязанности}
% \begin{itemize}
%     \item Разработка и поддержка бизнес-логики backend-системы
%     \item Проектирование и реализация базы данных (PostgreSQL)
%     \item Интеграция с внешними API
%     \item Участие в технических обсуждениях и планировании спринтов
%     \item Поддержка внутренней документации проекта
% \end{itemize}

% \subsection{Требования к кандидату}
% \begin{itemize}
%     \item Уверенное знание PHP
%     \item Опыт работы с Laravel (от 1 года)
%     \item Опыт проектирования и оптимизации баз данных (PostgreSQL)
%     \item Понимание REST API и принципов безопасности
%     \item Приветствуется опыт мобильной разработки и знание Docker
% \end{itemize}

% \subsection{Условия}
% \begin{itemize}
%     \item Гибкий график (удалённо или в офисе)
%     \item Оформление по ГПХ
%     \item Зарплата: от 200,000 рублей в месяц
%     \item Испытательный срок — 1 месяц
% \end{itemize}

% \subsection{Компенсационный пакет}
% Для привлечения и удержания квалифицированных специалистов в рамках проекта планируется предложить следующий компенсационный пакет:

% \begin{itemize}
%     \item \textbf{Конкурентоспособная заработная плата}, соответствующая уровню рынка и опыту кандидата;
%     \item \textbf{Оформление по ТК РФ} с полным социальным пакетом;
%     \item \textbf{Гибкий график работы} и возможность частичной удалёнки (по согласованию);
%     \item \textbf{Оплачиваемый отпуск} — 28 календарных дней в год;
%     \item \textbf{Оплачиваемые больничные листы};
%     \item \textbf{Возможность профессионального роста} и внутреннего обучения;
%     \item \textbf{Доступ в фитнес-центр или скидки на услуги} партнёрских спортивных клубов.
% \end{itemize}

% \subsection{Контактная информация}
% \begin{itemize}
%   \item Телефон: +7 (495) 123-45-67
%   \item Email: \texttt{contact@gym.tech}
%   \item Веб-сайт: \url{https://gym.tech}
% \end{itemize}

\subsection{Другие вакансии}

\subsubsection{Frontend-разработчик}
\begin{itemize}
  \item \textbf{Обязанности:} реализация пользовательского интерфейса, адаптивная вёрстка, интеграция с backend.
  \item \textbf{Требования:} JavaScript, опыт с Vue.js/Nuxt.js, знание основ UI-дизайна, PostgreSQL.
\end{itemize}

\subsubsection{Тестировщик}
\begin{itemize}
  \item \textbf{Обязанности:} написание тест-кейсов, выявление и фиксация багов, отчётность.
  \item \textbf{Требования:} знание методик тестирования, опыт работы с git и CI/CD.
\end{itemize}

\subsubsection{Маркетолог}
\begin{itemize}
  \item \textbf{Обязанности:} исследование рынка, разработка стратегии продвижения, анализ эффективности.
  \item \textbf{Требования:} опыт digital-маркетинга, аналитическое мышление, владение инструментами рекламы и SMM.
\end{itemize}

\subsection{План распространения информации о вакансии}
Информацию о вакансии планируется распространять через следующие каналы:

\begin{enumerate}
    \item \textbf{Сайты поиска работы:}
    \begin{itemize}
        \item hh.ru.
        \item Avito Работа.
        \item SuperJob.
    \end{itemize}

    \item \textbf{Специализированные сообщества:}
    \begin{itemize}
        \item Telegram-каналы, посвящённые разработке, фрилансу и стажировкам.
        \item Группы и паблики во «ВКонтакте», ориентированные на IT-вакансии.
    \end{itemize}

    \item \textbf{Места сосредоточения IT-аудитории:}
    \begin{itemize}
        \item Профессиональные порталы и форумы: Habr, iXBT.
        \item Тематические разделы и блоги: VC.ru, GeekBrains.
        \item Платформы для размещения портфолио: GitHub, Behance.
    \end{itemize}

    \item \textbf{Ярмарки вакансий:}
    \begin{itemize}
        \item Участие в студенческих карьерных мероприятиях вузов (МИРЭА, МГТУ им. Баумана и др.).
        \item Подача заявок на участие во всероссийских карьерных форумах и онлайн-хакатонах.
    \end{itemize}

    \item \textbf{Прямой поиск:}
    \begin{itemize}
        \item Поиск кандидатов через LinkedIn, GitHub, Stack Overflow Careers с индивидуальными приглашениями.
    \end{itemize}
\end{enumerate}

\section{План проведения собеседования}

Собеседование с кандидатами планируется проводить в четыре этапа, каждый из которых направлен на оценку ключевых аспектов профессиональной пригодности и соответствия культуре компании:

\textbf{Формат проведения:} очно, по адресу: \textbf{123456, г. Москва, ул. Примерная, д. 10, офис 101} — юридический адрес компании ООО «Gym Tech».

\begin{enumerate}
    \item \textbf{Ознакомительная часть:}
    На этом этапе устанавливается первый контакт с кандидатом и проводится краткое представление компании:
    \begin{itemize}
        \item Название организации — ООО «Gym Tech».
        \item Основное направление деятельности — разработка цифровых решений для фитнеса и здравоохранения.
        \item Основные достижения и реализованные проекты.
        \item Актуальный проект, в рамках которого открыта вакансия.
        \item Целевая аудитория проекта и его специфика.
    \end{itemize}

    \item \textbf{Техническая часть:}
    Этап, направленный на оценку профессиональной компетентности кандидата:
    \begin{itemize}
        \item Общие сведения о кандидате (личность, образование).
        \item Профессиональные знания и навыки (языки программирования, инструменты).
        \item Опыт работы (в том числе индивидуальные или командные проекты).
        \item Достижения и кейсы.
        \item Коммуникационные способности.
        \item Способность и мотивация работать в команде.
    \end{itemize}

    \item \textbf{Организационная часть:}
    Задаются вопросы, касающиеся условий работы и ожиданий кандидата:
    \begin{itemize}
        \item Причины отклика на вакансию.
        \item Ожидаемые перспективы от проекта.
        \item Готовность к полному рабочему дню и предлагаемой зарплате.
        \item Возможные сроки выхода на работу.
    \end{itemize}

    \item \textbf{Завершающая часть:}
    Финальный этап интервью:
    \begin{itemize}
        \item Ответы на оставшиеся вопросы кандидата.
        \item Согласование способа дальнейшей связи.
        \item Озвучивание условий и продолжительности испытательного срока.
        \item Обсуждение особенностей трудового оформления.
    \end{itemize}
\end{enumerate}

\end{document}
