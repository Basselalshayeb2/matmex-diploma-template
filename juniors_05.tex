% !TEX TS-program = xelatex
% !BIB program = bibtex
% !TeX spellcheck = ru_RU

% About magic macros see also
% https://tex.stackexchange.com/questions/78101/

% По умолчанию используется шрифт 14 размера.
% Если Вы не влезаете в лимит страниц и нужен 12-й шрифт,
% то уберите опцию [14pt]

\documentclass[14pt, russian]{matmex-diploma-custom}


\input{preamble.tex}

\begin{document}

\input{junior_05_title.tex}
\maketitle

\section*{Задание}
\begin{itemize}
    \item Опишите вакансию для выбранного проекта и план распространения информации о вакансии.
    \item Опишите план собеседования для предложенной вакансии.
\end{itemize}

\section{Вакансия: Backend-разработчик}

\subsection*{Информация о компании}

\subsubsection*{Полное наименование}
\textbf{Общество с ограниченной ответственностью «Gym Tech».}

\subsubsection*{Основной вид деятельности}
Разработка программного обеспечения для автоматизации бизнес-процессов в сфере фитнеса и здравоохранения.

\textbf{Основные направления:}
\begin{itemize}[leftmargin=1.5em]
    \item Создание цифровых платформ для управления фитнес-клубами.
    \item Разработка клиентских и администраторских интерфейсов.
    \item Интеграция с платёжными и аналитическими сервисами.
    \item Разработка мобильных приложений.
\end{itemize}

\subsection*{Описание проекта}

Проект направлен на создание web-платформы, позволяющей организовать онлайн- и офлайн-тренировки, вести расписание, управлять абонементами, а также наладить коммуникацию между клиентами и тренерами.

\subsection*{Основные обязанности}

\begin{itemize}[leftmargin=1.5em]
    \item Разработка и поддержка бизнес-логики backend-системы.
    \item Проектирование и реализация базы данных (PostgreSQL).
    \item Интеграция с внешними API.
    \item Участие в технических обсуждениях и планировании спринтов.
    \item Поддержка внутренней документации проекта.
\end{itemize}

\subsection*{Требования к кандидату}

\begin{itemize}[leftmargin=1.5em]
    \item Уверенное знание PHP.
    \item Опыт работы с Laravel (от 1 года).
    \item Опыт проектирования и оптимизации баз данных (PostgreSQL).
    \item Понимание REST API и принципов безопасности.
    \item Приветствуется опыт мобильной разработки и знание Docker.
\end{itemize}

\subsection*{Условия}
\begin{itemize}[leftmargin=1.5em]
    \item Гибкий график (удалённо или в офисе).
    \item Оформление по договору ГПХ.
    \item Зарплата: от 200\,000 рублей в месяц.
    \item Испытательный срок — 1 месяц.
\end{itemize}

\subsection*{Контактная информация}

\begin{itemize}[leftmargin=1.5em]
    \item Телефон: +7 (495) 123-45-67
    \item Email: \texttt{contact@gym.tech}
    \item Веб-сайт: \url{https://gym.tech}
\end{itemize}

% \subsection{Информация о компании}

% \subsubsection{Полное наименование}
% Общество с ограниченной ответственностью «Gym Tech»

% \subsubsection{Юридический адрес}
% 123456, г. Москва, ул. Примерная, д. 10, офис 101

% \subsubsection{Основной вид деятельности}
% Разработка программного обеспечения для автоматизации бизнес-процессов в сфере фитнеса и здравоохранения.
% Основные направления:
% \begin{itemize}
%     \item Создание цифровых платформ для управления фитнес-клубами
%     \item Разработка клиентских и администраторских интерфейсов
%     \item Интеграция с платёжными и аналитическими сервисами
%     \item Разработка мобильных приложений
% \end{itemize}

% \subsection{Описание проекта}
% Проект направлен на создание web-платформы, позволяющей организовать онлайн- и офлайн-тренировки, вести расписание, управлять абонементами, а также наладить коммуникацию между клиентами и тренерами

% \subsection{Основные обязанности}
% \begin{itemize}
%     \item Разработка и поддержка бизнес-логики backend-системы
%     \item Проектирование и реализация базы данных (PostgreSQL)
%     \item Интеграция с внешними API
%     \item Участие в технических обсуждениях и планировании спринтов
%     \item Поддержка внутренней документации проекта
% \end{itemize}

% \subsection{Требования к кандидату}
% \begin{itemize}
%     \item Уверенное знание PHP
%     \item Опыт работы с Laravel (от 1 года)
%     \item Опыт проектирования и оптимизации баз данных (PostgreSQL)
%     \item Понимание REST API и принципов безопасности
%     \item Приветствуется опыт мобильной разработки и знание Docker
% \end{itemize}

% \subsection{Условия}
% \begin{itemize}
%     \item Гибкий график (удалённо или в офисе)
%     \item Оформление по ГПХ
%     \item Зарплата: от 200,000 рублей в месяц
%     \item Испытательный срок — 1 месяц
% \end{itemize}

% \subsection{Компенсационный пакет}
% Для привлечения и удержания квалифицированных специалистов в рамках проекта планируется предложить следующий компенсационный пакет:

% \begin{itemize}
%     \item \textbf{Конкурентоспособная заработная плата}, соответствующая уровню рынка и опыту кандидата;
%     \item \textbf{Оформление по ТК РФ} с полным социальным пакетом;
%     \item \textbf{Гибкий график работы} и возможность частичной удалёнки (по согласованию);
%     \item \textbf{Оплачиваемый отпуск} — 28 календарных дней в год;
%     \item \textbf{Оплачиваемые больничные листы};
%     \item \textbf{Возможность профессионального роста} и внутреннего обучения;
%     \item \textbf{Доступ в фитнес-центр или скидки на услуги} партнёрских спортивных клубов.
% \end{itemize}

% \subsection{Контактная информация}
% \begin{itemize}
%   \item Телефон: +7 (495) 123-45-67
%   \item Email: \texttt{contact@gym.tech}
%   \item Веб-сайт: \url{https://gym.tech}
% \end{itemize}

\subsection{Другие вакансии}

\subsubsection{Frontend-разработчик}
\begin{itemize}
  \item \textbf{Обязанности:} реализация пользовательского интерфейса, адаптивная вёрстка, интеграция с backend.
  \item \textbf{Требования:} JavaScript, опыт с Vue.js/Nuxt.js, знание основ UI-дизайна, PostgreSQL.
\end{itemize}

\subsubsection{Тестировщик}
\begin{itemize}
  \item \textbf{Обязанности:} написание тест-кейсов, выявление и фиксация багов, отчётность.
  \item \textbf{Требования:} знание методик тестирования, опыт работы с git и CI/CD.
\end{itemize}

\subsubsection{Маркетолог}
\begin{itemize}
  \item \textbf{Обязанности:} исследование рынка, разработка стратегии продвижения, анализ эффективности.
  \item \textbf{Требования:} опыт digital-маркетинга, аналитическое мышление, владение инструментами рекламы и SMM.
\end{itemize}

\subsection{План распространения информации о вакансии}
Информацию о вакансии планируется распространять через следующие каналы:

\begin{enumerate}
    \item \textbf{Сайты поиска работы:}
    \begin{itemize}
        \item hh.ru.
        \item Avito Работа.
        \item SuperJob.
    \end{itemize}

    \item \textbf{Специализированные сообщества:}
    \begin{itemize}
        \item Telegram-каналы, посвящённые разработке, фрилансу и стажировкам.
        \item Группы и паблики во «ВКонтакте», ориентированные на IT-вакансии.
    \end{itemize}

    \item \textbf{Места сосредоточения IT-аудитории:}
    \begin{itemize}
        \item Профессиональные порталы и форумы: Habr, iXBT.
        \item Тематические разделы и блоги: VC.ru, GeekBrains.
        \item Платформы для размещения портфолио: GitHub, Behance.
    \end{itemize}

    \item \textbf{Ярмарки вакансий:}
    \begin{itemize}
        \item Участие в студенческих карьерных мероприятиях вузов (МИРЭА, МГТУ им. Баумана и др.).
        \item Подача заявок на участие во всероссийских карьерных форумах и онлайн-хакатонах.
    \end{itemize}

    \item \textbf{Прямой поиск:}
    \begin{itemize}
        \item Поиск кандидатов через LinkedIn, GitHub, Stack Overflow Careers с индивидуальными приглашениями.
    \end{itemize}
\end{enumerate}

\section{План проведения собеседования}

Собеседование с кандидатами планируется проводить в четыре этапа, каждый из которых направлен на оценку ключевых аспектов профессиональной пригодности и соответствия культуре компании:

\textbf{Формат проведения:} очно, по адресу: \textbf{123456, г. Москва, ул. Примерная, д. 10, офис 101} — юридический адрес компании ООО «Gym Tech».

\begin{enumerate}
    \item \textbf{Ознакомительная часть:}
    На этом этапе устанавливается первый контакт с кандидатом и проводится краткое представление компании:
    \begin{itemize}
        \item Название организации — ООО «Gym Tech».
        \item Основное направление деятельности — разработка цифровых решений для фитнеса и здравоохранения.
        \item Основные достижения и реализованные проекты.
        \item Актуальный проект, в рамках которого открыта вакансия.
        \item Целевая аудитория проекта и его специфика.
    \end{itemize}

    \item \textbf{Техническая часть:}
    Этап, направленный на оценку профессиональной компетентности кандидата:
    \begin{itemize}
        \item Общие сведения о кандидате (личность, образование).
        \item Профессиональные знания и навыки (языки программирования, инструменты).
        \item Опыт работы (в том числе индивидуальные или командные проекты).
        \item Достижения и кейсы.
        \item Коммуникационные способности.
        \item Способность и мотивация работать в команде.
    \end{itemize}

    \item \textbf{Организационная часть:}
    Задаются вопросы, касающиеся условий работы и ожиданий кандидата:
    \begin{itemize}
        \item Причины отклика на вакансию.
        \item Ожидаемые перспективы от проекта.
        \item Готовность к полному рабочему дню и предлагаемой зарплате.
        \item Возможные сроки выхода на работу.
    \end{itemize}

    \item \textbf{Завершающая часть:}
    Финальный этап интервью:
    \begin{itemize}
        \item Ответы на оставшиеся вопросы кандидата.
        \item Согласование способа дальнейшей связи.
        \item Озвучивание условий и продолжительности испытательного срока.
        \item Обсуждение особенностей трудового оформления.
    \end{itemize}
\end{enumerate}

\end{document}
