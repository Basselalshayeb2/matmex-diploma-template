% !TeX spellcheck = ru_RU
% !TEX root = stommis.tex

\section{Анализ существующих решений}

Существующие системы голосового ввода в медицинских информационных системах в большинстве случаев ограничиваются задачей преобразования речи в текст. При этом распознанная речь сохраняется в неизменном виде без выполнения семантического анализа и без привязки к конкретным элементам медицинской карты пациента.

В отличие от данных решений, предлагаемый в работе подход предполагает многоэтапную обработку голосового ввода. На первом этапе выполняется фильтрация звукового потока и выделение речевых команд. На втором этапе осуществляется распознавание речи, после чего применяется семантический анализ с использованием языковых моделей для определения типа команды и соответствующего действия в системе.

Таким образом, система ориентирована не на сохранение речи как текстового комментария, а на автоматизированное заполнение структурированных данных медицинской карты, включая выбор диагнозов, симптомов и процедур из заранее определённых медицинских справочников.
