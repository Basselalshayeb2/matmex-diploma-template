% !TeX spellcheck = ru_RU
% !TEX root = stommis.tex

% есть цель,

% есть вклад,

% есть результаты,

% есть направления развития.
\section{Заключение}
\label{sec:conclusion}

В работе разработан и реализован прототип голосового управления для стоматологической ИС, ориентированный на минимизацию ресурсоёмкости и стоимости распознавания за счёт предварительной фильтрации аудиопотока на клиенте.

Получены следующие результаты:
\begin{itemize}
    \item Реализован клиентский модуль захвата аудио в браузере и определения участков с речью/без речи в реальном времени на основе браузерного VAD (\texttt{@ricky0123/vad-web}) (результат к задаче~№1, раздел~\ref{subsec:task1}).
    \item Разработана логика активации и выделения команд (wake word + завершение по паузе), позволяющая отправлять на сервер только командные сегменты, тем самым снижая сетевую нагрузку и стоимость обработки (результат к задаче~№2, раздел~\ref{subsec:task2}).
    \item Выполнена интеграция с серверным контуром на Flask: приём сегмента как поля \texttt{file}, приведение формата аудио через \texttt{ffmpeg}, распознавание речи Yandex SpeechKit и преобразование текста команды в структурированное действие для заполнения карты пациента в STOMMIS (результат к задаче~№3, раздел~\ref{subsec:task3}).
    \item Предложен дизайн эксперимента и набор метрик для оценки ресурсоёмкости, задержки и качества выделения команд, позволяющий валидировать работоспособность решения в условиях клиники (раздел~\ref{sec:experiment}).
\end{itemize}

Дальнейшее развитие работы включает: расширение словаря команд и справочников предметной области, улучшение механизма ключевого слова (в том числе за счёт специализированного KWS‑модуля), а также проведение полномасштабной апробации на данных нескольких врачей с анализом устойчивости к шумам и индивидуальным особенностям речи.

% \textbf{Исходный код.}
% Репозиторий с исходным кодом и инструкциями по воспроизведению эксперимента будет предоставлен научному руководителю и комиссии (при необходимости с учётом ограничений доступа к медицинским данным).
