% !TeX spellcheck = ru_RU
% !TEX root = stommis.tex

% есть цель,

% есть вклад,

% есть результаты,

% есть направления развития.
\clearpage
\section{Заключение}
\label{sec:conclusion}

В рамках практики была разработана и исследована подсистема голосового управления для МИС «СТОММИС» в части заполнения зубной формулы и результатов первичного осмотра пациентов, ориентированная на использование голосовых команд в процессе лечения без необходимости постоянного взаимодействия с интерфейсом.
Ключевая идея решения --- многоэтапная обработка аудио: лёгкая сегментация речи в браузере (VAD), что снижает вычислительную и стоимостную нагрузку по сравнению с непрерывным распознаванием всего потока.

Основные результаты работы, соответствующие поставленным задачам:
\begin{itemize}
    \item реализован клиентский модуль захвата аудио в браузере и выделения участков речи/тишины (VAD) в реальном времени;
    \item реализована логика выделения команд и завершения сегмента по паузе (\enquote{2 секунды тишины});
    \item исследованы варианты активации голосового режима без участия рук: кнопка (\emph{push-to-talk}) и wake word; показано, что клиентский вариант wake word требует обучения и даёт высокий процент пропусков, поэтому в итоговой версии используется серверная проверка по ASR;
    \item реализована серверная обработка аудиосегментов: приём файла \texttt{file}, транскодирование (FFmpeg), распознавание речи через Yandex SpeechKit и интерпретация команды в структурированное действие внутри МИС с контролем допустимых операций;
    \item выполнена экспериментальная оценка и апробация прототипа, включая испытания в условиях СПб ГБУЗ «СП №8»; результаты подтверждают реализуемость подхода, при этом требуется дальнейшая стабилизация качества активации и интерпретации в условиях шума.
\end{itemize}

Направления дальнейшей работы включают расширение набора команд, улучшение устойчивости к шумам, а также уточнение механизма ограничения допустимых действий при интерпретации команд (в частности, формализацию схемы ответа и справочников предметной области).

% \textbf{Исходный код.}
% Репозиторий с исходным кодом и инструкциями по воспроизведению эксперимента будет предоставлен научному руководителю и комиссии (при необходимости с учётом ограничений доступа к медицинским данным).
