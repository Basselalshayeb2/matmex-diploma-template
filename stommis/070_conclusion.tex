% !TeX spellcheck = ru_RU
% !TEX root = stommis.tex

% есть цель,

% есть вклад,

% есть результаты,

% есть направления развития.
\section{Заключение}

В данной магистерской диссертации была рассмотрена задача разработки и оптимизации системы голосового управления, предназначенной для автоматизированного заполнения медицинской документации в процессе приёма пациента. Актуальность исследования обусловлена высокой нагрузкой на медицинский персонал и ограничениями традиционных интерфейсов взаимодействия в клинических условиях.

В ходе работы был предложен многоэтапный алгоритмический подход к обработке речевого сигнала, включающий детекцию речевой активности на стороне клиента, активацию системы по ключевому слову, распознавание речи и семантическую интерпретацию голосовых команд. Данный подход позволил существенно сократить объём обрабатываемых данных и снизить вычислительные затраты по сравнению с традиционными системами непрерывного распознавания речи.

Особое внимание в работе было уделено интерпретации команд врача и автоматическому заполнению структурированных элементов медицинской карты. Использование языковой модели в сочетании с ограничением пространства допустимых действий позволило обеспечить надёжную и контролируемую обработку команд без риска некорректного внесения данных.

Экспериментальные исследования подтвердили эффективность предложенного решения в условиях, приближенных к реальной стоматологической практике. Результаты показали снижение задержек обработки, уменьшение нагрузки на серверную часть системы и повышение удобства использования по сравнению с традиционными решениями голосового ввода.

Полученные результаты могут быть использованы при разработке медицинских информационных систем, ориентированных на повышение эффективности работы врачей и снижение когнитивной нагрузки при заполнении медицинской документации. В дальнейшем возможным направлением развития работы является расширение набора поддерживаемых команд, адаптация системы к другим медицинским специальностям и исследование методов локального распознавания речи.
