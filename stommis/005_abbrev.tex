% !TeX spellcheck = ru_RU
% !TEX root = stommis.tex

\section*{Список использованных сокращений}
\thispagestyle{withCompileDate}

\begin{description}
    \item[МИС] медицинская информационная система.
    \item[МИС «СТОММИС»] медицинская информационная система «СТОММИС» (объект внедрения в работе).
    \item[VAD] \textit{Voice Activity Detection} — детекция речевой активности (определение участков речи и «тишины»).
    \item[KWS] \textit{Keyword Spotting} — обнаружение ключевого слова (wake word).
    \item[wake word / trigger word] ключевое слово/фраза активации голосового режима.
    \item[ASR] \textit{Automatic Speech Recognition} — автоматическое распознавание речи (преобразование аудио в текст).
    \item[LLM] \textit{Large Language Model} — большая языковая модель, используемая для семантической интерпретации команд.
    \item[API] \textit{Application Programming Interface} — программный интерфейс взаимодействия компонентов.
    \item[JSON] \textit{JavaScript Object Notation} — формат обмена структурированными данными.
    \item[MFCC] \textit{Mel-Frequency Cepstral Coefficients} — кепстральные коэффициенты, часто используемые как признаки в задачах аудио.
    \item[FP/FN] \textit{False Positive / False Negative} — ложное срабатывание / пропуск (например, при детекции wake word).
    \item[FFmpeg] набор утилит для обработки мультимедиа (в работе используется для транскодирования аудио на сервере).
    \item[ONNX Runtime] библиотека выполнения моделей в формате ONNX (используется в клиентском VAD).
\end{description}
