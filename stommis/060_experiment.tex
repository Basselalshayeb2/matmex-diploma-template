% !TeX spellcheck = ru_RU
% !TEX root = stommis.tex

\section{Эксперимент}
\label{sec:experiment}

Цель эксперимента~--- проверить, что предложенная архитектура (VAD в браузере + отправка только командных сегментов) действительно уменьшает ресурсоёмкость и стоимость обработки по сравнению с наивным подходом непрерывного распознавания речи, сохраняя приемлемое качество выделения и распознавания команд.

\subsection{Условия эксперимента}
\label{subsec:exp_setup}

Эксперимент проводился в условиях, приближенных к рабочему месту врача:
\begin{itemize}
    \item \textbf{Клиент:} современный браузер (Chrome/Chromium), запуск веб‑страницы с VAD; доступ к микрофону через \texttt{getUserMedia}.
    \item \textbf{Сервер:} Python 3.x, Flask, модуль транскодирования через \texttt{ffmpeg}~\cite{FFmpeg}, модуль распознавания речи Yandex SpeechKit~\cite{YandexSpeechKit}.
    \item \textbf{Сеть:} локальная сеть клиники / тестовая сеть; фиксируются объём отправленных данных и задержки.
\end{itemize}

\textbf{Данные.}
Для оценки использовались записи (или воспроизведение через микрофон) типовых сценариев приёма: команды врача, обычная речь (объяснения пациенту), паузы, а также фоновые шумы кабинета.
Разметка включает:
\begin{itemize}
    \item временные границы команд (начало/конец);
    \item текстовую расшифровку команды;
    \item ожидаемое структурированное действие в системе (intent + параметры).
\end{itemize}
При сборе данных предполагается соблюдение требований приватности: данные деперсонифицированы, не содержат ФИО и иной чувствительной информации.

\subsection{Исследовательские вопросы}
\label{subsec:exp_rq}

\begin{description}
    \item[RQ1:] Насколько уменьшается объём аудио, отправляемого на сервер, и связанная с этим стоимость распознавания по сравнению с непрерывным ASR?
    \item[RQ2:] Каково качество выделения и распознавания команд (precision/recall) при использовании VAD + активации по ключевому слову?
    \item[RQ3:] Как влияет настройка параметров VAD (пороги, длительность «тишины» до завершения) на ложные срабатывания и задержку?
\end{description}

\subsection{Метрики}
\label{subsec:exp_metrics}

Для ответа на исследовательские вопросы используются следующие метрики:
\begin{itemize}
    \item \textbf{Доля переданного аудио} $S=\frac{T_{\text{sent}}}{T_{\text{total}}}$, где $T_{\text{sent}}$~--- суммарная длительность сегментов, отправленных на сервер, $T_{\text{total}}$~--- длительность исходного аудиопотока (RQ1).
    \item \textbf{Сетевой трафик} (байты/МБ), отправленный клиентом на сервер (RQ1).
    \item \textbf{Задержка} $L$~--- время от окончания произнесения команды до получения серверного ответа (медиана и 95‑й перцентиль) (RQ2, RQ3).
    \item \textbf{Качество выделения команд:} precision, recall и $F_1$ по факту обнаружения команды и её корректной интерпретации (RQ2).
    \item \textbf{Ресурсоёмкость клиента:} средняя загрузка CPU и потребление памяти браузером в режиме мониторинга и в режиме команды (RQ1).
\end{itemize}

\subsection{Результаты}
\label{subsec:exp_results}

Результаты эксперимента представляются в виде таблиц и графиков, сопоставляющих базовый подход и предложенное решение.

\textbf{Сравниваемые варианты.}
\begin{enumerate}
    \item \textbf{Baseline:} непрерывная запись и отправка аудио на сервер с последующим ASR.
    \item \textbf{Proposed:} VAD в браузере + активация + отправка только сегментов команд.
\end{enumerate}

\textbf{Шаблон таблицы результатов.}
В таблице~\ref{tab:exp_main} приводится рекомендуемый формат представления ключевых метрик (значения подставляются по результатам замеров).

\begin{table}[t]
    \centering
    \caption{Сравнение базового подхода и предложенного решения.}
    \label{tab:exp_main}
    \begin{tabular}{lrr}
        \hline
        Метрика & Baseline & Proposed \\
        \hline
        Доля переданного аудио $S$ & --- & --- \\
        Трафик, МБ & --- & --- \\
        Задержка $L$, мс (медиана) & --- & --- \\
        Precision команд & --- & --- \\
        Recall команд & --- & --- \\
        CPU в мониторинге, \% & --- & --- \\
        \hline
    \end{tabular}
\end{table}

\subsubsection{RQ1: ресурсоёмкость и стоимость}
Ожидается, что предложенное решение значительно уменьшит $T_{\text{sent}}$ и сетевой трафик за счёт исключения пауз и нерелевантной речи.
Дополнительно фиксируется загрузка CPU в браузере: в режиме мониторинга должна наблюдаться существенно меньшая нагрузка, чем при постоянной записи/кодировании и передаче аудио.

\subsubsection{RQ2: качество команд}
Для качества выделения команд важно учитывать два типа ошибок: ложные срабатывания (команда выделена, но её не было) и пропуски (команда произнесена, но не выделена/не распознана).
Оценка проводится на размеченных сценариях и отражается через precision/recall/$F_1$.

\subsubsection{RQ3: влияние параметров}
Проводится серия запусков при различных значениях порогов VAD и длительности паузы завершения.
Анализируется компромисс между задержкой (слишком длинная пауза увеличивает $L$) и устойчивостью (слишком короткая пауза может преждевременно обрывать команду).

\subsection{Обсуждение результатов}
\label{subsec:exp_discussion}

При интерпретации результатов важно учитывать особенности стоматологического кабинета: наличие фоновой речи, инструмента и оборудования, а также различия в манере диктовки у разных врачей.
Планируется отдельно анализировать типовые причины ошибок (например, смешение речи врача и пациента, тихая речь, высокая шумовая нагрузка) и влияние этих факторов на качество системы.

\subsection{Угрозы нарушения корректности}
\label{subsec:exp_threats}

Основные угрозы валидности эксперимента:
\begin{itemize}
    % \item \textbf{Смещение выборки (selection bias):} если сценарии собраны только от одного врача или в одном кабинете, результаты могут не переноситься на другие условия.
    \item \textbf{Разметка:} неточность временных границ команд и субъективность в определении «команды».
    \item \textbf{Сеть и инфраструктура:} задержки и потери в сети могут влиять на измеряемую латентность.
\end{itemize}

\subsection{Воспроизводимость}
\label{subsec:exp_repro}

Для воспроизводимости эксперимента в репозитории проекта фиксируются:
\begin{itemize}
    \item версии зависимостей клиентского и серверного модулей;
    \item параметры VAD, использованные в замерах;
    \item скрипты для запуска и сбора метрик (тайминг, объём отправленных данных);
    \item (при возможности) набор обезличенных тестовых аудиофайлов или инструкции по его получению.
\end{itemize}
