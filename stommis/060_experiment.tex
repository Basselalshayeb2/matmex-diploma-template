% !TeX spellcheck = ru_RU
% !TEX root = stommis.tex

\clearpage
\section{Эксперимент}
\label{sec:experiment}

Цель экспериментальной части --- оценить качество выделения речевых сегментов и команд, а также ресурсные и временные характеристики прототипа по сравнению с базовым подходом непрерывного распознавания аудиопотока.
Дополнительно выполнена апробация в реальных условиях стоматологической поликлиники СПб ГБУЗ «СП №8» (без детального раскрытия данных пациентов); результаты на текущем этапе нестабильны, но позволяют сделать предварительные выводы о применимости подхода.

\subsection{Исследовательские вопросы}
\label{subsec:exp_rq}

\begin{description}
    \item[RQ1:] Насколько уменьшается объём аудио, отправляемого на сервер, и связанная с этим стоимость распознавания по сравнению с непрерывным ASR?
    \item[RQ2:] Каково качество выделения и распознавания команд (precision/recall) при использовании VAD + активации по ключевому слову?
    \item[RQ3:] Как влияет настройка параметров VAD (пороги, длительность «тишины» до завершения) на ложные срабатывания и задержку?
    \item[RQ4:] Какова суммарная задержка \enquote{команда $\rightarrow$ действие в МИС} и какие компоненты конвейера вносят основной вклад?
\end{description}

\subsection{Метрики}
\label{subsec:exp_metrics}

Для оценки использовались следующие метрики:
\begin{itemize}
    \item \textbf{Latency} (мс): время от окончания произнесения команды до получения результата на клиенте; отдельно фиксируются компоненты: сегментация, передача, ASR, интерпретация, применение действия.
    \item \textbf{Нагрузка клиента:} влияние на отзывчивость UI и оценка потребления CPU при включенном VAD.
    \item \textbf{Нагрузка сервера:} число обращений к ASR и средняя длительность обрабатываемых аудиосегментов.
    \item \textbf{Качество активации:} доля ложных срабатываний (FP) и пропусков (FN) для wake word.
    \item \textbf{Качество сегментации:} доля корректно выделенных командных сегментов (по ручной разметке на небольшом наборе записей).
\end{itemize}

\subsection{Сценарии и базовая линия сравнения}
\label{subsec:setup}

Рассматривались два режима:
\begin{enumerate}
    \item \textbf{Базовый режим (baseline):} непрерывная запись и распознавание всего аудиопотока приёма (или всех сегментов речи), без VAD‑фильтрации на клиенте.
    \item \textbf{Предложенный режим:} непрерывный VAD в браузере (лёгкая сегментация) + активация голосового режима (wake word), после чего на сервер отправляется только сегмент команды.
\end{enumerate}

\subsection{Результаты и обсуждение}
\label{subsec:results}

Ключевые результаты сведены в табл.~\ref{tab:exp_results}. В рамках пилотной апробации наблюдаемая задержка от конца произнесения команды до получения структурированного результата на сервере составляет порядка 2--3 секунд; значения зависят от качества сети и акустической обстановки и требуют дальнейшей стабилизации.
Численные значения зависят от условий (микрофон, шум, сеть).

\begin{table}[t]
    \centering
    \caption{Сравнение базового и предложенного режимов (оценка на типовом сценарии)}
    \label{tab:exp_results}
    \begin{tabular}{|p{0.40\linewidth}|p{0.26\linewidth}|p{0.26\linewidth}|}
    \hline
    \textbf{Показатель} & \textbf{Baseline (непрерывное распознавание)} & \textbf{Предложенный режим (VAD + активация)} \\
    \hline
    Передаваемая длительность аудио за 10 минут приёма & $\approx$600 c (весь поток) & $\approx$18--30 c (6 команд по 3--5 c) \\
    \hline
    Число обращений к ASR & высокое (поток/чанки) & низкое (по подтверждённым сегментам) \\
    \hline
    Средняя длина одного распознаваемого фрагмента & 5--15 c (чанки потока) & 3--5 c (командный сегмент) \\
    \hline
    Задержка «конец команды $\rightarrow$ действие в МИС» & $\approx$2--3 c (но постоянно) & $\approx$2--3 c (по команде) \\
    \hline
    Риск обработки нерелевантной речи & высокий (комментарии/шум идут в ASR) & сниженный (фильтрация VAD и триггер) \\
    \hline
    Требования к взаимодействию врача & требуется управление записью/режимом & голосовое управление во время лечения (без рук) \\
    \hline
    \end{tabular}
    \end{table}

\subsubsection{RQ1: ресурсоёмкость и стоимость}
Ожидается, что предложенное решение значительно уменьшит $T_{\text{sent}}$ и сетевой трафик за счёт исключения пауз и нерелевантной речи.
Дополнительно фиксируется загрузка CPU в браузере: в режиме мониторинга должна наблюдаться существенно меньшая нагрузка, чем при постоянной записи/кодировании и передаче аудио.

\subsubsection{RQ2: качество команд}
Для качества выделения команд важно учитывать два типа ошибок: ложные срабатывания (команда выделена, но её не было) и пропуски (команда произнесена, но не выделена/не распознана).
Оценка проводится на размеченных сценариях и отражается через precision/recall/$F_1$.

\subsubsection{RQ3: влияние параметров}
Проводится серия запусков при различных значениях порогов VAD и длительности паузы завершения.
Анализируется компромисс между задержкой (слишком длинная пауза увеличивает $L$) и устойчивостью (слишком короткая пауза может преждевременно обрывать команду).

\subsubsection{RQ4: задержка и нагрузка}
Проводится серия замеров задержки и нагрузки на сервере при различных длительностях команд и числе обращений к ASR.
Анализируется влияние параметров VAD и триггера на активацию голосового режима.

В ходе апробации в СПб ГБУЗ «СП №8» было отмечено, что основной источник нестабильности связан с условиями акустики и вариативностью речи врачей, что особенно влияет на качество активации по ключевому слову.
Поэтому в текущей реализации предпочтение отдано серверной проверке wake word по ASR (см. раздел~\ref{subsec:wakeword}), обеспечивающей более низкую долю пропусков по сравнению с клиентским вариантом.

\subsection{Обсуждение результатов}
\label{subsec:exp_discussion}

При интерпретации результатов важно учитывать особенности стоматологического кабинета: наличие фоновой речи, инструмента и оборудования, а также различия в манере диктовки у разных врачей.
Планируется отдельно анализировать типовые причины ошибок (например, смешение речи врача и пациента, тихая речь, высокая шумовая нагрузка) и влияние этих факторов на качество системы.

\subsection{Угрозы нарушения корректности}
\label{subsec:exp_threats}

Основные угрозы валидности эксперимента:
\begin{itemize}
    % \item \textbf{Смещение выборки (selection bias):} если сценарии собраны только от одного врача или в одном кабинете, результаты могут не переноситься на другие условия.
    \item \textbf{Разметка:} неточность временных границ команд и субъективность в определении «команды».
    \item \textbf{Сеть и инфраструктура:} задержки и потери в сети могут влиять на измеряемую латентность.
\end{itemize}
