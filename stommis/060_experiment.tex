% !TeX spellcheck = ru_RU
% !TEX root = stommis.tex

\section{Экспериментальная оценка и анализ эффективности системы}

Целью экспериментальной части работы является оценка эффективности разработанной системы голосового управления в условиях, приближенных к реальной медицинской практике. В рамках экспериментов анализируются временные характеристики обработки команд, вычислительная нагрузка на клиентскую и серверную части, а также сравнительная эффективность предложенного подхода по отношению к традиционным решениям голосового ввода.


\subsection{Условия проведения экспериментов}

Экспериментальные исследования проводились в условиях, моделирующих рабочее место врача-стоматолога. Клиентская часть системы запускалась в современном веб-браузере на персональном компьютере, оснащённом стандартным микрофоном. Серверная часть системы функционировала в локальной вычислительной сети медицинского учреждения.

Для оценки устойчивости системы использовались различные сценарии речевого взаимодействия, включающие как корректные голосовые команды, так и нерелевантные речевые фрагменты, фоновые шумы и паузы в речи. Это позволило оценить работу системы в условиях, приближенных к реальной клинической среде.

\subsection{Временные характеристики обработки команд}

Одним из ключевых параметров системы голосового управления является задержка между произнесением команды врачом и её фактическим выполнением в медицинской информационной системе. Общая задержка обработки команды складывается из следующих компонентов:
\begin{itemize}
    \item время детекции речевой активности и завершения команды;
    \item время передачи аудиофрагмента на сервер;
    \item время распознавания речи;
    \item время семантической интерпретации команды;
    \item время внесения изменений в медицинскую карту.
\end{itemize}

Результаты измерений показали, что использование клиентской детекции речевой активности и активации по ключевому слову позволяет существенно сократить среднее время обработки команд за счёт уменьшения объёма передаваемых и обрабатываемых аудиоданных.


\subsection{Анализ вычислительной нагрузки}

Для оценки эффективности использования вычислительных ресурсов была проанализирована нагрузка на клиентскую и серверную части системы. На стороне клиента основная вычислительная нагрузка связана с выполнением алгоритма детекции речевой активности и предварительной обработки аудиосигнала.

Экспериментальные результаты показали, что использование VAD в браузере не оказывает заметного влияния на отзывчивость пользовательского интерфейса и не требует значительных вычислительных ресурсов. На серверной стороне нагрузка существенно снижается за счёт того, что распознавание речи и семантический анализ выполняются только для подтверждённых речевых команд.


\subsection{Сравнение с традиционным подходом}

Для оценки преимуществ предложенного решения было выполнено сравнение с традиционным подходом, при котором вся речь врача непрерывно передаётся на сервер и подвергается распознаванию.

Результаты сравнения показали, что использование многоэтапной фильтрации аудиосигнала позволяет существенно сократить количество обращений к сервису распознавания речи. В среднем количество распознаваемых аудиофрагментов уменьшается за счёт отбрасывания нерелевантной речи и фоновых звуков.

Кроме того, предложенный подход обеспечивает более высокую степень автоматизации, поскольку результатом обработки является не текстовая расшифровка речи, а структурированное изменение медицинской карты пациента.

\subsection{Качественная оценка удобства использования}

Качественная оценка системы проводилась на основе сценариев использования, характерных для стоматологической практики. Особое внимание уделялось возможности использования системы без физического взаимодействия с пользовательским интерфейсом.

Результаты оценки показали, что голосовое управление позволяет врачу сосредоточиться на лечебном процессе и снижает необходимость переключения внимания на экран компьютера. Использование голосовых команд особенно эффективно при заполнении структурированных разделов медицинской карты, требующих выбора значений из иерархических списков.

\subsection{Выводы по главе}

В результате экспериментальных исследований было показано, что предложенная система голосового управления обеспечивает эффективную обработку команд врача при умеренных вычислительных затратах. Использование детекции речевой активности и активации по ключевому слову позволяет снизить нагрузку на сервер и уменьшить задержки обработки.

Сравнение с традиционными решениями голосового ввода подтвердило целесообразность использования многоэтапной обработки речи и семантической интерпретации команд для автоматизированного заполнения медицинской документации.
