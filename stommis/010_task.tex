% !TeX spellcheck = ru_RU
% !TEX root = stommis.tex

\section{Постановка задачи}

\label{sec:task}

Целью работы является разработка прототипа подсистемы голосового управления для системы \textit{STOMMIS}, обеспечивающей выделение участков речи в аудиопотоке из браузера и последующую обработку голосовых команд при минимальной ресурсоёмкости и снижении затрат на распознавание речи.

Для достижения цели были поставлены следующие задачи:
\begin{enumerate}
    \item Разработать модуль захвата аудиопотока в браузере и реализовать выделение фрагментов речи и «тишины» (Voice Activity Detection) в режиме реального времени
    \label{subsec:task1}
    \item Реализовать логику выделения команды: определение начала команды по триггеру (кнопка или голосовой запуск) и окончания команды по паузе заданной длительности; подготовить аудиофрагмент к отправке на сервер
    \label{subsec:task2}
    \item Интегрировать клиентскую часть с серверным контуром обработки: приём аудиофайла, транскодирование при необходимости, распознавание речи через \textit{Yandex Speech} и интерпретация текста команд с использованием LLM для формирования структурированных действий в \textit{STOMMIS}
    \label{subsec:task3}
    \item Провести экспериментальную валидацию прототипа: оценить точность выделения фрагментов, задержки, потребление ресурсов и потенциальное снижение стоимости обработки по сравнению с вариантом постоянного распознавания аудиопотока
    \label{subsec:task4}
\end{enumerate}
