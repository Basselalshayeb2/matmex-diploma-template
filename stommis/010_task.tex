% !TeX spellcheck = ru_RU
% !TEX root = stommis.tex

\section{Постановка задачи}
\label{sec:task}

Целью работы является разработка и прототипирование инструмента голосового управления для стоматологической информационной системы, обеспечивающего выделение команд врача из аудиопотока при минимальном потреблении вычислительных и сетевых ресурсов.

Для достижения цели были поставлены следующие задачи:
\begin{enumerate}
    \item разработать клиентский модуль захвата аудио в браузере и определения участков с речью/без речи (VAD) в реальном времени (раздел~\ref{subsec:task1});
    \item реализовать механизм активации и выделения команд (ключевое слово + завершение команды по паузе), передающий на сервер только релевантные аудиосегменты (раздел~\ref{subsec:task2});
    \item интегрировать полученные аудиосегменты с серверным контуром распознавания речи и интерпретации команд для заполнения полей карты пациента в \textit{STOMMIS}, а также провести экспериментальную оценку решения (разделы~\ref{subsec:task3} и~\ref{sec:experiment}).
\end{enumerate}

Ограничения и допущения работы:
\begin{itemize}
    \item решение должно исполняться либо в браузере (JavaScript), либо на сервере в поликлинике (Python), без необходимости в специализированном оборудовании;
    \item обработка должна быть устойчивой к типовым помехам кабинета (разговоры, шум оборудования), а также не должна существенно увеличивать задержку взаимодействия врача с системой.
\end{itemize}
