% !TeX spellcheck = ru_RU
% !TEX root = stommis.tex

\clearpage

\section{Постановка задачи}
\label{sec:task}

Целью работы является разработка прототипа подсистемы голосового управления для МИС «СТОММИС», позволяющей врачу выполнять операции с медицинской картой \emph{в процессе лечения} с помощью голоса, без необходимости использовать клавиатуру/мышь и без сценария обязательного нажатия кнопки \emph{push-to-talk}.
При этом решение должно быть ресурсоэффективным: не выполнять постоянное распознавание всего аудиопотока и минимизировать сетевой трафик.

Для достижения цели были поставлены следующие задачи:
\begin{enumerate}
    \item Разработать клиентский модуль захвата аудио в браузере и определения участков с речью/без речи (VAD) в реальном времени (раздел~\ref{subsec:task1});
    \item Реализовать механизм начала и окончания команды без участия рук: исследовать варианты активации голосового режима (кнопка и ключевое слово), а также завершение команды по паузе заданной длительности; обеспечить формирование аудиосегмента и отправку на сервер только релевантных фрагментов (раздел~\ref{subsec:task2});
    \item Интегрировать обработку аудиосегментов с серверным контуром: приём аудиофайла, транскодирование при необходимости, распознавание речи (ASR) с использованием Yandex SpeechKit и интерпретация распознанного текста в структурированное действие для заполнения полей карты пациента в МИС «СТОММИС» (раздел~\ref{subsec:task3}).
    \item Провести экспериментальную валидацию прототипа, включая апробацию в реальных условиях (СПб ГБУЗ «СП №8»): оценить задержку, ресурсопотребление, частоту ложных срабатываний/пропусков и потенциальное снижение затрат по сравнению с вариантом постоянного распознавания аудиопотока (раздел~\ref{sec:experiment}).
\end{enumerate}

Ограничения и допущения работы:
\begin{itemize}
    \item Решение должно исполняться либо в браузере (JavaScript), либо на сервере в поликлинике (Python), без необходимости в специализированном оборудовании;
    \item Обработка должна быть устойчивой к типовым помехам кабинета (разговоры, шум оборудования), а также не должна существенно увеличивать задержку взаимодействия врача с системой.
\end{itemize}
