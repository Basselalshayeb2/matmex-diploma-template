% !TeX spellcheck = ru_RU
% !TEX root = stommis.tex

% Todo: Make the problem more focused on the specific case and our specific solution stommis.
\section*{Введение}
\thispagestyle{withCompileDate}

Цифровизация медицинских учреждений приводит к тому, что часть клинической работы переносится в информационные системы: врач фиксирует жалобы, анамнез, результаты осмотра и назначений в электронных формах.
В стоматологии это особенно заметно: итогом приёма является структурированная «карта пациента» (диагнозы, манипуляции, зубная формула, рекомендации), которая должна быть заполнена быстро и без потери качества.
На практике в системе \textit{STOMMIS} (используемой в партнёрской клинике) обнаружилась типовая проблема: во время осмотра врач работает в перчатках и часто не может удобно взаимодействовать с клавиатурой/мышью и деревьями выбора в интерфейсе, из‑за чего растёт время приёма и увеличивается количество пропусков и неточностей при заполнении карты.

Существующие решения на рынке нередко сводятся к диктовке «сплошного текста» в аудио/текстовый протокол или к полной записи консультации с последующей расшифровкой.
Подходы такого типа уменьшают объём ручного ввода, но не решают задачу \emph{структурирования} данных (заполнение конкретных полей карты) и создают дополнительную нагрузку: постоянное распознавание речи требует вычислительных ресурсов и, при использовании облачных сервисов, приводит к росту стоимости обработки и сетевого трафика.
Кроме того, при непрерывной обработке возрастают риски ложных срабатываний на разговоры пациента и фоновые звуки.

Целью работы является разработка и прототипирование инструмента голосового управления для стоматологической ИС, который позволяет выделять фрагменты речи и команды врача \emph{с минимальным потреблением ресурсов} и интегрируется с существующим серверным контуром распознавания и интерпретации команд.
Ключевое противоречие проекта: система должна потреблять минимум ресурсов и не выполнять дорогостоящее распознавание «всегда», но при этом надёжно фиксировать команды врача и не пропускать их.
В рамках работы рассматриваются две ключевые подзадачи: (1) автоматическое разделение звукового потока на участки с речью и «тишиной» (voice activity detection, VAD), (2) выделение в речи команд из ограниченного списка, запускаемых ключевым словом (wake word), и последующая интерпретация команд для заполнения полей карты пациента.

Предлагаемый подход основан на переносе части обработки на клиентскую сторону: аудиопоток захватывается в браузере средствами Web Audio API, далее лёгкий VAD‑модуль выделяет фрагменты речи и завершает сегмент после паузы заданной длительности.
На сервер передаются только релевантные сегменты (после активации ключевым словом), что уменьшает сетевую нагрузку и стоимость распознавания.
На сервере сегменты транскрибируются с помощью Yandex SpeechKit, после чего текстовая команда сопоставляется со структурированным действием (выбор диагноза/процедуры/поля карты) с использованием правил и/или LLM‑модуля.
Эффективность решения оценивается экспериментально по метрикам ресурсоёмкости, задержки и качества выделения команд.
