% !TeX spellcheck = ru_RU
% !TEX root = stommis.tex

% Todo: Make the problem more focused on the specific case and our specific solution stommis.
\section*{Введение}
\thispagestyle{withCompileDate}

Стоматологические клиники в России всё чаще используют специализированные медицинские информационные системы (МИС) для ведения карты пациента, фиксации осмотра и назначения лечения.
Практика показывает, что основная нагрузка на врача возникает именно на этапе внесения данных: врачу необходимо быстро и корректно заполнить структурированные поля (жалобы, анамнез, объективный статус, диагнозы, рекомендации), переключаясь между разделами интерфейса.
В системе \textit{STOMMIS} типовой сценарий подразумевает активную работу с клавиатурой и мышью, что неудобно в условиях приёма: врач часто работает в перчатках, руки заняты инструментами, а переключение в интерфейсе замедляет процесс и повышает вероятность пропусков.

Очевидное решение~--- использовать голосовой ввод.
Однако прямое «диктование всего приёма» даёт несколько проблем: (1) значительная часть речи не предназначена для карты (обсуждения, паузы, фоновые фразы), (2) постоянная потоковая отправка аудио на распознавание увеличивает стоимость и сетевую нагрузку, (3) качество распознавания ухудшается из-за шума в кабинете и перекрывающей речи.
Следовательно, возникает разрыв между желаемым UX («голосом управлять заполнением карты») и имеющимися подходами («либо всё записывать, либо постоянно распознавать поток»): требуется выделять только релевантные фрагменты и запускать распознавание в нужный момент при ограниченных ресурсах.

Цель данной работы~--- разработать и экспериментально оценить прототип подсистемы голосового управления для \textit{STOMMIS}, которая (а) анализирует аудиопоток, захватываемый в браузере, (б) определяет участки речи и «тишины», (в) по возможности выделяет голосовые команды из ограниченного списка, и (г) интегрируется с серверной частью, где выполняются распознавание речи и интерпретация команд для заполнения карты.
Ключевое противоречие проекта: система должна потреблять минимум ресурсов и не выполнять дорогостоящее распознавание «всегда», но при этом надёжно фиксировать команды врача и не пропускать их.

Подход работы основан на двухуровневой обработке: на клиенте в браузере постоянно работает лёгкий детектор голосовой активности (VAD), который определяет моменты появления речи и окончания высказывания по паузе; «дорогие» этапы включаются только при срабатывании триггера (кнопка или голосовой запуск).
После определения границ команды аудиофрагмент отправляется на сервер в виде файла, где при необходимости выполняется транскодирование, распознавание через \textit{Yandex Speech} и последующая интерпретация текста команд с использованием LLM для формирования структурированных действий в \textit{STOMMIS}.
Эффективность подхода оценивается по метрикам точности выделения фрагментов речи, частоте ложных срабатываний/пропусков, задержке, а также по ресурсным и стоимостным показателям по сравнению с базовым вариантом постоянного распознавания потока.
