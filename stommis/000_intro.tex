% !TeX spellcheck = ru_RU
% !TEX root = stommis.tex

% Todo: Make the problem more focused on the specific case and our specific solution stommis.
\section{Введение}
\thispagestyle{withCompileDate}

В последние годы в медицинских информационных системах наблюдается устойчивая тенденция к увеличению объёма цифровой документации, заполняемой врачами в процессе приёма пациентов. Несмотря на развитие электронных медицинских карт и специализированных программных решений, значительная часть рабочего времени врача по-прежнему затрачивается на ручной ввод данных, что снижает общую эффективность медицинского обслуживания и увеличивает когнитивную нагрузку на специалистов.

Особенно остро данная проблема проявляется в клинических условиях, где врач вынужден работать в перчатках, использовать стерильные инструменты и одновременно взаимодействовать с пациентом. В таких условиях использование традиционных средств ввода информации (клавиатура, мышь, сенсорный экран) является неудобным или невозможным, что актуализирует необходимость внедрения альтернативных интерфейсов взаимодействия с информационными системами.

\subsection{Актуальность исследования}

На российском рынке медицинских информационных систем в настоящее время доминируют решения, ориентированные на полную или частичную расшифровку речи врача с последующим сохранением полученного текста в электронную медицинскую карту. Такие системы, как правило, не выполняют структурный анализ речи и не интегрируют результаты распознавания с внутренними моделями данных медицинских учреждений.

В результате врач получает лишь текстовую запись устного комментария, которая требует последующей ручной обработки, классификации и распределения по соответствующим разделам медицинской карты. Данный подход не позволяет в полной мере автоматизировать процесс документооборота и не снижает нагрузку на медицинский персонал.

Особую специфику представляет стоматологическая практика, где осмотр пациента сопровождается постоянным использованием инструментов, ограниченной подвижностью врача и высокой концентрацией внимания. В этих условиях необходимость ручного управления программным обеспечением становится критическим фактором, снижающим эффективность работы.
