% !TeX spellcheck = ru_RU
% !TEX root = vkr.tex

\section*{Введение}
\thispagestyle{withCompileDate}

В последние годы в медицинских информационных системах наблюдается устойчивая тенденция к увеличению объёма цифровой документации, заполняемой врачами в процессе приёма пациентов. Несмотря на развитие электронных медицинских карт и специализированных программных решений, значительная часть рабочего времени врача по-прежнему затрачивается на ручной ввод данных, что снижает общую эффективность медицинского обслуживания и увеличивает когнитивную нагрузку на специалистов.

Особенно остро данная проблема проявляется в клинических условиях, где врач вынужден работать в перчатках, использовать стерильные инструменты и одновременно взаимодействовать с пациентом. В таких условиях использование традиционных средств ввода информации (клавиатура, мышь, сенсорный экран) является неудобным или невозможным, что актуализирует необходимость внедрения альтернативных интерфейсов взаимодействия с информационными системами.

\subsection{Актуальность исследования}

На российском рынке медицинских информационных систем в настоящее время доминируют решения, ориентированные на полную или частичную расшифровку речи врача с последующим сохранением полученного текста в электронную медицинскую карту. Такие системы, как правило, не выполняют структурный анализ речи и не интегрируют результаты распознавания с внутренними моделями данных медицинских учреждений.

В результате врач получает лишь текстовую запись устного комментария, которая требует последующей ручной обработки, классификации и распределения по соответствующим разделам медицинской карты. Данный подход не позволяет в полной мере автоматизировать процесс документооборота и не снижает нагрузку на медицинский персонал.

Особую специфику представляет стоматологическая практика, где осмотр пациента сопровождается постоянным использованием инструментов, ограниченной подвижностью врача и высокой концентрацией внимания. В этих условиях необходимость ручного управления программным обеспечением становится критическим фактором, снижающим эффективность работы.


\section{Постановка задачи}

Целью данной магистерской диссертации является разработка и исследование программной системы голосового управления, предназначенной для автоматизированного распознавания команд врача и структурированного заполнения электронной медицинской карты в процессе приёма пациента.

Для достижения поставленной цели в работе необходимо решить следующие задачи:

\begin{itemize}
    \item разработать механизм захвата и предварительной обработки звукового сигнала в браузере врача с минимальными вычислительными затратами;
    \item реализовать алгоритм детекции речевой активности, позволяющий выделять участки речи и условной тишины;
    \item разработать метод активации системы распознавания по голосовой команде без использования физических элементов управления;
    \item реализовать модуль распознавания речи с использованием облачных сервисов;
    \item разработать алгоритмы интерпретации распознанных команд и сопоставления их с элементами структуры медицинской карты;
    \item обеспечить интеграцию результатов обработки речи с существующей информационной системой медицинского учреждения.
\end{itemize}


\section{Анализ существующих решений}

Существующие системы голосового ввода в медицинских информационных системах в большинстве случаев ограничиваются задачей преобразования речи в текст. При этом распознанная речь сохраняется в неизменном виде без выполнения семантического анализа и без привязки к конкретным элементам медицинской карты пациента.

В отличие от данных решений, предлагаемый в работе подход предполагает многоэтапную обработку голосового ввода. На первом этапе выполняется фильтрация звукового потока и выделение речевых команд. На втором этапе осуществляется распознавание речи, после чего применяется семантический анализ с использованием языковых моделей для определения типа команды и соответствующего действия в системе.

Таким образом, система ориентирована не на сохранение речи как текстового комментария, а на автоматизированное заполнение структурированных данных медицинской карты, включая выбор диагнозов, симптомов и процедур из заранее определённых медицинских справочников.
