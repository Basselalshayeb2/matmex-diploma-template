% !TeX spellcheck = ru_RU
% !TEX root = stommis.tex

\section{Архитектура разрабатываемой системы}

Разрабатываемая система голосового управления предназначена для использования в условиях медицинского приёма и ориентирована на минимизацию вмешательства в рабочий процесс врача. В связи с этим архитектура системы проектировалась с учётом следующих требований:

\begin{itemize}
    \item минимальное использование вычислительных ресурсов на стороне клиента;
    \item возможность работы в браузере без установки дополнительного программного обеспечения;
    \item масштабируемость и независимость компонентов системы;
    \item возможность интеграции с существующими медицинскими информационными системами.
\end{itemize}

В соответствии с указанными требованиями система реализована в виде распределённого клиент-серверного решения и включает следующие основные компоненты:
\begin{itemize}
    \item клиентское веб-приложение врача;
    \item сервер обработки аудиоданных и команд;
    \item модуль распознавания речи;
    \item модуль семантической интерпретации команд;
    \item модуль интеграции с медицинской информационной системой.
\end{itemize}


\subsection{Клиентская часть системы}

Клиентская часть системы представляет собой веб-приложение, запускаемое в стандартном браузере врача. Основной задачей клиентского компонента является захват аудиопотока, предварительная фильтрация звуковых данных и передача релевантных фрагментов речи на сервер для дальнейшей обработки.

Для захвата звука используется стандартный Web API браузера, обеспечивающий доступ к микрофону пользователя. Обработка аудиопотока осуществляется в режиме реального времени с применением алгоритма детекции речевой активности (Voice Activity Detection, VAD), который позволяет разделять звуковой поток на участки речи и условной тишины.

Использование VAD на стороне клиента позволяет существенно снизить объём передаваемых данных и уменьшить нагрузку на сервер, поскольку на обработку отправляются только сегменты, содержащие речь врача.

\subsection{Механизм активации системы по голосовой команде}

Одной из ключевых особенностей разрабатываемой системы является отказ от физического управления процессом записи речи. В условиях медицинского приёма врач часто работает в перчатках и не имеет возможности взаимодействовать с интерфейсом посредством кнопок или сенсорных элементов.

Для решения данной проблемы в системе реализован механизм активации по голосовой команде. Клиентская часть непрерывно анализирует аудиопоток с использованием VAD, однако полноценная запись и обработка речи запускается только после обнаружения короткой голосовой команды активации.

Активация системы осуществляется по принципу ``ключевого слова''. При обнаружении короткого речевого сегмента он передаётся на сервер для проверки на соответствие ключевой фразе. В случае успешного распознавания команды активации система переходит в режим обработки основной команды врача.


\subsection{Серверная часть системы}

Серверная часть системы реализована в виде веб-сервиса и выполняет функции централизованной обработки аудиоданных и интерпретации команд. Сервер принимает аудиофрагменты, переданные с клиентской части, и выполняет их дальнейшую обработку в несколько этапов.

На первом этапе выполняется нормализация и, при необходимости, перекодирование аудиосигнала в формат, совместимый с используемым сервисом распознавания речи. Для обеспечения совместимости применяется конвертация входных данных в формат Ogg Opus.

На втором этапе выполняется распознавание речи с использованием внешнего сервиса автоматического распознавания речи. Результатом данного этапа является текстовая транскрипция голосовой команды врача.


\subsection{Семантическая интерпретация команд}

Полученная текстовая транскрипция не используется напрямую для заполнения медицинской карты. Вместо этого в системе реализован дополнительный этап семантического анализа команд, целью которого является определение типа действия и соответствующих параметров.

Для решения данной задачи используется языковая модель, способная интерпретировать естественный язык и сопоставлять распознанные фразы с заранее определёнными действиями системы. В результате семантического анализа команда врача преобразуется в структурированное представление, пригодное для автоматического заполнения полей медицинской карты.

Такой подход позволяет фильтровать нерелевантные фрагменты речи и снижает вероятность некорректного заполнения данных, что особенно важно в медицинских информационных системах.

\subsection{Отличие предлагаемого подхода от традиционных решений}

В отличие от традиционных систем голосового ввода, ориентированных на сохранение полной текстовой расшифровки речи врача, разрабатываемая система использует многоуровневую фильтрацию и интерпретацию команд.

Речь врача рассматривается не как конечный результат, а как источник управляющих сигналов, на основании которых выполняются структурированные изменения в медицинской карте пациента. Такой подход особенно эффективен в стоматологической практике, где врачу необходимо быстро фиксировать состояние пациента без отвлечения от лечебного процесса.
