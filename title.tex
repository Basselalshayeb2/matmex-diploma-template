% !TeX spellcheck = ru_RU
% !TEX root = vkr.tex

%% Если что-то забыли, при компиляции будут ошибки Undefined control sequence \my@title@<что забыли>@ru
%% Если англоязычная титульная страница не нужна, то ее можно просто удалить.
\filltitle{ru}{
    %% Актуально только для курсовых/практик. ВКР защищаются не на кафедре а в ГЭК по направлению,
    %%   и к моменту защиты вы будете уже не в группе.
    chair              = {Кафедра, на которой работает научник},
    group              = {ХХ.БХХ-мм},
    %
    %% Макрос filltitle ненавидит пустые строки, поэтому обязателен хотя бы символ комментария на строке
    %% Актуально всем.
    title              = {Шаблон отчёта по учебной практике},
    %
    %% Здесь указывается тип работы. Возможные значения:
    %%   production - производственная практика;
    %%   coursework - отчёт по курсовой работе (ОБРАТИТЕ ВНИМАНИЕ, у техпрога и ПИ нет курсовых, только практики);
    %%   practice - отчёт по учебной практике;
    %%   prediploma - отчёт по преддипломной практике;
    %%   master - ВКР магистра;
    %%   bachelor - ВКР бакалавра.
    type               = {practice},
    %
    %% Здесь указывается вид работы. От вида работы зависят критерии оценивания.
    %%   solution - «Решение». Обучающемуся поручили найти способ решения проблемы в области разработки программного обеспечения или теоретической информатики с учётом набора ограничений.
    %%   experiment - «Эксперимент». Обучающемуся поручили изучить возможности, достоинства и недостатки новой технологии, платформы, языка и т. д. на примере какой-то задачи.
    %%   production - «Производственное задание». Автору поручили реализовать потенциально полезное программное обеспечение.
    %%   comparison - «Сравнение». Обучающемуся поручили сравнить несколько существующих продуктов и/или подходов.
    %%   theoretical - «Теоретическое исследование». Автору поручили доказать какое-то утверждение, исследовать свойства алгоритма и т.п., при этом не требуя написания кода.
    kind               = {solution},
    %
    author             = {ФАМИЛИЯ Имя Отчество},
    %
    %% Актуально только для ВКР. Указывается код и название направления подготовки. Типичные примеры:
    %%   02.03.03 \enquote{Математическое обеспечение и администрирование информационных систем}
    %%   02.04.03 \enquote{Математическое обеспечение и администрирование информационных систем}
    %%   09.03.04 \enquote{Программная инженерия}
    %%   09.04.04 \enquote{Программная инженерия}
    %% Те, что с 03 в середине --- бакалавриат, с 04 --- магистратура.
    specialty          = {02.03.03 \enquote{Математическое обеспечение и администрирование информационных систем}},
    %
    %% Актуально только для ВКР. Указывается шифр и название образовательной программы. Типичные примеры:
    %%   СВ.5162.2020 \enquote{Технологии программирования}
    %%   СВ.5080.2020 \enquote{Программная инженерия}
    %%   ВМ.5665.2022 \enquote{Математическое обеспечение и администрирование информационных систем}
    %%   ВМ.5666.2022 \enquote{Программная инженерия}
    %% Шифр и название программы можно посмотреть в учебном плане, по которому вы учитесь.
    %% СВ.* --- бакалавриат, ВМ.* --- магистратура. В конце --- год поступления (не обязательно ваш, если вы были в академе/вылетали).
    programme          = {СВ.5162.2020 \enquote{Технологии программирования}},
    %
    %% Актуально всем.
    %% Должно умещаться в одну строчку, допускается использование сокращений, но без переусердствования,
    %% короткая строка с большим количеством сокращений выглядит странно
    %supervisorPosition = {проф. кафeдры системного программирования, д.ф.-м.н.,}, % Терехов А. Н.
    %supervisorPosition = {ст. преподаватель кафедры ИАС, к.~ф.-м.~н. (если есть),}, % Смирнов К. К.
    supervisorPosition = {доцент кафедры системного программирования, к.~ф.-м.~н. (если есть),},
    supervisor         = {Научник~Н.~Н.},
    %
    %% Актуально только для практик и курсовых. Если консультанта нет или он совпадает с научником, закомментировать или удалить вовсе.
    consultantPosition = {должность, ООО \enquote{Место работы}, степень  (если есть),},
    consultant         = {Консультант~К.~К.},
    %
    %% Актуально только для ВКР.
    reviewerPosition   = {должность, ООО \enquote{Место работы}, степень (если есть),},
    reviewer           = {Рецензент~Р.~Р.},
}

% Английский титульник нужен только для ВКР, остальные виды работ могут его смело игнорировать.
\filltitle{en}{
    chair              = {Advisor's chair},
    group              = {ХХ.BХХ-mm},
    title              = {Template for SPbU qualification works},
    type               = {bachelor},
    author             = {FirstName Surname},
    %
    %% Possible choices:
    %%   02.03.03 \foreignquote{english}{Software and Administration of Information Systems}
    %%   02.04.03 \foreignquote{english}{Software and Administration of Information Systems}
    %%   09.03.04 \foreignquote{english}{Software Engineering}
    %%   09.04.04 \foreignquote{english}{Software Engineering}
    %% Те, что с 03 в середине --- бакалавриат, с 04 --- магистратура.
    specialty          = {02.03.03 \foreignquote{english}{Software and Administration of Information Systems}},
    %
    %% Possible choices:
    %%   СВ.5162.2020 \foreignquote{english}{Programming Technologies}
    %%   СВ.5080.2020 \foreignquote{english}{Software Engineering}
    %%   ВМ.5665.2022 \foreignquote{english}{Software and Administration of Information Systems}
    %%   ВМ.5666.2022 \foreignquote{english}{Software Engineering}
    programme          = {СВ.5162.2020 \foreignquote{english}{Programming Technologies}},
    %
    %% Note that common title translations are:
    %%   кандидат наук --- C.Sc. (NOT Ph.D.)
    %%   доктор ... наук --- Sc.D.
    %%   доцент --- docent (NOT assistant/associate prof.)
    %%   профессор --- prof.
    supervisorPosition = {Sc.D, prof.},
    supervisor         = {S.S. Supervisor},
    %
    consultantPosition = {position at \foreignquote{english}{Company}, degree if present},
    consultant         = {C.C. Consultant},
    %
    reviewerPosition   = {position at \foreignquote{english}{Company}, degree if present},
    reviewer           = {R.R. Reviewer},
}
