% !TeX spellcheck = ru_RU
% !TEX root = vkr2.tex

\section{Обзор}
\label{sec:relatedworks}

В данном разделе рассматриваются существующие решения для выполнения сетевых эмуляций, используемые технологии и их сравнительный анализ.

\subsection{Существующей архитектуры Miminet}

Miminet использует централизованную модель управления эмуляцией, где:
\begin{itemize}
    \item Вся эмуляция выполняется внутри одного контейнера Docker;
    \item Виртуальные узлы и соединения создаются с использованием Linux network namespaces и эмулируемых Ethernet-интерфейсов;
    \item Эмуляция управляется через Python\cite{python} API Mininet, обеспечивая настройку узлов, коммутаторов и маршрутизаторов;
    \item Весь процесс выполняется в одном окружении, что ограничивает масштабируемость при увеличении количества узлов;
\end{itemize}

\subsubsection{Ограничения текущей реализации}

Несмотря на преимущества контейнеризированного подхода, текущая архитектура Miminet имеет ряд ограничений, которые могут снижать эффективность работы системы.

Первое ограничение связано с \textbf{масштабируемостью}. Поскольку вся эмуляция выполняется в одном контейнере, невозможно динамически распределять нагрузку между несколькими процессами или узлами. При увеличении количества узлов нагрузка на вычислительные ресурсы хоста (CPU и RAM) возрастает, что приводит к увеличению времени выполнения эмуляции и ограничивает её масштабируемость.

- \textbf{Текущее время выполнения эмуляции}: Каждая эмуляция занимает примерно 10 секунд, с использованием 10\% CPU.

Второе ограничение касается \textbf{использования вычислительных ресурсов}. В текущей реализации все процессы эмуляции выполняются в одном потоке без задействования многопроцессорности. Это означает, что даже если хостовая машина имеет несколько доступных ядер процессора, они не используются эффективно, что снижает общую производительность.

- \textbf{Пример текущего использования ресурсов}: CPU загружен на 10\%, но при увеличении количества эмуляций и сложности сетевой топологии, эффективность использования CPU значительно снижается, что увеличивает время выполнения.

Третье ограничение относится к \textbf{сетевым аспектам}. Все контейнеры используют общий сетевой стек, что приводит к затруднениям при одновременном запуске нескольких эмуляций. Это создаёт потенциальные конфликты и снижает гибкость в управлении ресурсами сети. Кроме того, отсутствие явной сетевой изоляции делает невозможным параллельное выполнение эмуляций с разными конфигурациями.

\subsection{используемых технологий}

В данном проекте применяются следующие технологии:

\begin{itemize}
    \item \textbf{Python\cite{python}} — основной язык программирования проекта, благодаря своей читаемости, широкому сообществу и множеству библиотек.
    \item \textbf{Flask\cite{flask}} — лёгкий веб-фреймворк на Python, используемый для обработки HTTP-запросов, маршрутизации и взаимодействия с базой данных.
    \item \textbf{Jinja2\cite{jinja2}} — шаблонизатор, встроенный во Flask, позволяющий генерировать HTML-страницы на основе Python-данных.
    \item \textbf{Mininet\cite{mininet}} — инструмент для эмуляции компьютерных сетей. Используется для создания виртуальных узлов, маршрутизаторов и соединений между ними.
    \item \textbf{Docker\cite{docker}} — система контейнеризации, обеспечивающая изоляцию симуляций и упрощающее развертывание компонентов проекта.
    \item \textbf{RabbitMQ\cite{rabbitmq}} и \textbf{Celery\cite{celery}} — инструменты для управления задачами и реализации очередей. Используются для распределения симуляций между контейнерами.
    \item \textbf{SQLite\cite{sqlite}} — встроенная реляционная база данных, применяемая для хранения конфигураций и результатов симуляций.
    \item \textbf{SQLAlchemy\cite{sqlalchemy}} — ORM для Python, позволяющая удобно работать с базой данных на уровне объектов.
    \item \textbf{JavaScript\cite{javascript}}, \textbf{jQuery\cite{jquery}} и \textbf{AJAX} — используются для создания интерактивного пользовательского интерфейса и динамической загрузки данных без перезагрузки страницы.
    \item \textbf{Bootstrap\cite{bootstrap}} — CSS-фреймворк, применяемый для создания адаптивного и современного дизайна интерфейса.
\end{itemize}

\subsection{Выводы}

Текущая архитектура Miminet ограничена возможностями масштабируемости и эффективного использования ресурсов хоста.
Для устранения этих проблем необходимо перераспределить нагрузку между несколькими контейнерами, обеспечив изоляцию сетевого стека и динамическое управление эмуляциями.
