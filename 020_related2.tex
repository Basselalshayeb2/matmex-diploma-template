% !TeX spellcheck = ru_RU
% !TEX root = vkr2.tex

\section{Обзор}
\label{sec:relatedworks}

В данном разделе рассматриваются существующие решения для выполнения сетевых симуляций, используемые технологии и их сравнительный анализ.

\subsection{существующей архитектуры Miminet}

Miminet — это симулятор компьютерных сетей, основанный на Mininet и адаптированный для работы в контейнеризированной среде Docker.
Его основная цель — предоставление удобного инструмента для моделирования сетевых взаимодействий между узлами, коммутаторами и маршрутизаторами в изолированной среде.

Miminet использует централизованную модель управления симуляцией, где:
\begin{itemize}
    \item Вся симуляция выполняется внутри одного контейнера Docker;
    \item Виртуальные узлы и соединения создаются с использованием Linux network namespaces и эмулируемых Ethernet-интерфейсов;
    \item Симуляция управляется через Python\cite{python} API Mininet, обеспечивая настройку узлов, коммутаторов и маршрутизаторов;
    \item Весь процесс выполняется в одном окружении, что ограничивает масштабируемость при увеличении количества узлов;
\end{itemize}

\subsubsection{Ограничения текущей реализации}

Несмотря на преимущества контейнеризированного подхода, текущая архитектура Miminet имеет ряд ограничений, которые могут снижать эффективность работы системы.

Первое ограничение связано с \textbf{масштабируемостью}. Поскольку вся симуляция выполняется в одном контейнере, невозможно динамически распределять нагрузку между несколькими процессами или узлами. При увеличении количества узлов нагрузка на вычислительные ресурсы хоста (CPU и RAM) возрастает, что приводит к увеличению времени выполнения симуляции и ограничивает её масштабируемость.

- \textbf{Текущее время выполнения симуляции}: Каждая симуляция занимает примерно 10 секунд, с использованием 10\% CPU. При увеличении числа узлов, время выполнения растёт, что приводит к возникновению узких мест в производительности.

Второе ограничение касается \textbf{использования вычислительных ресурсов}. В текущей реализации все процессы симуляции выполняются в одном потоке без задействования многопроцессорности. Это означает, что даже если хостовая машина имеет несколько доступных ядер процессора, они не используются эффективно, что снижает общую производительность.

- \textbf{Пример текущего использования ресурсов}: CPU загружен на 10\%, но при увеличении количества симуляций и сложности сетевой топологии, эффективность использования CPU значительно снижается, что увеличивает время выполнения.

Третье ограничение относится к \textbf{сетевым аспектам}. Все контейнеры используют общий сетевой стек, что приводит к затруднениям при одновременном запуске нескольких симуляций. Это создаёт потенциальные конфликты и снижает гибкость в управлении ресурсами сети. Кроме того, отсутствие явной сетевой изоляции делает невозможным параллельное выполнение симуляций с разными конфигурациями.

- \textbf{Пример снижения пропускной способности}: При текущей архитектуре, пропускная способность системы составляет X симуляций в час. При увеличении сложности сетевой топологии, эта пропускная способность снижается, что подчеркивает необходимость оптимизации.

\subsection{используемых технологий}

В данном проекте применяются следующие технологии:

    \begin{itemize}
        \item \textbf{Flask} — популярный фреймворк для разработки веб-приложений и API на языке Python\cite{python}. Его преимущества включают лёгкость освоения, поддержку шаблонов и гибкость в создании серверной логики. В данном проекте \textbf{Flask} используется для обработки HTTP-запросов и взаимодействия с базой данных.
        \item \textbf{Jinja2} — встроенный в Flask шаблонизатор, позволяющий динамически генерировать HTML-страницы. Использование \textbf{Jinja2} обеспечивает удобную интеграцию Python\cite{python}-кода в шаблоны, что упрощает создание и поддержку интерфейса приложения.
        \item \textbf{JavaScript} — язык программирования, применяемый для добавления интерактивности на веб-страницах. В проекте используется совместно с \textbf{AJAX} для динамической загрузки данных без необходимости полной перезагрузки страницы, что улучшает удобство и скорость работы приложения.
        \item \textbf{jQuery} — библиотека JavaScript, которая облегчает работу с элементами DOM, управление событиями и обработку запросов \textbf{AJAX}. Её использование также помогает обеспечить кроссбраузерную совместимость и упрощает клиентскую часть приложения.
        \item \textbf{Bootstrap} — библиотека для создания адаптивных пользовательских интерфейсов. Она предоставляет готовые компоненты и стили, ускоряя разработку страниц с современным и единообразным дизайном.
        \item \textbf{SQLite} — встроенная реляционная база данных, которая используется для хранения и управления данными проекта. Её лёгкость и простота конфигурации идеально подходят для текущих задач приложения.
        \item \textbf{SQLAlchemy} — ORM-библиотека для Python\cite{python}, упрощающая взаимодействие с реляционными базами данных. В проекте она обеспечивает удобную работу с данными, предоставляя инструменты для построения запросов и управления объектами.
    \end{itemize}

\subsection{Выводы}

Текущая архитектура Miminet ограничена возможностями масштабируемости и эффективного использования ресурсов хоста.
Для устранения этих проблем необходимо перераспределить нагрузку между несколькими контейнерами, обеспечив изоляцию сетевого стека и динамическое управление симуляциями.
